\newcommand{\tableofoptionsnew}[2]{
  \begin{tabularx}{\linewidth}{@{}>{\hsize=0.2\hsize \columncolor{gray!10}[0pt]}XX@{}}%
    \amatica\fontsize{19pt}{19pt}\selectfont\center
    \makebox[0pt]{\rotatebox{90}{#2}} &
    \begin{description}[font =\sffamily, after=\vspace*{-\topsep}, itemsep=3mm]
        \foreach \param/\descshort/\desclong in #1 {
        \item[\fontsize{15pt}{15pt}\selectfont{\param}] \textit{\descshort} \newline\desclong
        }
    \end{description}
  \end{tabularx}%
}

\newcommand{\tableofbuttons}[1]{
  \foreach \buttonname/\buttondesc in {#1} {
    \begin{tabularx}{\linewidth}{@{}>{\hsize=0.2\hsize \columncolor{gray!10}[0pt]}XX@{}}%
      \buttonname & \smallskip
      \begin{description}[font =\sffamily, after=\vspace*{-\topsep}]
        \item \textit{\buttondesc}
      \end{description}
    \end{tabularx}
  }
}

\ExplSyntaxOn

\NewDocumentCommand{\newlistx}{mm}{
  \clist_clear_new:c { l_walfi_list_#1_clist }
  \clist_set:cn  { l_walfi_list_#1_clist } { #2 }
}

\NewDocumentCommand{\looplist}{mm}
{
  \cs_gset:Nn \__walfi_list_loop:n { #2 }
  \clist_map_function:cN  { l_walfi_list_#1_clist } \__walfi_list_loop:n
}

\ExplSyntaxOff

\newcommand{\makerow}[2]{\textbf{#1} & #2 \\}
\newcommand{\makeitem}[2]{
\item[#1] #2}

\newcommand{\tableofset}[2]{
  \begin{tabularx}{\linewidth}{@{}>{\hsize=0.2\hsize \columncolor{gray!10}[0pt]}XX@{}}%
    #2 & \smallskip
    \begin{description}[font =\sffamily, after=\vspace*{-\topsep}]
        \looplist{#1}{\makeitem#2}
    \end{description}
  \end{tabularx}
  % }
}

% \usepackage[labelwidth]{enumitem}

\NewDocumentCommand{\specialitem}{O{white} O{black} m}{%
\item[\colorbox{#2}{\textcolor{#1}{{#3}}}]%
}

\newcommand{\makesupopt}[1]{%
  \begin{description}[leftmargin = !,labelwidth=6em]%
      \foreach \sublabel\subdesc in #1 {%
        \specialitem{\sublabel} \subdesc%
      }%
  \end{description}%
}

\usepackage{listofitems}
\usepackage{ragged2e} % allow ragged-right with proper hyphenation

\newcommand{\addtabs}[1]{%
  {\noexpandarg\StrSubstitute{#1}{E}{ZZZ}}%
}

\usetikzlibrary{arrows}
% \input{arrowsnew}

\newcommand{\makereferencecard}[4]{
  \begin{center}
    \begin{tikzpicture}[remember picture, node distance=1mm and 1mm, inner sep=0,
        every node/.style={font=\fontsize{8}{8}\selectfont, inner sep=0, outer sep=0, distance=1mm and 1mm},
        bignode/.style     = {font=\fontsize{7}{7}\selectfont},
        mymatrix/.style={matrix of nodes, nodes=typetag, outer ysep=2mm, row sep=1mm, anchor=south west},
        mycontainer/.style={inner sep=1ex},
        typetag/.style={bignode, draw=m_lightgray, text=m_green, inner sep=1mm,  minimum height=1cm}
      ]

      \node (main_img) at (0,0) {\includegraphics[width=5.5cm]{bg_image_the_centre.png}};

      \setsepchar{,}
      \readlist\itemdefs{#1}
      \readlist\itemgb{#2}
      \readlist\itempb{#3}
      \readlist\itemrb{#4}

      \matrix[
        matrix of nodes,
        % nodes=typetag,
        nodes={draw=m_lightgray, text=m_purple, minimum width=6cm, minimum height = 1.2cm, outer sep=0mm, align=right, text width=5.6cm},
        outer sep=2mm,
        inner sep=1mm,
        row sep=1mm,
        anchor=east,
      ] (pb_mxl) at (main_img.west){
        \setitemcol{\hfill \itempb[1]} \\
        \setitemcol{\hfill \itempb[2]} \\
        \setitemcol{\hfill \itempb[3]} \\
        \setitemcol{\hfill \itempb[4]} \\%
      };

      \matrix[
        matrix of nodes,
        % nodes=typetag,
        column 1/.style={anchor=base west},
        nodes={draw=m_lightgray, text=m_purple, minimum width=6cm, minimum height = 1.2cm, text width=5.6cm, align=left},
        outer sep=2mm,
        inner sep=1mm,
        row sep=1mm,
        anchor=west,
      ] (pb_mxr) at (main_img.east){
        \setitemcol{\itempb[8] \hfill} \\
        \setitemcol{\itempb[7] \hfill} \\
        \setitemcol{\itempb[6] \hfill} \\
        \setitemcol{\itempb[5] \hfill} \\%
      };
      \matrix[
        % mymatrix,
        matrix of nodes,
        column 1/.style={anchor=south west},
        nodes={draw=m_lightgray, text=m_green, minimum height = 1.0cm, align=right, inner sep=1mm},
        row 1/.style={minimum width=9.5cm, text width=9.2cm},
        row 2/.style={minimum width=9cm, text width=8.7cm},
        row 3/.style={minimum width=8.5cm, text width=8.2cm},
        outer ysep=2mm,
        outer xsep=1mm,
        inner sep=1mm,
        row sep=1mm,
        anchor=south west,
      ] (gb_mxl) at (pb_mxl.north west){
        \setitemcol{\hfill \itemgb[3]} \\
        \setitemcol{\hfill \itemgb[2]} \\
        \setitemcol{\hfill \itemgb[1]} \\%
      };
      \matrix[
        matrix of nodes,
        column 1/.style={anchor=south west},
        nodes={draw=m_lightgray, text=m_green, minimum height = 1.0cm, align=left, inner sep=1mm},
        row 1/.style={minimum width=8.5cm, text width=8.2cm},
        outer ysep=2mm,
        outer xsep=1mm,
        inner sep=1mm,
        row sep=1mm,
        anchor=north east
      ] (gb_mxr) at (gb_mxl.north-|pb_mxr.east){
        \setitemcol{\itemgb[4] \hfill} \\%
      };

      \matrix[
        matrix of nodes,
        % nodes=typetag,
        column 1/.style={anchor=base west},
        nodes={draw=m_lightgray, text=m_orange, minimum width=5.75cm, minimum height = 2.0cm, text width=7.6cm, align=left, inner sep=1mm},
        outer sep=2mm,
        inner sep=1mm,
        anchor=east,
      ] (ob_mx) at ($(gb_mxr.south east)!0.5!(pb_mxr.north east)$){
        % ] (ob_mx) at (pb_mxr.north east){
        \setitemcol{\itemdefs[1] \hfill} \\%
      };

      \matrix[
        matrix of nodes,
        % nodes=typetag,
        column 1/.style={anchor=base west},
        nodes={draw=m_lightgray, text=m_red, minimum height = 1.0cm, align=right, inner sep=1mm},
        row 1/.style={minimum width=7.6cm, text width=7.4cm},
        row 2/.style={minimum width=8.1cm, text width=7.9cm},
        row 3/.style={minimum width=8.6cm, text width=8.4cm},
        row 4/.style={minimum width=9.1cm, text width=8.9cm},
        outer ysep=2mm,
        outer xsep=1mm,
        inner sep=1mm,
        row sep=1mm,
        anchor=north west,
      ] (rb_mxl) at (pb_mxl.south west){
        \setitemcol{\hfill \itemrb[1]} \\
        \setitemcol{\hfill \itemrb[2]} \\
        \setitemcol{\hfill \itemrb[3]} \\
        \setitemcol{\hfill \itemrb[4]} \\%
      };

      \matrix[
        matrix of nodes,
        column 1/.style={anchor=base east},
        nodes={draw=m_lightgray, text=m_red, minimum height = 1.0cm, align=left, inner sep=1mm},
        row 1/.style={minimum width=7.6cm, text width=7.4cm},
        row 2/.style={minimum width=8.1cm, text width=7.9cm},
        row 3/.style={minimum width=8.6cm, text width=8.4cm},
        row 4/.style={minimum width=9.1cm, text width=8.9cm},
        outer ysep=2mm,
        outer xsep=1mm,
        inner sep=1mm,
        row sep=1mm,
        anchor=north east,
      ] (rb_mxr) at (pb_mxr.south east){
        \setitemcol{\itemrb[8] \hfill} \\
        \setitemcol{\itemrb[7] \hfill} \\
        \setitemcol{\itemrb[6] \hfill} \\
        \setitemcol{\itemrb[5] \hfill} \\%
      };

      \selannotationcol{\itempb[1]}{m_purple}
      \draw[-{Circle},thick] (pb_mxl-1-1.center -|pb_mxl-1-1.east) +(0,-1.0em) -- (-1.83, 0.69); % node[sloped] {};
      \selannotationcol{\itempb[2]}{m_purple}
      \draw[-{Circle} ,thick] (pb_mxl-2-1.center -|pb_mxl-2-1.east) +(0,0.0em)-- (-1.54, -0.10); % node[sloped] {};
      \selannotationcol{\itempb[3]}{m_purple}
      \draw[-{Circle} ,thick] (pb_mxl-3-1.center -|pb_mxl-3-1.east) +(0,-0.5em)-- (-0.95, -0.20); % node[sloped] {};
      \selannotationcol{\itempb[4]}{m_purple}
      \draw[-{Circle} ,thick] (pb_mxl-4-1.center -|pb_mxl-4-1.east) +(0,1.0em)-- (-0.35, -0.25); % node[sloped] {};
      \selannotationcol{\itempb[8]}{m_purple}
      \draw[-{Circle} ,thick] (pb_mxr-1-1.center -|pb_mxr-1-1.west) +(0,-1.0em) -- (1.76, 0.69); % node[sloped] {};
      \selannotationcol{\itempb[7]}{m_purple}
      \draw[-{Circle} ,thick] (pb_mxr-2-1.center -|pb_mxr-2-1.west) +(0,0.0em)  -- (1.54, -0.10); % node[sloped] {};
      \selannotationcol{\itempb[6]}{m_purple}
      \draw[-{Circle} ,thick] (pb_mxr-3-1.center -|pb_mxr-3-1.west) +(0,-0.5em) -- (0.95, -0.20); % node[sloped] {};
      \selannotationcol{\itempb[5]}{m_purple}
      \draw[-{Circle} ,thick] (pb_mxr-4-1.center -|pb_mxr-4-1.west) +(0,1.0em)  -- (0.35, -0.25); % node[sloped] {};

      \draw[{Circle[scale=1.3]}-, m_orange, thick] (1.615,1.305) coordinate (b1) -- (b1 |- ob_mx-1-1.south);

      \selannotationcol{\itemgb[4]v}{m_green}
      \draw[{Circle}-, thick] (0.7,0.1) coordinate (b1) -- (b1 |- gb_mxr-1-1.south);
      \selannotationcol{\itemgb[3]}{m_green}
      \draw[{Circle}-, thick] (0.2,0.1) coordinate (b1) -- (b1 |- gb_mxl-1-1.south);
      \selannotationcol{\itemgb[2]}{m_green}
      \draw[{Circle}-, thick] (-0.25,0.1) coordinate (b1) -- (b1 |- gb_mxl-2-1.south);
      \selannotationcol{\itemgb[1]}{m_green}
      \draw[{Circle}-, thick] (-0.73,0.1) coordinate (b1) -- (b1 |- gb_mxl-3-1.south);

      \selannotationcol{\itemrb[5]}{mred}
      \draw[{Circle[scale=0.9]}-, thick] (0.23,-1.1) coordinate (b1) -- (b1 |- rb_mxr-4-1.north);
      \selannotationcol{\itemrb[6]}{mred}
      \draw[{Circle[scale=0.9]}-, thick] (0.725,-1.1) coordinate (b1) -- (b1 |- rb_mxr-3-1.north);
      \selannotationcol{\itemrb[7]i}{mred}
      \draw[{Circle[scale=0.9]}-, thick] (1.20,-1.1) coordinate (b1) -- (b1 |- rb_mxr-2-1.north);
      \selannotationcol{\itemrb[8]ii}{mred}
      \draw[{Circle[scale=0.9]}-, thick] (1.695,-1.1) coordinate (b1) -- (b1 |- rb_mxr-1-1.north);
      \selannotationcol{\itemrb[4]}{mred}
      \draw[{Circle[scale=0.9]}-, thick] (-0.23,-1.1) coordinate (b1) -- (b1 |- rb_mxl-4-1.north);
      \selannotationcol{\itemrb[3]}{mred}
      \draw[{Circle[scale=0.9]}-, thick] (-0.725,-1.1) coordinate (b1) -- (b1 |- rb_mxl-3-1.north);
      % \draw[{Circle}-, m_red, thick] (0.22,-1.1) coordinate (b1) -- (b1 |- rb_mxr-4-1.north);
      % \draw[{Circle}-, m_red, thick] (0.72,-1.1) coordinate (b1) -- (b1 |- rb_mxr-3-1.north);
      % \draw[{Circle}-, m_red, thick] (1.2,-1.1) coordinate (b1) -- (b1 |- rb_mxr-2-1.north);
      % \draw[{Circle}-, m_red, thick] (1.7,-1.1) coordinate (b1) -- (b1 |- rb_mxr-1-1.north);
      % \draw[{Circle}-, m_red, thick] (-0.22,-1.1) coordinate (b1) -- (b1 |- rb_mxl-4-1.north);
      % \draw[{Circle}-, m_red, thick] (-0.72,-1.1) coordinate (b1) -- (b1 |- rb_mxl-3-1.north);
      % \draw[{Circle}-, m_red, thick] (-1.2,-1.1) coordinate (b1) -- (b1 |- rb_mxl-2-1.north);
      % \draw[{Circle}-, m_red, thick] (-1.7,-1.1) coordinate (b1) -- (b1 |- rb_mxl-1-1.north);
    \end{tikzpicture}
  \end{center}
}

\newcommand\modnamelink[1]{\underline{\nameref{section:mod#1}}}
\newcommand{\modmakesection}[1]{
\section{\csname modnameshort#1\endcsname{ - }\csname modnamelong#1\endcsname}\label{section:mod#1}\def\modname{\csname modnameshort#1\endcsname{ - }\csname modnamelong#1\endcsname}}

%this one is used to show reference card inside user's manual
\newcommand{\showrefcardsection}[1]{
\subsection{Mapping of Controls}\csname refcard#1\endcsname}
\newcommand{\showrefcardsubsection}[1]{
\subsubsection{Mapping of Controls}\csname refcard#1\endcsname}
\newcommand{\modrefcard}[1]{\newpage\modmakesection{#1}\csname refcard#1\endcsname}

% \def\lfsbutmain{
%    {Edit}/{Invokes Shape Editor},
%    {Set}/{Open Settings},
%    {Inputs}/{Open Inputs},
%    {Back}/{Return to Patch Menu}
% }



\def\wtoset{
  {{Audio Output}/{オーディオ出力 - ステレオ}/{WTO の音声出力先を選択します。 ポリフォニーモードを含むすべてのオシレーターは、単一チャンネル (モノラルまたはステレオ) から出力されます。 \newline 参照 \underline{\nameref{section:audiooutputs}}}},
  {{Polyphony}/{WTOオシレーターのパラフォニーモード}/{
    \makesupopt{{
      {{1}/{パラフォニー無効、1つのWTOピッチコントロール}},
      {{2, 3, 4}/{ピッチの異なる 2 - 4 つのオシレーターのパラフォニー。 オシレーター1のピッチが V/OCT1 入力から来る場合、次のオシレーターのピッチは対応する V/OCT2 - V/OCT4 入力から来ます。}}%
    }}
  }},
  {{Glide Scaled}/{ピッチの距離によるグライドタイムのスケーリング}/{
    \makesupopt{{
      {{Off}/{グライド タイムは、どのノート間のピッチポルタメントでも同じになります。}},
      {{On}/{グライドタイムはピッチの距離によって変化します。 Input Glide Time (以下の「Inputs」を参照) によって設定された Glide Time は、1 オクターブのピッチ間の距離になります。 同じオクターブ内で演奏される音符の場合、次の音符のピッチを半音の分数で区切ります。}}%
    }}
  }},
  {{Warp Mode}/{\label{wtoset:warpmode}オシレーション時の波形を加工する機能}/{ワープ モードは、入力 \textbf{\textit{Warp mode}} を通じたる変調に従って波形を処理するために使用される追加機能を有します。 現在、1 つの機能のみが有効になっています。
  \newline
  \makesupopt{{
    {{Off}/{波形の処理なし}},
    {{FM}/{ワープモード入力が波形の周波数 (FM) を変調します}}%
  }}
  }}%
    % {{Running Mode}/{Select type of running mode of low frequency shape}/{
    %    \makesupopt{{
    %      {{Continuous}/{Oscillator keeps running and never stops. Gate High (Trigger) signal will reset oscillator phase to 0.}},
    %      % \item [\tikz{\node[text=white, fill=black] {Continuous};}] Oscillator keeps running and never stops. Gate High (Trigger) signal will reset oscillator phase to 0.\
    %      {{Trigger}/{Oscillator runs once upon the gate signal high and runs complete shape once and stops generation when phase of shape ends. Another trigger signal will reset phase to 0.}},
    %      {{Gate length}/{Oscillator runs as long as gate signal is high. Stops immediatelly when signal goes to low.}}%
    %   }}
    % }},
    % {{Timing Mode}/{Select timing mode for LFS}/{Timing for LFS can be either selected to be calculated in Hz for more time based experience or tied to BPM and measured in length of notes or bars.
    % \makesupopt{{
    %    {{BPM}/{Timing based on note duration}},
    %    {{Hz}/{Timing based on frequency}}%
    % }}
    % }},
    % {{Polarity}/{Polarity of output}/{
    %    \makesupopt{{
    %      {{Bipolar}/{Outputs shape in range -5V to +5V. Good for Ring Modulation or FM}},
    %      {{Unipolar}/{Outputs shape in range 0V to +5V. Good for VCA}}%
    %   }}
    % }}%
}
% \def\inpclock{
%    {{Clock}/{Clock for timing oscillator's phase duration in BPM mode (otherwise 120BPM is used)}/{}}
% }

\def\wtoinp{
  {{NOTE}/{オシレーターのピッチコントロール}/{Note は、オシレーターのピッチまたは周波数を制御します。
  \newline 参照 \underline{\nameref{section:pitchcontrol}}
  \makesupopt{{
    {{VOCT}/{ノートコントロール用の1V/Oct入力/}},
    {{OCT}/{オクターブ単位のチューニングノート +/- 8 オクターブ}},
    {{NOTE}/{半音単位のチューニング音 +12 半音 (1 オクターブ)}},
    {{FINE}/{セント +/- 半音単位のチューニング ノート}}%
  }}%
  }},
  {{Unison}/{ユニゾンモード(1~16ボイス)}/{ユニゾンモードを有効にし、並列で動作するオシレーターの数を制御します。}},
  {{Detune}/{デチューンオシレーター}/{ユニゾン モードでオシレーターをデチューン範囲で均等にデチューンします。 デチューンを最大に設定すると、スプレッド範囲は 2 半音になります。 オシレーターの数が奇数の場合、中央のオシレーターは常にノート入力のピッチに従います。 偶数のオシレーターの場合、ノートの後に続くオシレーターはなく、各オシレーターはセンター ノートから +/- 離調されます。 \newline $\bigstar$ \textit{Unison で 2 つのオシレーターを実行し、デチューンを最大 (1.0) に設定し、ピッチをノート D に設定すると、ノート C\# を再生するオシレーターとノート D\# を再生するオシレーターが 1つあります (実際にノート C を再生するオシレーターはありません)。}}}%
}
\def\wtoinpi{
  {{Wavetable}/{Wavetable position}/{ウェーブテーブルの波形の位置を調整します。 CV で制御すると、ウェーブテーブルに含まれる一連の波形をスキャンしてサウンドの質感を変更できます。}},
  {{Reset Osc}/{オシレーターCV入力のリセット}/{CV 入力に接続して高レベルに設定すると、オシレーターの位相が 0 にリセットされます。同期でオシレーターを実行するのに便利です。}},
  {{AM}/{Amplitude Modulation}/{振幅変調はサウンドレベルを変化させます。 Envelope の CV 信号と接続すると VCA として機能します。 LFO と接続すると、リング モジュレーターが作成されます。}},
  {{Detune WT}/{ユニゾン モードでオシレーターのウェーブテーブル位置をデチューンする}/{ユニゾン モードでは、このパラメーターは個々のオシレーターのウェーブテーブルの波形の位置をデチューンします。 各オシレーターは異なる波形を再現するため、よりテクスチャーのあるサウンドが作成されます。\newline$\bigstar$ \textit{3 つの波形 (トライアングル、サイン、スクエア) と 3 つのオシレーターのユニゾンと 1.0 に設定された Detune WT パラメーターを備えたウェーブテーブルで、各オシレーターは固有の波形を再生します。}}},
  {{Frequency Coarse}/{大きなステップでオシレータの周波数を調整}/{オシレータの周波数を 0Hz から 100Hz の範囲 (周波数モード) または 1/256 ノートから 8bar の間 (BPM/ノートデュレーションモード) で調整します。}},
  {{Frequency Fine}/{オシレーターの周波数を微調整する}/{非常に小さな段階でオシレーターの周波数を調整します。}},
  {{Glide Time}/{ポルタメント(ピッチグライド)タイム}/{ピッチの変更時に、あるピッチ(Note)から別のピッチに移る時間 。グライド時間は 0 秒から 1 秒の範囲です。 \newline $\blacksquare$ このパラメータは、*Glide Scaled* の設定の影響を受けます。 グライドスケールをオフに設定すると、2 つのノート間をグライドするのにかかる時間は一定であり、グライド タイムによって定義されます。グライドスケールをオンにすると、時間は演奏されるノートのピッチ距離によって決まります。}},
  {{Warp Mode}/{ワープモードは、ワープモードパラメータ設定で選択されたワープモード機能で使用されるCVを介したモジュレーションパラメータです。 \underline{\hyperref[wtoset:warpmode]{Warp Mode}} 参照}}%
 %  {{Trigger}/{Trigger CV input}/{Trigger or gate signal that initiates or resets oscillator}},
  %  {\inpclock},
  %  {{Frequency Coarse}/{Adjust Frequency of oscillator (coarse)}/{Adjusts frequency of oscillator in range of 0Hz to 100Hz (in frequency mode) or between 1/256 note and 8bar (in BPM/note duration mode)}},
  %  {{Frequency Fine}/{Fine tune frequency of oscillator}/{Adjust frequency of oscillator in very small steps}}%
}

    
    % {}/{}/{}%
    % {}/{}/{}%
    % {}/{}/{}%

\def\wtoout{
    {{OUT}/{WTO出力}/{ウェーブテーブルオシレーターの出力}},
    {{RST}/{オシレーターリセット}/{発振器の位相がリセットされるとhighに設定されます。}}%
}



\def\refcardwto{
  \makereferencecard{
    \textcolor{black}{\textbf{[回す]}} 現在のディレクトリ内のウェーブテーブル間を移動\newline\textcolor{black}{\textbf{[クリック]}} File Browser に入り、選択したウェーブテーブルに変更し、ブラウジングディレクトリを変更\newline\textcolor{black}{\textbf{[長押し]}} Wavetable Editorに入る
    }{
      \textcolor{black}{\textbf{[2D/3D]}} ウェーブテーブルの表示切り替え,
      \textcolor{black}{\textbf{[SET]}} モジュールの設定,
      \textcolor{black}{\textbf{[Inputs]}} モジュールの入力を構成する,
      \textcolor{black}{\textbf{[Back]}} パッチメニューに戻る,
    }{
      \textcolor{black}{\textbf{[WT]}} ウェーブテーブル内のウェーブの位置。 CV によって制御されている場合は、テーブルを移動します。
      \textcolor{black}{\textbf{[UNI]}} ユニゾンモード。複数(最大16)のオシレーターを作成,
      \textcolor{black}{\textbf{[DET]}} Detune. 最大範囲 +/- 半音のユニゾン モードでオシレーターをデチューン,
      \textcolor{black}{\textbf{[AM]}} Amplitude Modulation Level (VCAとして使用可能),
      \textcolor{black}{\textbf{[DTWT]}} Detune WT,
      \textcolor{black}{\textbf{[FRQC]}} Frequency Coarse,
      \textcolor{black}{\textbf{[FRQF]}} Frequency Fine,
      \textcolor{black}{\textbf{[GLDT]}} Glide Time,
    }{
      Smooting \- ウェーブテーブルの再生方法を変更,
      NA,
      NA,
      NA,
      \textcolor{black}{\textbf{[NOTE-OCT]}} ノートをオクターブ単位でチューニング (+/- 8 オクターブ)),
      \textcolor{black}{\textbf{[NOTE-NOTE]}} 音を半音単位でチューニング (+12 半音 1 オクターブ),
      \textcolor{black}{\textbf{[NOTE-FINE]}} セント単位でノートを微調整 (+/- 半音),
      音声出力先の選択,
    }
}

\def\modnameshortwto{WTO}
\def\modnamelongwto{Wavetable Oscillator}

\newcommand{\modwto}{
  \newpage
  \modmakesection{wto}

  \paragraph*{}
Wavetable Oscillator (WTO) は、The Centreサウンドの主要な構成要素です。 WTO は、ユニゾン モードで動作する 1 ~ 16 のオシレーターで構成され、ピッチ及び、波形のデチューンが可能なウェーブテーブルを再現します。
  
  \showrefcardsection{wto}

  \newpage
  \section*{\modname}  
  \tableofoptionsnew{\wtoset}{\optnameset}
  \tableofoptionsnew{\wtoinp}{\optnameinp}
  \newpage
  \section*{\modname}
  \tableofoptionsnew{\wtoinpi}{\optnameinp}
  \tableofoptionsnew{\wtoout}{\optnameout}
}




% \def\lfsbutmain{
%    {Edit}/{Invokes Shape Editor},
%    {Set}/{Open Settings},
%    {Inputs}/{Open Inputs},
%    {Back}/{Return to Patch Menu}
% }



\def\vcoset{
  {{Audio Output}/{VCOのオーディオ出力}/{VCO の音声出力先を選択します。
  \newline 参照: \underline{\nameref{section:audiooutputs}}}},
  {{Waveform}/{波形の選択}/{
    \makesupopt{{
      {{Triangle}/{標準Triangle波形。波形は Input: Skew の影響を受け、Ramp から Triangle を経て Saw に変化します。}},
      {{Square}/{標準Square波形。  波形は、PWM (Pulse Width Modulation) を変更するInput (Skew) の影響を受けます。}},
      {{Sine}/{標準Sine波形。波形は、2 つの非対称sine phasesに変更する Input(Skew) の影響を受けます。
      \newline$\blacksquare$ Skew パラメータを変更して負と正のハーフ サインの特定の比率を調整すると、Diode Ladderフィルターと相性の良い非常に興味深いハーモニクスが追加されます。
      }}%
    }}
  }},
}
% \def\inpclock{
%    {{Clock}/{Clock for timing oscillator's phase duration in BPM mode (otherwise 120BPM is used)}/{}}
% }

% \def\vcoinp{
%   {{Unison}/{Unison mode (1-16 voices)}/{Enables unison mode and controls number of oscillators running in parallel.}},
%   {{Detune}/{Detune oscillators}/{Detunes oscillators in Unison mode by spreading them equally in Detune range. With detune set at maximum the spread range is 2 semi-tones. With odd number of oscillators the centre oscillator will always be following pitch from Note Input. For even number of oscillators there is no oscillator followin Note and each oscillator is detuned +/- from the center note. \newline $\bigstar$ \textit{With Unison running two oscillators, detune set to maximum (1.0) and pitch set at note D there is one oscillator playing note C\# and one playing note D\# (no oscillator is actually playing note C)}}},
%   {{Wavetable}/{Wavetable position}/{Adjusts position of waveform in wavetable. When controlled by CV allows changing texture of sound by scanning through set of waveforms contained in wavetable}},
%   {{Reset Osc}/{Reset Oscillator CV input}/{When connected with CV input and set to high level it resets phase of oscillator to 0. Useful to run oscillators in Sync}}%
% }

\def\vcoinp{
  {{NOTE}/{オシレーターのピッチコントロール}/{Note は、オシレーターのピッチまたは周波数を制御します。
  \newline 参照 \underline{\nameref{section:pitchcontrol}}
  \makesupopt{{
    {{VOCT}/{ノートコントロール用の1V/Oct入力/}},
    {{OCT}/{オクターブ単位のチューニングノート +/- 8 オクターブ}},
    {{NOTE}/{半音単位のチューニング音 +12 半音 (1 オクターブ)}},
    {{FINE}/{セント +/- 半音単位のチューニング ノート}}%
  }}%
  }},
   {{AM}/{Amplitude Modulation}/{振幅変調はサウンドレベルを変化させます。 Envelope の CV 信号と接続すると VCA として機能します。 LFO と接続すると、リング モジュレーターが作成されます。}},
  {{FM}/{Frequency Modulation}/{Frequency modulationは、変調による周波数の変化です。
  \newline$\blacksquare$  FM の CV 入力にモジュレーターを追加し、実際の VCO に対して特定の周波数比で動作させると、興味深いサウンド テクスチャが作成されます。}},
  {{Skew}/{波形の非対称性を変形}/{Skew は、Triangle を Saw または Ram に変えたり、Square Wave の PWM を変更したりすることで、波形の非対称性に影響を与えます。上記の設定: Waveformを参照してください。}},
  {{Hard Sync}/{オシレータの位相をリセット}/{オシレータの位相をリセットし、波形の先頭からオシレーションを開始します。
  \newline$\bigstar$ VCO1 Output: Reset から VCO2 Input: Hard Sync にわずかにデチューンした 2 つの VCO を接続することで、シンプルなオシレーションの非常に豊かなテクスチャーを作成できます。}},
  {{PM}/{Phase Modulation}/{Phase modulation により、モジュラーはオシレーターの位相を変更できます。 FM (Frequency Modulation) とは異なり、Phase Modulation は方向を変えることでオシレーターをリバースモードで動作させることができます。}}%
} 

\def\vcoout{
    {{OSC}/{VCO出力}/{オシレーター信号の出力}},
    {{RST}/{オシレーターリセット}/{発振器の位相がリセットされるとhighに設定されます。}}%
}

\def\refcardvco{
  \makereferencecard{
    \textcolor{black}{\textbf{[回す]}} 波形の切り替え(Triangle Square Sine),
  }{
    \textcolor{black}{\textbf{[ON/OFF]}} モジュールのオン/オフ (Bypass),
    \textcolor{black}{\textbf{[SET]}} モジュールの設定,
    \textcolor{black}{\textbf{[Inputs]}} モジュールの入力を構成する,
    \textcolor{black}{\textbf{[Back]}} パッチメニューに戻る,
  }{
    \textcolor{black}{\textbf{[SKEW]}} 波形の PWM または PWM styleの効果レベルを調整する,
    NA,
    NA,
    \textcolor{black}{\textbf{[AM]}} Amplitude Modulation Level (VCAとして使用可能),
    \textcolor{black}{\textbf{[FM]}} Frequency Modulation Level,
    \textcolor{black}{\textbf{[PM]}} Phase Modulation Level,
    NA,
    NA,
  }{
    NA,
    NA,
    NA,
    NA,
    \textcolor{black}{\textbf{[NOTE-OCT]}} ノートをオクターブ単位でチューニング (+/- 8 オクターブ)),
      \textcolor{black}{\textbf{[NOTE-NOTE]}} 音を半音単位でチューニング (+12 半音 1 オクターブ),
      \textcolor{black}{\textbf{[NOTE-FINE]}} セント単位でノートを微調整 (+/- 半音),
      音声出力先の選択,
    }
}

\def\modnameshortvco{VCO}
\def\modnamelongvco{Voltage Controlled Oscillator}

\newcommand{\modvco}{
  \newpage
  \modmakesection{vco}

  \paragraph*{}
Voltage Controlled Oscillatorは、波形のオシレーションを生成する音源です。
  
  \showrefcardsection{vco}

  \subsection{波形}
  \paragraph*{}VCO は基本的な形状 (Sine, Triangle, Square) を取り、特定の周波数でこれらの波形の単相をオシレーションします。各 \textbf{波形} は、Skew CV入力でそのピボットパラメータで変更できます。 Skew パラメータをTriangleに適用すると、\textit{Ramp} または \textit{Saw} のいずれかに変わります。 Square の場合、Skew パラメータは、Square パルスの長さを変更する PWM (Pulse Width Modulation) を調整します。 Sine の場合、Skewパラメータは、Sine波の正と負の部分の位相の比率を変更します。


  \newpage
  \section*{\modname}  
  \tableofoptionsnew{\vcoset}{\optnameset}%
  \tableofoptionsnew{\vcoinp}{\optnameinp}%
  \tableofoptionsnew{\vcoout}{\optnameout}%
}




\def\lfsbutmain{
   {Edit}/{Shape Editorを呼び出す },
   {Set}/{Settingsを開く},
   {Inputs}/{Inputsを開く},
   {Back}/{Patch Menuに戻る}
}

\def\lfsset{
    {{Running Mode}/{low frequency shapeのRunning Modeの種類を選択}/{
       \makesupopt{{
         {{Continuous}/{オシレーターは動き続け、止まることはありません。ゲート ハイ (トリガー) 信号は、オシレーターの位相を 0 にリセットします。}},
         {{Trigger}/{オシレータは、ゲート信号がハイになると 1 回実行され、完全なシェイプが 1 回実行され、シェイプのフェーズが終了すると生成が停止します。別のトリガー信号は位相を 0 にリセットします。}},
         {{Gate length}/{オシレータは、ゲート信号がハイになると 1 回実行され、完全なシェイプが 1 回実行され、シェイプのフェーズが終了すると生成が停止します。別のトリガー信号は位相を 0 にリセットします。}}%
      }}
    }},
    {{Timing Mode}/{LFSのタイミングモードを選択}/{LFS のタイミングは、時間ベースでHz で計算するか、BPM に関連付けて音符または小節の長さで測定するように選択できます。
    \makesupopt{{
       {{BPM}/{ノートの長さに基づくタイミング}},
       {{Hz}/{フリクエンシーに基づくタイミング}}%
    }}
    }},
    {{Polarity}/{出力のPolarity}/{
       \makesupopt{{
         {{Bipolar}/{出力は -5V から +5V の範囲で整形します。リングモジュレーションやFMに最適。}},
         {{Unipolar}/{出力は 0V から +5V の範囲で整形します。 VCAに最適。}}%
      }}
    }}%
}

\def\lfsinp{
   {{Trigger}/{Trigger CV入力}/{オシレーターを開始またはリセットするトリガーまたはゲート信号}},
   {{Clock}/{BPM モードでのタイミングオシレータの位相持続時間のクロック (未設定では120BPM が使用されます)}/{}},
   {{Frequency Coarse}/{オシレータの周波数を大まかに調整する}/{オシレータの周波数を 0Hz から 100Hz の範囲 (周波数モード) または 1/256 ノートから 8bar の間 (BPM/ノートデュレーションモード) で調整します。}},
   {{Frequency Fine}/{オシレーターの周波数を微調整する}/{オシレーターの周波数を細かく調整します。}}%
}

\def\lfsout{
    {{SHP}/{シェイプオシレーター出力 }/{他の信号を変調するために使用できるシェイプオシレーターの出力}}%
}

\def\refcardlfs{
   \makereferencecard{\textcolor{black}{\textbf{[回す]}} 現在のディレクトリ内の図形間を移動する%
   \newline\textcolor{black}{\textbf{[クリック]}} ファイルブラウザに入り、シェイプまたはブラウジングディレクトリを変更
}{
   \textcolor{black}{\textbf{[Edit]}} Shape Editorに入る,
   \textcolor{black}{\textbf{[SET]}} モジュールのセッティング,
   \textcolor{black}{\textbf{[Inputs]}} モジュールの入力を構成する,
   \textcolor{black}{\textbf{[Back]}} パッチメニューに戻る,
}{
   \textcolor{black}{\textbf{[FRQC]}} LFO の周波数コース (HZ: 0 ~ 100Hz または BPM: 1/256 ~ 8bar),
   \textcolor{black}{\textbf{[FRQF]}} 周波数の微調整,
   NA,
   NA,
   NA,
   NA,
   NA,
   NA,
   }{
      \textcolor{black}{\textbf{[RUN]}} Running mode を選択 (Trigger Continuous Gate Length),
      Timing Modeを選択: BPM or Hz,
      NA,
      NA,
      \textcolor{black}{\textbf{[CLK]}} CLOCKソースCVを選択,
      \textcolor{black}{\textbf{[TRIG]}} Trigger CVを選択,
      NA,
      シェイプの出力先を選択,
   }
}

\def\refcardlfsedit{
  \makereferencecard{\textcolor{black}{\textbf{[回す]}} シェイプのポイントを移動%
      \newline\textcolor{black}{\textbf{[クリック]}} シェイプの 2 点を水平方向に整列 - 前の点まで
   }{
      \textcolor{black}{\textbf{[Add]}}ポイントを追加してシェイプのセグメントを作成,
      \textcolor{black}{\textbf{[Del]}} 現在のポイントを,
      \textcolor{black}{\textbf{[Free/Quant]}} フリー編集とクオンタイズ編集の切り替え,
      \textcolor{black}{\textbf{[Back]}} シェイプメニューに戻る,
   }{
      NA,
      NA,
      \textcolor{black}{\textbf{[Y]}} ポイントの垂直位置を調整する,
      \textcolor{black}{\textbf{[X]}} ポイントの水平位置を調整する,
      \textcolor{black}{\textbf{[BEND-]}} 選択した点より前の形状の曲率を変更する,
      \textcolor{black}{\textbf{[BEND+]}} 選択した点の後の形状の曲率を変更する,
      NA,
      NA,
      NA,
      NA,
      NA,
      NA,
   }{
      NA,
      NA,
      NA,
      NA,
      NA,
      NA,
      NA,
      NA,
   }
}

\def\modnameshortlfs{LFS}
\def\modnamelonglfs{Low Frequency Shaper}

\newcommand{\modlfs}{
   \newpage
   \modmakesection{lfs}
   \def\modname{\modnameshortlfs - \modnamelonglfs}% chktex 8
   % \section{\modname}\label{section:modlfs}

   \subsection*{LFS \- Low Frequency Shaper (不規則なシェイプを生成するLow Frequency Oscillator)}
   \paragraph*{} 
このモジュールは、不規則なシェイプの low frequency oscillationsを作成します。シェイプは、ユーザーが編集するか、.shp (Serum) および .vitallfo (Vital) 形式からインポートできます。

   \showrefcardsection{lfs}

   \subsection{シェイプのロード}
シェイプをロードするには、\textbf{[SELECT]} (エンコーダーを下に) を押し、ファイル ブラウザでシェイプのあるディレクトリに移動し、\textbf{[SELECT]} をもう一度押して目的のシェイプをロードします。 エンコーダーを回転させると、現在のディレクトリ内のシェイプが変化します。


   \newpage
   \section*{\modname}
   \tableofoptionsnew{\lfsset}{\optnameset}
   \tableofoptionsnew{\lfsinp}{\optnameinp}
   \tableofoptionsnew{\lfsout}{\optnameout}

   \newpage
   \section{LFS - Shape Editor}\label{section:shapeedit}
   \showrefcardsection{lfsedit}
   \subsection{シェイプのエディット}
   シェイプを編集するには、LFS 画面で \textbf{Edit} を押します。ボタン \textbf{Add} と \textbf{Del} を使用して、シェイプに点を追加できます (点は現在の点の間、つまり中央に追加されます)。 \textbf{LVL3} と \textbf{LVL5} を使用して、ポイントの垂直位置と水平位置を適切に調整します (注意: 最初と最後のポイントの水平位置は調整できません)。 \textbf{LVL5} は選択した点より前のシェイプセグメントの曲率を調整し、\textbf{LVL6} は選択した点の後のセグメントの曲率を調整します。
   \subsection{Position Quantisation}
   ボタン \textbf{Free} を押すと、モードをフリー編集に変更できます。次に、ボタン \textbf{Quant} (同じボタン) を押すと、モードを Quantized編集に切り替えます。このモードでは、垂直位置が 12 の分割に量子化 (整列) されます (半音をシミュレートします)、水平位置は 32 分割に整列され、ノートの長さをシミュレートする整列を提供します。 
   \newline $\blacksquare$ Aligmentまたはquantisationで、音符または半音は実際には整列しませんが、シェイプが BPM の速度で再生されたときに配置が改善されます。 
   \newline $\bigstar$ 整列されたシェイプは、アルペジエーター効果を作成するために VCO または WTO の周波数に適用可能です。

}


\def\lfoset{
  {{Audio Output}/{Audio Output of LFO}/{Selects audio output destination for LFO.
  \newline See: \underline{\nameref{section:audiooutputs}}}},
  {{Waveform}/{Selection of Waveform}/{
    \makesupopt{{
      {{Triangle}/{Standard triangle waveform.  The waveform is affected by Input: Skew and changes from Ramp through Triangle to Saw}},
      {{Square}/{Standard square waveform.  The waveform is affected by Input: Skew that modifies PWM (Pulse Width Modulation) of the waveform}},
      {{Sine}/{Standard sine waveform.  The waveform is affected by Input: Skew that modifies it to two assymetric sine phases.
      \newline$\blacksquare$ Adjusting certain ratios of negative and positive half sines by modifying Skew parameter will add very interestic harmonics that go well with Diode Ladder filter.
      }}%
    }}
  }},
}
% \def\inpclock{
%    {{Clock}/{Clock for timing oscillator's phase duration in BPM mode (otherwise 120BPM is used)}/{}}
% }

% \def\lfoinp{
%   {{Unison}/{Unison mode (1-16 voices)}/{Enables unison mode and controls number of oscillators running in parallel.}},
%   {{Detune}/{Detune oscillators}/{Detunes oscillators in Unison mode by spreading them equally in Detune range. With detune set at maximum the spread range is 2 semi-tones. With odd number of oscillators the centre oscillator will always be following pitch from Note Input. For even number of oscillators there is no oscillator followin Note and each oscillator is detuned +/- from the center note. \newline $\bigstar$ \textit{With Unison running two oscillators, detune set to maximum (1.0) and pitch set at note D there is one oscillator playing note C\# and one playing note D\# (no oscillator is actually playing note C)}}},
%   {{Wavetable}/{Wavetable position}/{Adjusts position of waveform in wavetable. When controlled by CV allows changing texture of sound by scanning through set of waveforms contained in wavetable}},
%   {{Reset Osc}/{Reset Oscillator CV input}/{When connected with CV input and set to high level it resets phase of oscillator to 0. Useful to run oscillators in Sync}}%
% }

\def\lfoinp{
  {{NOTE}/{Pitch control of oscillator}/{Note controls pitch or frequency of oscillator.
  \newline See: \underline{\nameref{section:pitchcontrol}}
  \makesupopt{{
    {{VOCT}/{1V/Oct input for note control/}},
    {{OCT}/{Tuning note by octaves +/- 8 octaves}},
    {{NOTE}/{Tuning note by semitones +12 semitones (one octave)}},
    {{FINE}/{Tuning note by cents +/- one semitone}}%
  }}%
  }},
  {{AM}/{Amplitude Modulation}/{Amplitude modulation changes sound level. Connected with CV signal of Envelope acts as VCA. Connected with LFO creates ring modulator}},
  {{FM}/{Frequency Modulation}/{Frequency modulation is a change of frequency by modulation.
  \newline$\blacksquare$  Adding a modulator to CV input of FM and having it running at certain ratio of frequency to the actual LFO will create interesting sound texture}},
  {{Skew}/{Modify assymetry of waveform}/{Skew affects assymetry of waveform by turning Triangle into Saw or Ram or modyfying PWM of Square wave. See above in Settings: Waveform}},
  {{Hard Sync}/{Reset phase of oscillator}/{Resets phase of oscillator and starts oscillating from the beginning of waveform. 
  \newline$\bigstar$ By connecting two LFOs detuned slightly through LFO1 Ooutput: Reset to LFO2 Input: Hard Sync we can create ver rich texture of simple oscillation}},
  {{PM}/{Phase Modulation}/{Phase modulation allows modulator to modify phase of oscillator. Unlike FM (Frequency Modulation) Phase Modulation can operate oscillator in reverse mode by turning its direction.}}%
} 

\def\lfoout{
    {{OSC}/{Output of LFO}/{Output of oscillator signal}},
    {{RST}/{Oscillator reset}/{Set to high when oscillator phase resets}}%
}

\def\refcardlfo{
  \makereferencecard{
    \textcolor{black}{\textbf{[ROTATE]}} Switches between waveforms (Triangle Square Sine),
  }{
    \textcolor{black}{\textbf{[ON/OFF]}} Turning module On or Off (Bypass),
    \textcolor{black}{\textbf{[SET]}} Settings of the module,
    \textcolor{black}{\textbf{[Inputs]}} Configure Inputs of the module,
    \textcolor{black}{\textbf{[Back]}} Return back to Patch Menu,
  }{
    \textcolor{black}{\textbf{[SKEW]}} Adjust level of PWM or PWM-Style effect on waveform,
    \textcolor{black}{\textbf{[FRQC]}} Frequency Coarse for LFO 0 - 28Hz,
    \textcolor{black}{\textbf{[FRQF]}} Frequency fine tune (+ 10\% of coarse frequency),
    \textcolor{black}{\textbf{[AM]}} Amplitude Modulation Level (can be use as VCA),
    \textcolor{black}{\textbf{[FM]}} Frequency Modulation Level,
    \textcolor{black}{\textbf{[PM]}} Phase Modulation Level,
    NA,
    NA,
  }{
    NA,
    NA,
    NA,
    NA,
    NA,
    NA,
    NA,
    Select Audio Output Destiantion,
    }
}

\def\modnameshortlfo{LFO}
\def\modnamelonglfo{Low Frequency Oscillator}

\newcommand{\modlfo}{
  \newpage
  \modmakesection{lfo}

  \paragraph*{}
  \modnamelonglfo is a source of modulation that oscillates basic waveforms at very low frequencies (usually below audible level).
  
  \showrefcardsection{lfo}

  \subsection{Waveforms}
  \paragraph*{}LFO takes basic shapes: Sine, Triangle and Square and oscillates the single phase of those waveforms at given frequency. Each \textbf{Waveform} can be modified by bending it's pivot params with the Input:Skew CV input. Applying Skew parameter to triangle turns it into either \textit{Ramp} or \textit{Saw}. For Square, Skew parameter adjusts PWM (Pulse Width Modulation) which modifies length of square pulse. For Sine, Skew parameter changes the ratio of phase for positive and negative parts of Sine wave.


  \newpage
  \section*{\modname}  
  \tableofoptionsnew{\lfoset}{{Settings}}%
  \tableofoptionsnew{\lfoinp}{{Inputs}}%
  \tableofoptionsnew{\lfoout}{{Outputs}}%
}




\def\envset{
  {{Audio Output}/{Output of ENV}/{Audio Output of processed signal by ENV. Signal has amplitude modulated by sum of two modulators: Level and Sidechain. \newline See: \underline{\nameref{section:audiooutputs}}}},
  {{Audio Input}/{Audio signal to be modulated}/{Audio signal (can be any signal) that will have amlitude (level) modulated with modulator Input: Level}},
  {{Mode}/{Envelope Mode}/{Mode switch between AHDSR (with Sustain) and AHDR (no sustain) envelopes
    \makesupopt{{
      {{Gate}/{The envelope lasts as long as the gate signal is open (high) and sustains volume at the Sustain level}},
      {{Trigger}/{The enevelope only lasts as long as sum of stages AHDR and never keeps note sustained}}%
    }}
  }}%
}

\def\envinp{
  {{Attack}/{Attack Stage}/{Attack is the raising first stage of envelope. Duration between 0s and 32s}},
  {{Hold}/{Hold Stage}/{Hold is the full volume level sustained second stage of envelope. Duration between 0s and 32s}},
  {{Decay}/{Decay Stage}/{Decay is the falling third stage of envelope going from full level of volume to level set . Duration between 0s and 32s}},
  {{Sustain}/{Sustain Stage}/{Sustain is the level of the sound after AHD stages at which sound is kept as long as Gate is open (high). Level of sound in decibels}},
  {{Release}/{Release Stage}/{Release is the final stage where the sound decays to level 0. Duration between 0s and 32s}},
  {{Gate}/{Gate CV Source}/{Gate or Trigger signal to start and sustain envelope generation}},
  {{Attack Curve}/{Exponent of Attack stage}/{Attack Curve is the exponent of the attack stage to modify stage to exponential rather than linear}},
  {{Decay Curve}/{Exponent of Decay stage}/{Decay Curve is the exponent of the Decay stage to modify stage to exponential rather than linear}},
  {{Release Curve}/{Exponent of Release stage}/{Release Curve is the exponent of the Release stage to modify stage to exponential rather than linear}}%
}

\def\envout{
  {{Envelope}/{Output of Envelope}/{Output of Envelope signal to be routed to other modules as a Control Voltage}}%
}

\def\refcardenv{
  \makereferencecard{
    \textcolor{black}{\textbf{[ROTATE]}} Switches between waveforms (Triangle Square Sine),
  }{
    \textcolor{black}{\textbf{[ON/OFF]}} Turning module On or Off (Bypass),
    \textcolor{black}{\textbf{[SET]}} Settings of the module,
    \textcolor{black}{\textbf{[Inputs]}} Configure Inputs of the module,
    \textcolor{black}{\textbf{[Back]}} Return back to Patch Menu,
  }{
   Adjust ATTACK in range of 0.0 to 31.5s,
   Adjust HOLD in range 0 to 31.5s. \newline$\blacksquare$ When HOLD is 0s envelope is ADSR,
   Adjust DECAY time in range of 0.0s to 31.5s,
   Adjust SUSTAIN level,
   Adjust RELEASE time in range of 0.0s to 31.5s,
   NA,
   NA,
   NA,
  }{
   Adjust curve shape of ATTACK,
   NA,
   Adjust curve shape of DECAY,
   Adjust curve shape of RELEASE,
   \textcolor{black}{\textbf{[GAT]}} Select GATE Input CV,
   \textcolor{black}{\textbf{[MODE]}} Change between Trigger (AHDR) and Gate (AHDSR) modes,
   NA,
   Select AUDIO Output of Envelope (allows to send envelope out to control other devices),
  }
}

\def\modnameshortenv{ENV}
\def\modnamelongenv{Envelope Generator}

\newcommand{\modenv}{
  \newpage
  \modmakesection{env}

  \paragraph*{}Envelope Generator produces a modulation signal that in its substance is of raise and decay at the given rates.
  
  \showrefcardsection{env}

  \subsection{Operation}
  Envelope Generator produces AHDSR (Attack, Hold, Decay, Sustain and Release) or AHDR envelopes that are Control Voltage signals to modulate parameters of modules in the synthesiser (usually, but not limited to, Amplitude of ENV and Cutoff Frequency of VCF).
  \subsection{Envelope Stages}
  \makesupopt{{
    {{A: Attack}/{Attack stage is first and the raising stage of signal from 0 to 100\% volume of signal during given period of time.}},
    {{H: Hold}/{Hold stage keeps signal at the highest volume level for a given period of time}},
    {{D: Decay}/{Decay stage determines the length of time the sound volume will be dropping from \textbf{Hold} level to \textbf{Sustain} level. The Decay stage length is given as period of time.}},
    {{S: Sustain}/{Sustain stage determines the volume level at which sound will keep playing before the gate signal that is relative to played note length keeps high signal. Sustain stage unlike other parameters of envelope is not given in duration of time but rather as a level of volume.}},
    {{R: Release}/{Release is the final stage of sound when it either slowly decays or rapienv ends. Stage is defined as period of time. }}%
  }}%
  \section*{\modname}

$\bigstar$ If the \textbf{Gate} signal length (played note length) is shorter than combined length of \textbf{Atack} + \textbf{Hold} + \textbf{Decay} then some of those stages will be skipped all the way to Sustain stage as the Sustain stage starts when \textbf{Gate} signal becomes low. Therefore if \textbf{Trigger} signal is applied instead of \textbf{Gate} signal then all stages (A+H+D) will be skipped and the envelope will be limited to only Release stage.

  \subsection{Envelope Types}
  The most popular envelops out there are ADSR (Attack, Decay, Sustain and Release) - this is envelope designed for Gate signal of variable length that is relative to length of played note. The envelope upon raise of gate signal will execute Attack and Decay stages and will keep amplitude at Sustain level until gate signal turns low (note length ends) and the release stage will help amplitude to continue decay.
  On the contrary, AD envelopes are designed for Trigger signal and they execute both Attack and Decay stage regardless of note length. The length of note is dictated by length of Attack and Decay stages.
  \paragraph*{} \textbf{Hold} stage in \textbf{AHDSR} envelope is just extension of typical ADSR envelope where the maximum raised stage (after \textbf{Attack}) can be sustained for period of time. With Hold stage period length set to 0s AHDSR envelope becomes standard ADSR envelope.
  
  \subsection{Trigger Mode}
  Envelopes can be generated in two modes which can be activated by switching Setting:Mode btween \textbf{Gate} and \textbf{Trigger}.
  \makesupopt{{
    {{Gate}/{In this mode the envelope is based on the length of provided gate signal being high. When the gate is in high mode the envelope processes through AHD stages and stays in Sustain stage until gate closes (signal turns low) and the Release stage is being processed}},
    {{Trigger}/{In this mode the envelope is always fully processed however without Sustain element. The envelope is beign triggered by high Trigger (Gate) signal and regardless of length of that signal will execute AHDR stages and the Sustain part will be ommited.
    \newline$\bigstar$ This mode is ideal to emulate older AD envelopes by setting stages A and D to desired values and every other stage to 0}}%
  }}%
  % \newpage
  \section*{\modname}
  \tableofoptionsnew{\envset}{{Settings}}
  \tableofoptionsnew{\envinp}{{Inputs}}
  \tableofoptionsnew{\envout}{{Outputs}}
}


\def\vcaset{
  {{Audio Output}/{VCA出力}/{VCA で処理された信号のオーディオ出力。信号は、LevelとSidechainの 2 つのモジュレーターの合計によって変調された振幅を持っています。  \newline 参照: \underline{\nameref{section:audiooutputs}}}},
  {{Audio Input}/{モジュレーションされる音声信号}/{モジュレーターで変調された振幅 (レベル) を持つオーディオ信号 (任意の信号) 入力: Level}}%
}

\def\vcainp{
  {{Level}/{サウンドレベルのモジュレーション}/{サウンドレベルの変調。Unipolar(正のみ)モジュレーター(通常はエンベロープ)による振幅変調}},
  {{Sidechain}/{レベルの逆変調}/{サイドチェーンは、プライマリ信号変調のレベルを減衰させるリバース信号です。
  \newline$\bigstar$
  通常は、干渉する周波数を除去するためにキックドラムが入ってきたときにベースラインの振幅を下げるために使用されます。}}%
}

\def\vcaout{
  {{NONE}/{}/{}}%
}

\def\refcardvca{
  \makereferencecard{
    NA,
  }{
    \textcolor{black}{\textbf{[ON/OFF]}} モジュールのオン/オフ (バイパス),
    \textcolor{black}{\textbf{[SET]}} モジュールの設定,
    \textcolor{black}{\textbf{[Inputs]}} モジュールの入力を構成する,
    \textcolor{black}{\textbf{[Back]}} パッチメニューに戻る,
  }{
    Amplificationレベルの調整,
    NA,
    NA,
    NA,
    NA,
    NA,
    NA,
    NA,
  }{
    Level の CV 入力の減衰を調整,
    NA,
    NA,
    NA,
    Level Modulationのソースを選択,
    NA,
    VCAの入力を選択,
   }
}

\def\modnameshortvca{VCA}
\def\modnamelongvca{Voltage Controlled Amplifier}

\newcommand{\modvca}{
  \newpage
  \modmakesection{vca}

  \paragraph*{}Voltage Controlled Amplifierはモジュレーターレベルで信号レベルを変調します。. 
  
  \showrefcardsection{vca}

  \subsection{操作}
  VCA はこれらの信号の単純なAmplitude Modulationを実行し、オーディオ信号の音量 (レベル) を下げます。 Audio Input に入ってくる信号は、Input:Modulation から取得したモジュレーターで変調 (減衰) されます。 実際、VCA はAmplitude Modulationの原理を使用して動作しますが、オーディオ信号を変調するには正のunipolar CV 信号のみが必要です。
  \newline$\bigstar$ 通常、モジュレーター信号は AD、ADSR、または AHDSR エンベロープのエンベロープ ジェネレーター (ENV) から来ており、オシレーターなどのオーディオ ソースからのオーディオ信号に適用されると、Pluck音や上昇音として現れます。
  \subsection{Sidechain}
  サイドチェーンは、別のオーディオ信号をより際立たせる必要がある場合に、オーディオ信号をダッキング (低減) する技術です。サイドチェーン モジュレーターは主信号のエンベロープであり、VCA に入力されるオーディオ信号のエンベロープから差し引かれ、セカンダリ シグナルのレベルを下げます。
  \newline$\bigstar$ サイドチェインは、キックドラムが演奏されているときにベースラインのレベルを下げるために使用されます。多くの場合、キックドラム/バスドラムとベースラインの両方が同じ低周波数を共有しており、それらが混ざるとこもった音になります。キックドラムが鳴るポイントでベースラインの振幅を下げることにより、キックドラムのエンベロープが VCA のサイドチェーン入力に送られ、VCA はサイドチェーンの量だけベースラインのレベルを下げます。このようにして、キックドラムの低周波がベースライン上で聞こえます。
  
  \newpage
  \section*{\modname}
  \tableofoptionsnew{\vcaset}{\optnameset}
  \tableofoptionsnew{\vcainp}{\optnameinp}
  \tableofoptionsnew{\vcaout}{\optnameout}
}


\def\brmset{
  {{Audio Output}/{Output of BRM}/{Audio Output of processed signal by BRM. Signal has amplitude modulated by sum of two modulators: Level and Sidechain. \newline See: \underline{\nameref{section:audiooutputs}}}},
  {{Audio Input 1}/{Input of the first Audio Signal}/{Audio signal (can be any signal) That will be combined with the second signal}},
  {{Audio Input 2}/{Input of the second Audio Signal}/{Audio signal (can be any signal) That will be combined with the first signal}}%
}

\def\brminp{
  {{Balance}/{Balance between signals}/{The ratio between two signals when combining them. The volume level of the signals will be adjusted respectively before combining)}}%
}

\def\brmout{
  {{NONE}/{}/{}}%
}

\def\refcardbrm{
  \makereferencecard{
    NA,
  }{
    \textcolor{black}{\textbf{[ON/OFF]}} Turning module On or Off (Bypass),
    \textcolor{black}{\textbf{[SET]}} Settings of the module,
    \textcolor{black}{\textbf{[Inputs]}} Configure Inputs of the module,
    \textcolor{black}{\textbf{[Back]}} Return back to Patch Menu,
  }{
    Adjust Level of Cross modulation,
    NA,
    NA,
    NA,
    NA,
    NA,
    NA,
    NA,
  }{
    Adjust attenuation of CV input for Modulation,
    NA,
    NA,
    NA,
    Select source of Level Modulation,
    Select the first Audio Input,
    Select the first Audio Input,
    Select the Audio Output of combined signal by Ring Modulator,
   }
}


\def\modnameshortbrm{BRM}
\def\modnamelongbrm{Balanced Ring Modulator}

\newcommand{\modbrm}{
  \newpage
  \modmakesection{brm}

  \paragraph*{}Balanced Ring Modulator takes two Audio signals and mcombines them to gether producing output signal.
  
  \showrefcardsection{brm}

  \subsection{Operation}
  BRM performs cross modulation of two signals by combining Aduio Input 1 with Audio Input 2. The Input:Balance setting is the ratio between those two signals.
  
  \newpage
  \section*{\modname}
  \tableofoptionsnew{\brmset}{{Settings}}
  \tableofoptionsnew{\brminp}{{Inputs}}
  \tableofoptionsnew{\brmout}{{Outputs}}
}


\def\smpset{
  {{Audio Output}/{Audio Output of SMP - Stereo}/{Selects audio output destination for SMP. All oscillators including Polyphony mode are output through single channel (Mono or Stereo). \newline See: \underline{\nameref{section:audiooutputs}}}},
  {{Running Mode}/{Select type of running mode of low frequency shape}/{
    \makesupopt{{
      {{Continuous}/{Sample playback keeps looping. Gate High (Trigger) signal will reset sample playback to start position.}},
      % \item [\tikz{\node[text=white, fill=black] {Continuous};}] Oscillator keeps running and never stops. Gate High (Trigger) signal will reset oscillator phase to 0.\
      {{Trigger}/{Sample plays once upon the gate signal high and runs complete shape once and stops at the end position. Another trigger during playback signal will reset playback to start position}},
      {{Gate length}/{Sample plays as long as gate signal is high. Stops immediatelly when signal goes to low.}}%
   }}
 }},
}

\def\smpinp{
  {{NOTE}/{Pitch control of oscillator}/{Note controls pitch or frequency of oscillator.
  \newline See: \underline{\nameref{section:pitchcontrol}}
  \makesupopt{{
    {{VOCT}/{1V/Oct input for note control/}},
    {{OCT}/{Tuning note by octaves +/- 8 octaves}},
    {{NOTE}/{Tuning note by semitones +12 semitones (one octave)}},
    {{FINE}/{Tuning note by cents +/- one semitone}}%
  }}%
  }},
  {{Gate}/{Gate CV input}/{Gate or Trigger signal that initiates or resets playback of sample}},
  {{AM}/{Amplitude Modulation}/{Amplitude modulation changes sound level. Connected with CV signal of   Envelope acts as VCA. Connected with LFO creates ring modulator}},
  {{Sample Start}/{Start position of sample}/{The starting position of sample that sample player starts playing upon receiving gate or trigger signals}},
  {{Loop Start}/{Start of loop}/{The restarting position of loop upon completing single phase of playback}},
  {{Loop End}/{End of loop}/{The final position of sample that will trigger restart of playback from Loop Start}},
}
\def\smpout{
    {{OUT}/{Output of SMP}/{Output of wavetable oscillator}},
    {{RST}/{Oscillator reset}/{Set to high when oscillator phase resets}}%
}



\def\refcardsmp{
  \makereferencecard{
    \textcolor{black}{\textbf{[ROTATE]}} Navigate between samples in current directory
    \newline\textcolor{black}{\textbf{[CLICK]}} Enter File Browser to change to chosen sample and to change browsing directory
    \newline\textcolor{black}{\textbf{[LONG PRESS]}} Enter Wavetable Editor
    }{
      \textcolor{black}{\textbf{[2D/3D]}} Switch between display of wavetable,
      \textcolor{black}{\textbf{[SET]}} Settings of the module,
      \textcolor{black}{\textbf{[Inputs]}} Configure Inputs of the module,
      \textcolor{black}{\textbf{[Back]}} Return back to Patch Menu,
      }{
      \textcolor{black}{\textbf{[AM]}} Amplitude Modulation Level (can be used as VCA),
      NA,
      NA,
      Control the window in ZOOM mode,
      \textcolor{black}{\textbf{[START]}} Adjust start pointer of the sample,
      \textcolor{black}{\textbf{[LOOP-START]}} Adjust loop start pointer of the sample,
      \textcolor{black}{\textbf{[LOOP-END]}} Adjust loop end of the sample,
      Zoom to the actual pointer for fine tuning,
    }{
      \textcolor{black}{\textbf{[RUN]}} Select running mode (Trigger Continuous Gate Length),
      \textcolor{black}{\textbf{[GATE]}} Select Gate CV,
      NA,
      \textcolor{black}{\textbf{[NOTE-VOCT]}} Select VOCT input,
      \textcolor{black}{\textbf{[NOTE-OCT]}} Tune note by octave (+/- 8 octaves),
      \textcolor{black}{\textbf{[NOTE-NOTE]}} Tune sound by semitone (+12 semitones one octave),
      \textcolor{black}{\textbf{[NOTE-FINE]}} Fine-tune note in cents (+/- semitone),
      Select Audio Output Destiantion,
    }
}

\def\modnameshortsmp{SMP}
\def\modnamelongsmp{Sample Player}

\newcommand{\modsmp}{
  \newpage
  \modmakesection{smp}

  \paragraph*{}
  \modnamelongsmp{ }(SMP) is the sound source generating sound by playing samples at requested pitch.
  
  \showrefcardsection{smp}

  \subsection{Operation}
  
  \modnamelongsmp{ }reproduces samples at given pitch provided by Inputs:Note and trigerred by Inputs:Gate. Samples can be loaded from SD Card by pressing encoder (SELECT) down. After the sample is loaded from directory, rotating the encoder allows loading next and previous samples from the same directory.
  \subsection{Looping Samples and Grains}%
  \paragraph*{}Sample Player allows creating loops of samples controlled by CV thus converting it into granular sample player. The sample player can be looped by changing Loop Start and Loop End parameters via CV inputs effectively converting it into granular synthesiser. To control loop use knobs:
  \begin{itemize}
    \item \textbf{[LVL5 - Start]} start position of sample. When gate signal triggers sample it will start playing from this position. If Start is bigger than Loop Start then Loop Start becomes starting position for sample.
    \item \textbf{[LVL6 - Loop Start]} loop start position, after reaching loop end, sample will start playing from this point
    \item \textbf{[LVL7 - Loop End]} If loop end is set to value above 0 the sample looping will be enabled.
  \end{itemize}
  \paragraph*{}By adding CV control to any of the above parameters the size of grain can be modified during play with CV input for example LFO or LFS.

  \newpage
  \section*{\modname}  
  \tableofoptionsnew{\smpset}{{Settings}}
  \tableofoptionsnew{\smpinp}{{Inputs}}
  \tableofoptionsnew{\smpout}{{Outputs}}
}




\def\noiset{
  {{Audio Output}/{Noise Generatorのオーディオ出力}/{ノイズのオーディオ出力先を選択します。
  \newline 参照: \underline{\nameref{section:audiooutputs}}}},
}
% \def\inpclock{
%    {{Clock}/{Clock for timing oscillator's phase duration in BPM mode (otherwise 120BPM is used)}/{}}
% }

% \def\noiinp{
%   {{Unison}/{Unison mode (1-16 voices)}/{Enables unison mode and controls number of oscillators running in parallel.}},
%   {{Detune}/{Detune oscillators}/{Detunes oscillators in Unison mode by spreading them equally in Detune range. With detune set at maximum the spread range is 2 semi-tones. With odd number of oscillators the centre oscillator will always be following pitch from Note Input. For even number of oscillators there is no oscillator followin Note and each oscillator is detuned +/- from the center note. \newline $\bigstar$ \textit{With Unison running two oscillators, detune set to maximum (1.0) and pitch set at note D there is one oscillator playing note C\# and one playing note D\# (no oscillator is actually playing note C)}}},
%   {{Wavetable}/{Wavetable position}/{Adjusts position of waveform in wavetable. When controlled by CV allows changing texture of sound by scanning through set of waveforms contained in wavetable}},
%   {{Reset Osc}/{Reset Oscillator CV input}/{When connected with CV input and set to high level it resets phase of oscillator to 0. Useful to run oscillators in Sync}}%
% }

\def\noiinp{
  {{LPF}/{Low Pass Filter}/{CVコントロールのLow Pass Filter}},
  {{HPF}/{High Pass Filter}/{CVコントロールのHigh Pass Filter}}%
} 

\def\noiout{
    {{OSC}/{NOI出力}/{ノイズ信号の出力}},
}

\def\refcardnoi{
 \makereferencecard{
    \textcolor{black}{\textbf{[回す]}} 波形の切り替え(Triangle Square Sine),
  }{
    \textcolor{black}{\textbf{[ON/OFF]}} モジュールのオン/オフ (Bypass),
    \textcolor{black}{\textbf{[SET]}} モジュールの設定,
    \textcolor{black}{\textbf{[Inputs]}} モジュールの入力を構成する,
    \textcolor{black}{\textbf{[Back]}} パッチメニューに戻る,
  }{
    \textcolor{black}{\textbf{[LPF]}} ローパスフィルターのカットオフ周波数 - ノイズのカラーを調整します,
    \textcolor{black}{\textbf{[HPF]}} ハイパスフィルターのカットオフ周波数 - ノイズのカラーを調整します,
    NA,
    NA,
    NA,
    NA,
    NA,
    NA,
  }{
    \textcolor{black}{\textbf{[LPF]}} ローパスフィルターのCV入力を減衰,
    \textcolor{black}{\textbf{[HPF]}} ハイパスフィルターのCV入力を減衰,
    NA,
    NA,
    \textcolor{black}{\textbf{[LPF]}} ローパスフィルターのCV入力を選択,
    \textcolor{black}{\textbf{[HPF]}} ハイパスフィルターのCV入力を選択,
    NA,
    ノイズのオーディオ出力先を選択,
    }
}

\def\modnameshortnoi{NOI}
\def\modnamelongnoi{Noise Generator}

\newcommand{\modnoi}{
  \newpage
  \modmakesection{noi}

Noise Generator は、さまざまな音色のノイズを生成する音源です。
  
  \showrefcardsection{noi}

  \subsection{波形}
  \paragraph*{} ノイズ ジェネレーターは、さらにフィルタリングするためのベースとしてホワイト ノイズを生成します。 2 つの組み込みフィルター (ローパス フィルターとハイパス フィルター) を使用すると、ノイズの音色を CV 入力の助けを借りて調整できます。


  \newpage
  \section*{\modname}  
  \tableofoptionsnew{\noiset}{\optnameset}%
  \tableofoptionsnew{\noiinp}{\optnameinp}%
  \tableofoptionsnew{\noiout}{\optnameout}%
}




\def\dlyset{
  {{Audio Output}/{Delayのオーディオ出力}/{ディレイで処理された信号のオーディオ出力。オーディオ出力は、バッファされた遅延信号と合計されたオーディオ入力信号です。\newline 参照: \underline{\nameref{section:audiooutputs}}}},
  {{Audio Input}/{Delayをかけるオーディオ入力信号}/{エコーを生成するためにバッファリングされ減衰された信号と遅延および合計されるオーディオ信号。}}%
}

\def\dlyinp{
  {{Delay}/{Delayの長さ}/{0 と 4 の間の遅延期間の長さ。Delayはバッファのサイズです。}},
  {{Echo}/{Echoレベル}/{さらにレイヤリングするためにバッファリングされる信号のレベル}}%
}

\def\dlyout{
  {{NONE}/{}/{}}%
}

\def\refcarddly{
  \makereferencecard{
    \textcolor{black}{\textbf{[クリック]}} Exit
    }{
    \textcolor{black}{\textbf{[ON/OFF]}} モジュールのオン/オフ (バイパス),
    \textcolor{black}{\textbf{[SET]}} モジュールの設定,
    \textcolor{black}{\textbf{[Inputs]}} モジュールの入力を構成する,
    \textcolor{black}{\textbf{[Back]}} パッチメニューに戻る,
  }{
    \textcolor{black}{\textbf{[DLY]}} 遅延時間 0 秒~4 秒,
    \textcolor{black}{\textbf{[ECH]}} 遅延期間後に追加されたエコーの割合,
    NA,
    NA,
    NA,
    NA,
    NA,
    NA,
  }{%
    NA,
    NA,
    NA,
    NA,
    NA,
    NA,
    \textcolor{black}{\textbf{[AUDIO IN]}} オーディオ入力ソースの選択,
    \textcolor{black}{\textbf{[AUDIO OUT]}} オーディオ出力先の選択,
  }
}

\def\modnameshortdly{DLY}
\def\modnamelongdly{Delay}

\newcommand{\moddly}{
  \newpage
  \modmakesection{dly}

  \paragraph*{}
  Delay (DLY) エフェクトプロセッサは、信号をバッファリングし、それを合計して遅延エコー効果を提供します。
  \showrefcardsection{dly}

  \subsection{操作}
  \paragraph*{} ディレイ (DLY) エフェクト プロセッサは、入力信号を取得し、Input:Delay パラメータに基づいて要求された期間バッファリングし、Input:Echo パラメータで指定された減衰レベルでバッファリングされた信号と合計します。ディレイは可聴エコー効果を生み出します。

  \newpage
  \section*{\modname}
  \tableofoptionsnew{\dlyset}{\optnameset}
  \tableofoptionsnew{\dlyinp}{\optnameinp}
  \tableofoptionsnew{\dlyout}{\optnameout}
}


\def\dstset{
  {{Audio Output}/{Distortionのオーディオ出力}/{歪んだ信号の音声出力。オーディオ出力は、選択したアルゴリズムで歪んだオーディオ入力信号です。\newline See: \underline{\nameref{section:audiooutputs}}}},
  {{Audio Input}/{遅延させるオーディオ入力信号}/{エコーを生成するためにバッファリングされ減衰された信号と遅延および合計されるオーディオ信号。}}%
}

\def\dstinp{
  {{Drive}/{Signal Gain}/{歪みを処理する際のオーディオ入力信号のゲインのレベルで、効果がより顕著になります。}},
  {{Mix}/{Dry/Wet Mix}/{処理されていない (元の) 入力信号と歪んだオーディオ信号の混合量}}%
}

\def\dstout{
  {{NONE}/{}/{}}%
}

\def\refcarddst{
  \makereferencecard{
   \textcolor{black}{\textbf{[クリック]}} Exit
    }{
    \textcolor{black}{\textbf{[ON/OFF]}} モジュールのオン/オフ (バイパス),
    \textcolor{black}{\textbf{[SET]}} モジュールの設定,
    \textcolor{black}{\textbf{[Inputs]}} モジュールの入力を構成する,
    \textcolor{black}{\textbf{[Back]}} パッチメニューに戻る,
  }{
    \textcolor{black}{\textbf{[DLY]}} 遅延時間 0 秒~4 秒,
    \textcolor{black}{\textbf{[ECH]}} 遅延期間後に追加されたエコーの割合,
    NA,
    NA,
    NA,
    NA,
    NA,
    NA,
  }{%
    NA,
    NA,
    NA,
    NA,
    NA,
    NA,
   \textcolor{black}{\textbf{[AUDIO IN]}} オーディオ入力ソースの選択,
    \textcolor{black}{\textbf{[AUDIO OUT]}} オーディオ出力先の選択,
  }
}

\def\modnameshortdst{DST}
\def\modnamelongdst{Distortion}

\newcommand{\moddst}{
  \newpage
  \modmakesection{dst}

  \paragraph*{}
  Distortion (DST) Effect Processor は、信号を歪ませたり損傷させたりして粗いテクスチャを追加します。
  \showrefcardsection{dst}

  \subsection{操作}
  \paragraph*{} ディストーション (DST) エフェクト プロセッサは、入力信号を受け取り、選択したアルゴリズムを使用して歪ませます。 Wet Signal (歪んだ入力信号) は、Input:Mix パラメータで制御された比率で、Dry Signal (未処理の入力信号) とミックスされます。

  \newpage
  \section*{\modname}
  \tableofoptionsnew{\dstset}{\optnameset}
  \tableofoptionsnew{\dstinp}{\optnameinp}
  \tableofoptionsnew{\dstout}{\optnameout}
}


% \def\lfsbutmain{
%    {Edit}/{Invokes Shape Editor},
%    {Set}/{Open Settings},
%    {Inputs}/{Open Inputs},
%    {Back}/{Return to Patch Menu}
% }



\def\vcfset{
  {{Audio Output}/{VCFのオーディオ出力}/{VCF でフィルタリングされた音声の音声出力先を選択します。
  \newline 参照: \underline{\nameref{section:audiooutputs}}}},
  {{Audio Input}/{VCF を介してフィルタリングされるオーディオ信号}/{まざまな種類のフィルターでフィルター処理されるオーディオ信号 (任意の信号)}},
  {{Filter Type}/{Filterの選択}/{
    \makesupopt{{
      {{OP Lowpass, OP Highpass}/{One Pole filters.}/{単極の最も単純なフィルターですが、高周波を低減するのに優れた結果をもたらします}},
      % {{OP Highpass}/{One Pole High Pass filter.}/{The simpliest filter with single pole but giving nice results in reducing low frequency}},
      {{BQ Lowpass, Highpass, Bandpass, Low Shelf, High Shelf, Peaking, Notch, Allpass}/{Biquad filters}},
      % {{BQ Highpass}/{Standard triangle waveform.  The waveform is affected by Input: Skew and changes from Ramp through Triangle to Saw}},
      % {{BQ Bandpass}/{Standard triangle waveform.  The waveform is affected by Input: Skew and changes from Ramp through Triangle to Saw}},
      % {{BQ Low Shelf}/{Standard triangle waveform.  The waveform is affected by Input: Skew and changes from Ramp through Triangle to Saw}},
      % {{BQ High Shelf}/{Standard triangle waveform.  The waveform is affected by Input: Skew and changes from Ramp through Triangle to Saw}},
      % {{BQ Peaking}/{Standard triangle waveform.  The waveform is affected by Input: Skew and changes from Ramp through Triangle to Saw}},
      % {{BQ Notch}/{Standard triangle waveform.  The waveform is affected by Input: Skew and changes from Ramp through Triangle to Saw}},
      % {{BQ Allpass}/{Standard triangle waveform.  The waveform is affected by Input: Skew and changes from Ramp through Triangle to Saw}},
      {{DL Ladder}/{Diode Ladder filter}},
      {{MG Ladder}/{MG Type Ladder filter}},
      {{LD Lowpass 12, Highpass 12, Bandpass 12, Lowpass 24, Highpass 24, Bandpass 24}/{Ladder filters 12dB and 24dB versions}},
      % {{LD Lowpass 24}/{Standard triangle waveform.  The waveform is affected by Input: Skew and changes from Ramp through Triangle to Saw}},
      % {{LD Highpass 12}/{Standard triangle waveform.  The waveform is affected by Input: Skew and changes from Ramp through Triangle to Saw}},
      % {{LD Highpass 24}/{Standard triangle waveform.  The waveform is affected by Input: Skew and changes from Ramp through Triangle to Saw}},
      % {{LD Bandpass 12}/{Standard triangle waveform.  The waveform is affected by Input: Skew and changes from Ramp through Triangle to Saw}},
      % {{LD Bandpass 24}/{Standard triangle waveform.  The waveform is affected by Input: Skew and changes from Ramp through Triangle to Saw}}%
    }}%
  }}%
}
% \def\inpclock{
%    {{Clock}/{Clock for timing oscillator's phase duration in BPM mode (otherwise 120BPM is used)}/{}}
% }

% \def\vcfinp{
%   {{Unison}/{Unison mode (1-16 voices)}/{Enables unison mode and controls number of oscillators running in parallel.}},
%   {{Detune}/{Detune oscillators}/{Detunes oscillators in Unison mode by spreading them equally in Detune range. With detune set at maximum the spread range is 2 semi-tones. With odd number of oscillators the centre oscillator will always be following pitch from Note Input. For even number of oscillators there is no oscillator followin Note and each oscillator is detuned +/- from the center note. \newline $\bigstar$ \textit{With Unison running two oscillators, detune set to maximum (1.0) and pitch set at note D there is one oscillator playing note C\# and one playing note D\# (no oscillator is actually playing note C)}}},
%   {{Wavetable}/{Wavetable position}/{Adjusts position of waveform in wavetable. When controlled by CV allows changing texture of sound by scanning through set of waveforms contained in wavetable}},
%   {{Reset Osc}/{Reset Oscillator CV input}/{When connected with CV input and set to high level it resets phase of oscillator to 0. Useful to run oscillators in Sync}}%
% }



\def\vcfinp{
  {{NOTE}/{オシレーターのピッチコントロール}/{Note は、オシレーターのピッチまたは周波数を制御します。
  \newline 参照 \underline{\nameref{section:pitchcontrol}}
  \makesupopt{{
    {{VOCT}/{ノートコントロール用の1V/Oct入力/}},
    {{OCT}/{オクターブ単位のチューニングノート +/- 8 オクターブ}},
    {{NOTE}/{半音単位のチューニング音 +12 半音 (1 オクターブ)}},
    {{FINE}/{セント +/- 半音単位のチューニング ノート}}%
  }}%
  }},
  {{Cutoff}/{Cutoff Frequency}/{フィルターが信号のフィルター処理を開始する周波数境界}},
  {{Resonance}/{Resonance}/{信号の抑制または増強のレベル}},
  {{Gain}/{Pre-Gain of Signal}/{フィルタリング前の信号の増幅}},
} 

\def\vcfout{
  {{NONE}/{}}%
}
  
\def\refcardvcf{
  \makereferencecard{
    \textcolor{black}{\textbf{回す]}} フィルターの種類を切り替える
  }
  {
    \textcolor{black}{\textbf{[ON/OFF]}} モジュールのオン/オフ (バイパス),
    \textcolor{black}{\textbf{[SET]}} モジュールの設定,
    \textcolor{black}{\textbf{[Inputs]}} モジュールの入力を構成する,
    \textcolor{black}{\textbf{[Back]}} パッチメニューに戻る
  }
  {
    \textcolor{black}{\textbf{[CTO]}} カットオフレベルの調整,
    \textcolor{black}{\textbf{[RES]}} Resonanceの調整,
    \textcolor{black}{\textbf{[GAI]}} フィルターのゲインを調整する,
    NA,
    \textcolor{black}{\textbf{[VOCT]}} キー トラッキング CV の減衰,
    \textcolor{black}{\textbf{[VOCT]}} キートラッキング CV を選択,
    NA,
    NA,
  }
  {
    \textcolor{black}{\textbf{[CTO]}} Cutoff CVを減衰,
    \textcolor{black}{\textbf{[RES]}} Resonance CVを減衰,
    \textcolor{black}{\textbf{[GAI]}} Filter CVのゲインを減衰,
    \textcolor{black}{\textbf{[CTO]}} Cutoff CV,
    \textcolor{black}{\textbf{[RES]}} Resonance CV,
    \textcolor{black}{\textbf{[GAI]}} Filter CVのゲイン,
    オーディオ入力を選択,
    オーディオ出力を選択,
  }
}
    


\def\modnameshortvcf{VCF}
\def\modnamelongvcf{Voltage Controlled Filter}

\newcommand{\modvcf}{
  \newpage
  \modmakesection{vcf}

  \paragraph*{}
Voltage Controlleed Filter (VCF) はCutoff, Resonance, Gain を制御するデジタル フィルターのセットです。 

\showrefcardsection{vcf}
% \subsubsection*{LFS Settings}
\subsection{操作}電圧制御フィルター (VCF) は、デジタル領域に実装されたさまざまなフィルターのセットで、音の周波数を制限する制御を可能にします。

\subsection{トラッキング} VCF は、複数の方法で制御できるカットオフ周波数追跡メカニズムを実装していますが、Cufoff 周波数を制御するための 3 つの標準パラメーターがあります。

Cutoff CV \- これは Input:Cutoff パラメータの一部であり、ベース カットオフ周波数を変調するために、外部または内部の制御電圧を割り当てることができます。

Tracking V/OCT と Tracking CV は、同じ入力コントロール Input:Tracking の一部である 2 つの電圧制御パラメーターです。トラッキング V/OCT は、外部 (V/OCT ジャック) または内部 (内部モジュールの NOTE 出力) からのピッチに割り当てられ、そのアッテネーターで調整できます。逆にトラッキング CV は、エンベロープや LFO などの CV モジュレーターに割り当てられます。
\newline$\bigstar$ VCF の Input:Tracking と VCO または WTO の Input:NOTE ピッチコントロールに同じ V/OCT を割り当てることで、ローパスフィルターを使用したときに、弾いた音のキーに追従してピッチの変化にパンチのあるサウンドを作成できます。



  
  \newpage
  \section*{\modname}
  \tableofoptionsnew{\vcfset}{\optnameset}%
  \tableofoptionsnew{\vcfinp}{\optnameinp}%
  \tableofoptionsnew{\vcfout}{\optnameout}%
% \section*{\top}

% \subsubsection*{LFS Settings}

% \tableofoptionsnew{\vcoinpi}{\optnameinp}
}




\def\drcset{
  {{Audio Output}/{Audio Output of Drum Rack}/{Selects audio output destination for DRC. All 4 sample slots are sharing the same audio output where they get downmixed \newline See: \underline{\nameref{section:audiooutputs}}}},
}

\def\drcinp{
  {{Gate 1}/{Gate CV Input for Slot 1}/{Gate or trigger signal that initiates sample playback}},
  {{Pitch 1}/{Pitch Adjustment for Slot 1}/{relative pitch adjustment for sample playback}},
  {{Level 1}/{Amplitude (Level) for Slot 1}/{Level (Amplitude Modulation) for sample playback}},
  {{Steps 2}/{Number of Steps for Channel 2}/{Number of Steps within a channel.}},
  {{Beats 2}/{Number of Beats for Channel 2}/{Number of beats within a channel.}},
  {{Offset 2}/{Offest of first step for  Channel 2}/{Offset of first step for channel}},
  {{Steps 3}/{Number of Steps for Channel 3}/{Number of Steps within a channel.}},
  {{Beats 3}/{Number of Beats for Channel 3}/{Number of beats within a channel.}},
  {{Offset 3}/{Offest of first step for  Channel 3}/{Offset of first step for channel}},
  {{Steps 4}/{Number of Steps for Channel 4}/{Number of Steps within a channel.}},
  {{Beats 4}/{Number of Beats for Channel 4}/{Number of beats within a channel.}},
  {{Offset 4}/{Offest of first step for  Channel 4}/{Offset of first step for channel}},
}

\def\drcout{
  % {{Gate 1}/{Gate 1 Output}/{Output of Channel 1 setting gate to high upon beat}},
  % {{Gate 2}/{Gate 2 Output}/{Output of Channel 2 setting gate to high upon beat}},
  % {{Gate 3}/{Gate 3 Output}/{Output of Channel 3 setting gate to high upon beat}},
  % {{Gate 4}/{Gate 4 Output}/{Output of Channel 4 setting gate to high upon beat}}%
}

\def\refcarddrc{
  \makereferencecard{
    \textcolor{black}{\textbf{[ROTATE]}} Select sample slot
    \newline\textcolor{black}{\textbf{[CLICK]}} Enter file browser to select sample from SD card
  }{
    \textcolor{black}{\textbf{[On/Off]}} Controls module on and off,
    \textcolor{black}{\textbf{[SET]}} Settings of the module,
    \textcolor{black}{\textbf{[Inputs]}} Configure Inputs of the module,
    \textcolor{black}{\textbf{[Back]}} Return back to Patch Menu,
  }{
    \textcolor{black}{\textbf{[PITCH]}} Pitch Control for Slot 1,
    \textcolor{black}{\textbf{[PITCH]}} Pitch Control for Slot 2,
    \textcolor{black}{\textbf{[PITCH]}} Pitch Control for Slot 3,
    \textcolor{black}{\textbf{[PITCH]}} Pitch Control for Slot 4,
    \textcolor{black}{\textbf{[LEVEL]}} Level (Amplitude) Control for Slot 1,
    \textcolor{black}{\textbf{[LEVEL]}} Level (Amplitude) Control for Slot 2,
    \textcolor{black}{\textbf{[LEVEL]}} Level (Amplitude) Control for Slot 3,
    \textcolor{black}{\textbf{[LEVEL]}} Level (Amplitude) Control for Slot 4,
  }{%
    \textcolor{black}{\textbf{[GATE]}} Select Gate CV for Slot 1,
    \textcolor{black}{\textbf{[GATE]}} Select Gate CV for Slot 2,
    \textcolor{black}{\textbf{[GATE]}} Select Gate CV for Slot 3,
    \textcolor{black}{\textbf{[GATE]}} Select Gate CV for Slot 4,
    \textcolor{black}{\textbf{[OUT]}} Select Audio Output Destination,
    NA,
    NA,
  }
}



\def\modnameshortdrc{DRC}
\def\modnamelongdrc{Drum Rack}

\newcommand{\moddrc}{
  \newpage
  \modmakesection{drc}

  \paragraph*{}
  Drum Rack (DRC) is a 4 slot simple sample player with pitch adjustment and volume level control (amplitude modulation).

  \showrefcardsection{drc}

  \subsection{Operation}
  \paragraph*{} Drum rack consists of 4 slots for loading samples. Upon receiving trigger or gare signal on \textbf{GATE} input the sample starts as signle shot. Thera re only two parameters currently to be adjusted and both can be controlled either via setting static value or using CV to modulate them. Those parameters as Volume Level (Amplitude) and Pitch Correction. Pitch correction adjusts pitch of sample by +/- two octaves.
  % \newline$\blacksquare$ Without clock source the BPM is fixed at 120 BPM the pattern
  
  \newpage
  \section*{\modname}
  \tableofoptionsnew{\drcset}{{Settings}}
  \tableofoptionsnew{\drcinp}{{Inputs}}
  \tableofoptionsnew{\drcout}{{Outputs}}
}


\def\clkset{
  {{Clock Output}/{Clock Output}/{Selects audio output destination for CLK. The VOUT1-4 can carry clock out as they are DC coupled. Clock will output square clock signal through the selected output.\newline See: \underline{\nameref{section:audiooutputs}}}},
  {{Sync}/{Sync clock with external clock}/{Sync will synchronise clock module with external clock's BPM rate.}},
  {{Clocks PQN}/{Clocks Per Quarter Note for External Clock}/{Selection of clocks for external source of BPM. User needs to select the number of clocks per BPM the external source generates}}%
}

\def\clkinp{
  {{BPM}/{Beats Per Minute}/{Selection of Beats Per Minute setting}},
  {{Clock}/{External Clock Input}/{External Clock input used to calculate BPM of external source\newline$\blacksquare$ Use only with CVY1-4 inputs}}%
}

\def\clkout{
  {{Clock}/{Clock signal}/{Output of generated clock signal relative to BPM}}%
}


\def\refcardclk{
  \makereferencecard{
    NA,
    }{
    \textcolor{black}{\textbf{[ON/OFF]}} Turning module On or Off (Bypass),
    \textcolor{black}{\textbf{[SET]}} Settings of the module,
    \textcolor{black}{\textbf{[Inputs]}} Configure Inputs of the module,
    \textcolor{black}{\textbf{[Back]}} Return back to Patch Menu,
  }{
    \textcolor{black}{\textbf{[BPM]}} Adjust BPM Rate,
    NA,
    NA,
    NA,
    NA,
    NA,
    NA,
    NA,
  }{%
    \textcolor{black}{\textbf{[SYNC]}} Sync BPM to external BPM,
    \textcolor{black}{\textbf{[PPQN]}} Pulses per quarter note [BPM] for extrnal clock,
    NA,
    NA,
    \textcolor{black}{\textbf{[EXT CLK]}} External clock CV input selector,
    NA,
    NA,
    \textcolor{black}{\textbf{[OUT]}} Select Audio Output Destination,
  }
}

\def\modnameshortclk{CLK}
\def\modnamelongclk{Clock Generator}

\newcommand{\modclk}{
  \newpage
  \modmakesection{clk}

  \paragraph*{}
  Clock Generator (CLK) produces clock pulses based on BPM to synchronise modules.
   
  \showrefcardsection{clk}

  \subsection{Operation}
  \paragraph*{} Clock Generator (CLK) produces steady pulse signal of a rate related to the BPM (Beats Per Minute). Number of generated clocks vary depending on global Setting:Clocks Per Quarter Note (CPQN). BPM determines how many beats are per minute and CPQN determines how many clocks will be generated per beat.
  \newline$\blacksquare$ MIDI standard sets 24 clocks per quarter note however this value can be modified as different equipment uses different assumptions.
  \newline$\bigstar$ Clock should be supplied to modules like \modnamelink{lfs} or \modnamelink{ply} to change their default timing from 120 BPM to required one set in CLK
  \subsection{External Clock}
  \paragraph*{} When connecting Ext Clock Input \textbf{[INPUT:CLOCK]} the module will start displaying external clock BPM in accordance to configured CPQN (Clocks Per Quarter Note) in Settings of the module.
  After turning \textbf{[SET:Sync]} to ON the internal BPM can no longer be configured with the \textbf{[Input:BPM]} but it will use the value of external clock BPM
  \newline$\blacksquare$ When using external clock only usable are CVY1-4 inputs as those are low latency inputs. See: \underline{\nameref{section:connectors}}
  \subsection{Clock Output}
  \paragraph*{}Clock Module can provide master clock to external modules via Clock Output. Clock Output can be configured via settings to one of VOUT1-4 3.5mm outputs or to VBuf \underline{\nameref{section:audiooutputs}}

  \newpage
  \section*{\modname}
  \tableofoptionsnew{\clkset}{{Settings}}
  \tableofoptionsnew{\clkinp}{{Inputs}}
  \tableofoptionsnew{\clkout}{{Outputs}}
}


\def\gatset{
  {{NONE}/{}/{}}%
}

\def\gatinp{
  {{Clock}/{クロック CV ソース}/{Pulse CV source to base divisions on}},
  {{Reset}/{CV ソースをリセット}/{すべての位置を開始に設定することにより、すべてのディバイダの位置を同期するリセット信号}},
  {{Intverval 1}/{パルスインターバル}/{高ゲート時のゲートインターバル}},
  {{Gate Len 1}/{パルスの長さ}/{チャンネル内のビート数}},
  {{Delay 1}/{パルスのトリガー前のディレイ}/{ディレイは、パルス信号がゲートをHighにするまでの期間}},
  {{Intverval 2}/{パルスインターバル}/{高ゲート時のゲートインターバル}},
  {{Gate Len 2}/{パルスの長さ}/{チャンネル内のビート数.}},
  {{Delay 2}/{パルスのトリガー前のディレイ}/{ディレイは、パルス信号がゲートをHighにするまでの期間}},
  {{Intverval 3}/{パルスインターバル}/{高ゲート時のゲートインターバル}},
  {{Gate Len 3}/{パルスの長さ}/{チャンネル内のビート数}},
  {{Delay 3}/{パルスのトリガー前のディレイ}/{ディレイは、パルス信号がゲートをHighにするまでの期間}},
  {{Intverval 4}/{パルスインターバル}/{高ゲート時のゲートインターバル}},
  {{Gate Len 4}/{パルスの長さ}/{チャンネル内のビート数}},
  {{Delay 4}/{パルスのトリガー前のディレイ}/{ディレイは、パルス信号がゲートをHighにするまでの期間}}%
}

\def\gatout{
  {{Gate 1}/{Gate 1出力}/{チャンネル 1 パルス信号の出力}},
  {{Gate 2}/{Gate 2 出力}/{チャンネル 2 パルス信号の出力}},
  {{Gate 3}/{Gate 3 出力}/{チャンネル 3 パルス信号の出力}},
  {{Gate 4}/{Gate 4 出力}/{チャンネル 4 パルス信号の出力}}%
}

\def\refcardgat{
  \makereferencecard{
    \textcolor{black}{\textbf{[回す]}}Gate Dividerチャンネル 1 ~ 4 を選択,
    }{
    \textcolor{black}{\textbf{[ON/OFF]}} モジュールのオン/オフ (バイパス),
    \textcolor{black}{\textbf{[SET]}} モジュールの設定,
    \textcolor{black}{\textbf{[Inputs]}} モジュールの入力を構成する,
    \textcolor{black}{\textbf{[Back]}} パッチメニューに戻る,
  }{
    \textcolor{black}{\textbf{[INTV1]}} チャンネル 1 のインターバル,
    \textcolor{black}{\textbf{[GLEN1]}} チャンネル 1 のゲートの長さ,
    \textcolor{black}{\textbf{[INTV2]}} チャンネル 2 のインターバル,
    \textcolor{black}{\textbf{[GLEN2]}} チャンネル 2 のゲートの長さ,
    \textcolor{black}{\textbf{[INTV3]}} チャンネル 3 のインターバル,
    \textcolor{black}{\textbf{[GLEN3]}} チャンネル 3 のゲートの長さ,
    \textcolor{black}{\textbf{[INTV4]}} チャンネル4 のインターバル,
    \textcolor{black}{\textbf{[GLEN4]}} チャンネル 4 のゲートの長さ,
  }{%
    \textcolor{black}{\textbf{[DLY1]}} Channel 1のDelay,
    \textcolor{black}{\textbf{[DLY2]}} Channel 2のDelay,
    \textcolor{black}{\textbf{[DLY3]}} Channel 3のDelay,
    \textcolor{black}{\textbf{[DLY4]}} Channel 4のDelay,
    \textcolor{black}{\textbf{[CLK]}}CLOCKソースCVを選択,
    \textcolor{black}{\textbf{[RST]}} RESETソースCVを選択,
    NA,
    NA,
  }
}

\def\modnameshortgat{GAT}
\def\modnamelonggat{Gate Divider}

\newcommand{\modgat}{
  \newpage
  \modmakesection{gat}

  \paragraph*{}
 Gate Divider (GAT) は、パルス クロック信号 (クロック) をより長い周期のパルスクロック信号に分割します。

  \refcardgat%

  \subsection{操作}
  \paragraph*{} Gate Divider (GAT) は、入力されたパルス信号 (通常はクロック) を分割してパルス信号 (位相幅の異なるSquare波) を生成します。入力信号は 4 チャネルのカウンターをトリガーし、設定されたパラメーターに基づいて、位相の幅が変更された長いパルス信号 (負と正の両方) を作成します。
  \newline$\blacksquare$ このような分割は、クロック信号を等間隔の休符を持つ等間隔の音符、または特定の音符の長さを持つ等間隔の音符トリガーに変換することと見なすことができます。
  \newline$\bigstar$ 分割されたゲートは、繰り返しのタイム ジェネレーター (ドラム マシン) として使用される繰り返しパターンの優れたソースです。gate dividerからドラム サンプルをトリガーし、gate divider divisionsを異なるノートに設定すると、通常のビートパターンを作成できます。

  \newpage
  \section*{\modname}
  \tableofoptionsnew{\gatset}{\optnameset}
  \tableofoptionsnew{\gatinp}{\optnameinp}
  \tableofoptionsnew{\gatout}{\optnameout}
}


\def\eucset{
  {{Step Length}/{1 ステップの長さ}/{1ステップの長さ、パターンは任意の数の等しい長さで構成されています}}%
  % {{Gate length}/{Length of gate}/{
  %   \makesupopt{{
  %     {{Trigger}/{Trigger only}/{Unsustained signal when the pulse occurs}}%
  %   }}%
  % }}%
}

\def\eucinp{
  {{Clock}/{クロック CV ソース}/{Poly Rhythmを他のモジュールと同期させるためのクロックソース}},
  {{Reset}/{CV ソースをリセット}/{トリガー時にパターン内のすべてのチャンネルの位置をリセットする}},
  {{Steps 1}/{チャンネル 1 のステップ数}/{チャンネル内のステップ数}},
  {{Beats 1}/{チャンネル 1 のビート数}/{チャンネル内のビート数}},
  {{Offset 1}/{チャンネル1 の最初のステップのオフセット}/{チャンネルの最初のステップのオフセット}},
  {{Steps 2}/{チャンネル 2 のステップ数}/{チャンネル内のステップ数}},
  {{Beats 2}/{チャンネル 2 のビート数}/{チャンネル内のビート数}},
  {{Offset 2}/{チャンネル2 の最初のステップのオフセット}/{チャンネルの最初のステップのオフセット}},
  {{Steps 3}/{チャンネル 3 のステップ数}/{チャンネル内のステップ数}},
  {{Beats 3}/{チャンネル 3 のビート数3}/{チャンネル内のビート数}},
  {{Offset 3}/{チャンネル3 の最初のステップのオフセット3}/{チャンネルの最初のステップのオフセット}},
  {{Steps 4}/{チャンネル 4のステップ数}/{チャンネル内のステップ数}},
  {{Beats 4}/{チャンネル 4 のビート数}/{チャンネル内のビート数}},
  {{Offset 4}/{チャンネル4 の最初のステップのオフセット}/{チャンネルの最初のステップのオフセット}},
}

\def\eucout{
  {{Gate 1}/{Gate 1出力}/{ビートに対してゲートをHighにするチャネル1出力}},
  {{Gate 2}/{Gate 2 出力}/{ビートに対してゲートをHighにするチャネル2出力}},
  {{Gate 3}/{Gate 3 出力}/{ビートに対してゲートをHighにするチャネル3出力}},
  {{Gate 4}/{Gate 4 出力}/{ビートに対してゲートをHighにするチャネル4出力}}%
}

\def\refcardeuc{
  \makereferencecard{
    NA%\textcolor{black}{\textbf{[ROTATE]}} Turn note on or off and in Scale Selector turn to Scale Select mode
  }{
    \textcolor{black}{\textbf{[RESET]}} パターン内のすべてのチャンネルが開始位置にリセット,
    \textcolor{black}{\textbf{[SET]}} モジュールの設定,
    \textcolor{black}{\textbf{[Inputs]}} モジュールの入力を構成する,
    \textcolor{black}{\textbf{[Back]}} パッチメニューに戻る,
  }{
    \textcolor{black}{\textbf{[STEP]}} チャンネル 1 のステップ数,
    \textcolor{black}{\textbf{[BEAT]}} チャンネル 1 のビート数,
    \textcolor{black}{\textbf{[STEP]}} チャンネル 2 のステップ数2,
    \textcolor{black}{\textbf{[BEAT]}} チャンネル 2 のビート数,
    \textcolor{black}{\textbf{[STEP]}} Nチャンネル 3 のステップ数,
    \textcolor{black}{\textbf{[BEAT]}} チャンネル 3 のビート数,
    \textcolor{black}{\textbf{[STEP]}} チャンネル 4 のステップ数,
    \textcolor{black}{\textbf{[BEAT]}}チャンネル 4 のビート数,
  }{%
    \textcolor{black}{\textbf{[OFFSET1]}} チャンネル1のオフセット開始,
    \textcolor{black}{\textbf{[OFFSET2]}} チャンネル2のオフセット開始,
    \textcolor{black}{\textbf{[OFFSET3]}} チャンネル3のオフセット開始,
    \textcolor{black}{\textbf{[OFFSET4]}} チャンネル4のオフセット開始,
    \textcolor{black}{\textbf{[CLK]}} CLOCKソースCVを選択,
    \textcolor{black}{\textbf{[LEN]}} ステップの長さを調整する,
    NA,
    NA,
  }
}



\def\modnameshorteuc{EUC}
\def\modnamelongeuc{Euclidean Rhythm Generator}

\newcommand{\modeuc}{
  \newpage
  \modmakesection{euc}

  \paragraph*{}
  Euclidean Rhythm Generator(EUC) は、ユークリッドのアルゴリズムに基づいてリズムパターンを生成し、期間を均等に分割します。

  \showrefcardsection{euc}

  \subsection{操作}
  \paragraph*{} ポリリズムは、パターンの長さを均等なステップに分割することにより、4 チャンネルのリズムを生成します。パターンは、指定された BPM (拍) での小節数 (4 拍または 4 分音符) をカバーする期間として定義されます。チャンネルのステップ数は、Inputs:Beats X (X はチャンネル 1 ~ 4 の番号) で制御できます。
  \newline$\blacksquare$ クロックソースがない場合、BPM は 120 BPM に固定されます。
  
  \newpage
  \section*{\modname}
  \tableofoptionsnew{\eucset}{\optnameset}
  \tableofoptionsnew{\eucinp}{\optnameinp}
  \tableofoptionsnew{\eucout}{\optnameout}
}


\def\plyset{
  {{Pattern Length}/{barでのパターンの長さ}/{バーでのパターンの長さ。各パターンは Inputs:Pulses で設定されたステップ数に分割されます}},
  {{Gate length}/{gateの長さ}/{
    \makesupopt{{
      {{Trigger}/{Triggerのみ}/{パルス発生時の非持続信号}}%
    }}%
  }}%
}

\def\plyinp{
  {{Clock}/{Clock CV Source}/{Poly Rhythmを他のモジュールと同期させるためのクロックソース}},
  {{Reset}/{Reset CV Source}/{トリガー時にパターン内のすべてのチャンネルの位置をリセット}},
  {{Pulses 1}/{チャンネル 1 のパルス数}/{パターンが分割され、各ステップの開始時にビートを生成するステップの数}},
  {{Pulses 2}/{チャンネル 2 のパルス数}/{パターンが分割され、各ステップの開始時にビートを生成するステップの数}},
  {{Pulses 3}/{チャンネル 3 のパルス数}/{パターンが分割され、各ステップの開始時にビートを生成するステップの数}},
  {{Pulses 4}/{チャンネル 4 のパルス数}/{パターンが分割され、各ステップの開始時にビートを生成するステップの数}}%
}

\def\plyout{
  {{Gate 1}/{Gate 1 出力}/{ビートに対してPoly RhythmゲートをHighにする出力}},
  {{Gate 2}/{Gate 2 出力}/{ビートに対してPoly RhythmゲートをHighにする出力}},
  {{Gate 3}/{Gate 3 出力}/{ビートに対してPoly RhythmゲートをHighにする出力}},
  {{Gate 4}/{Gate 4 出力}/{ビートに対してPoly RhythmゲートをHighにする出力}}%
}

\def\refcardply{
  \makereferencecard{
    NA%\textcolor{black}{\textbf{[ROTATE]}} Turn note on or off and in Scale Selector turn to Scale Select mode
  }{
    \newline\textcolor{black}{\textbf{[RESET]}} ボタンを放すと、パターン内のすべてのチャンネルがリセットされます,
    \newline\textcolor{black}{\textbf{[SET]}} モジュールの設,
    \newline\textcolor{black}{\textbf{[Inputs]}} モジュールの入力を構成する,
    \newline\textcolor{black}{\textbf{[Back]}} パッチメニューに戻る,
  }{
    チャネル 1 のステップ数を選択,
    チャネル 2 のステップ数を選択,
    チャネル 3 のステップ数を選択,
    チャネル 4のステップ数を選択,
    NA,
    NA,
    NA,
    NA,
  }{%
    \textcolor{black}{\textbf{[PLEN]}}パターンの長さを選択,
    NA,
    NA,
    NA,
    \textcolor{black}{\textbf{[CLK]}} CLOCKソースCVを選択,
    NA,
    NA,
    NA,
  }
}


\def\modnameshortply{PLY}
\def\modnamelongply{Polyrhythm}

\newcommand{\modply}{
  \newpage
  \modmakesection{ply}  
  \def\modname{PLY - Polyrhythm}% chktex 8
  

  \paragraph*{}
  Polyrhythm (PLY) は、パターン内で対照なリズムを生成します。

  \showrefcardsection{ply}

  \subsection{操作}
  \paragraph*{} Polyrhythmは、パターンの長さを入力パラメータ Inputs:Steps で決定される等しいステップ数に分割することにより、4 チャンネルのリズムを生成します。パターンは、指定された BPMでの小節数 (4 拍または 4 分音符) をカバーする期間として定義されます。チャンネルのステップ数は、Inputs:Beats X (X はチャンネル 1 ~ 4 の番号) で制御できます。
  \newline$\blacksquare$ クロックソースがない場合、BPM は 120 BPM に固定されます。
  
  \newpage
  \section*{\modname}
  \tableofoptionsnew{\plyset}{\optnameset}
  \tableofoptionsnew{\plyinp}{\optnameinp}
  \tableofoptionsnew{\plyout}{\optnameout}
}


\def\rngset{
  {{Pattern Length}/{Length of pattern in bars}/{Length of pattern in bars. Each pattern is divided into number of steps configured by Inputs:Pulses}},
  {{Gate length}/{Length of gate}/{
    \makesupopt{{
      {{Trigger}/{Trigger only}/{Unsustained signal when the pulse occurs}}%
    }}%
  }}%
}

\def\rnginp{
  {{Clock}/{Clock CV Source}/{Clock source to generate notes automatically and measure length of notes}},
  {{Trigger}/{Trigger CV Source}/{Triggers generation of next note, if Trigger is not connected notes and rests get generated automatically upon their period ends}}%
}

\def\rngout{
  {{Pitch}/{Pitch Output of generated note}/{Output of generated random pitch limited by setting of selected musical scale}},
  {{Gate}/{Gate Output of generated note}/{Output of Poly Rhythm setting gate to high upon beat}},
  {{Velocity}/{Velocity Output of generated note}/{Output of Poly Rhythm setting gate to high upon beat}}%
}


\def\refcardrng{
  \makereferencecard{
    \textcolor{black}{\textbf{[ROTATE]}} Move between scale selector and note lengths
    \newline\textcolor{black}{\textbf{[CLICK]}} Turn on Scale Selector or Enable Note/Rest Lengths for generation
    }{
    \textcolor{black}{\textbf{[ON/OFF]}} Turning module On or Off (Bypass),
    \textcolor{black}{\textbf{[SET]}} Settings of the module,
    \textcolor{black}{\textbf{[Inputs]}} Configure Inputs of the module,
    \textcolor{black}{\textbf{[Back]}} Return back to Patch Menu,
  }{
    NA,
    NA,
    NA,
    NA,
    NA,
    NA,
    NA,
    NA,
  }{%
    \textcolor{black}{\textbf{[SCL]}} Select Scale,
    \textcolor{black}{\textbf{[NOTES]}} Select note lengths that will be included in randomiser,
    \textcolor{black}{\textbf{[RESTS]}} Select rests lengths to be included in randomiser,
    NA,
    \textcolor{black}{\textbf{[TRG]}} Select Trigger CV - Ignores CLOCK,
    \textcolor{black}{\textbf{[PLEN]}} Select pattern length,
    NA,
    NA,
  }
}


\def\modnameshortrng{RNG}
\def\modnamelongrng{Rendom Note Generator}

\newcommand{\modrng}{
  \newpage
  \modmakesection{rng}

  \paragraph*{}
  Rendom Note Generator (RNG) produces random notes.

  \showrefcardsection{rng}

  \subsection{Operation}
  \paragraph*{} Random Note Generator (RNG) produces random values for output of pitch, gate length and velocity. The randomistaion happens every time upon receiving trigger signal at Input:Trigger. The randomisation process can be controlled by selecting range of available notes and rests as well as predefines musical scale to limit generated pitches. 
  \newline$\blacksquare$ Without clock source the BPM is fixed at 120 BPM the pattern
  \newline$\bigstar$ The musical scale selection is not necessary as routing output of RNG through QNT (Quantiser) and limit available pitch there gives much better flexibility.

  \newpage
  \section*{\modname}
  \tableofoptionsnew{\rngset}{{Settings}}
  \tableofoptionsnew{\rnginp}{{Inputs}}
  \tableofoptionsnew{\rngout}{{Outputs}}
}


\def\qntset{
  {{Mode}/{操作モード}/{
    \makesupopt{{
      {{Simple}/{入力ピッチが整列されるノートの選択に基づいてクオンタイズが行われる動作モード}},%
      {{Advanced}/{クオンタイズが Scala (SCL) ファイルに基づく場合の動作モード}}%
    }}%
  }},
  {{Scale}/{シンプルモードの音階}/{
    \makesupopt{{
      {{Custom}/{ユーザーが選択したノート}},
      {{Chromatic}/{Chromaticスケール}},
      {{Major}/{Majorスケール}}
      {{Natural Minor}/{Natural Majorスケール}}
      {{Harmonic Minor}/{Harmonic Majorスケール}}
      {{Melodic Minor}/{Melodic Majorスケール}}
    }}%
  }},
  {{Key}/{選択したスケールのキーを変更 (カスタム スケールを除く)}}%
}

\def\qntinp{
  {{NOTE}/{クオンタイズのためのノート入力}/{入力ノートは、クオンタイズ前に調整およびチューニングできます。
  \newline 参照: \underline{\nameref{section:pitchcontrol}}
  \makesupopt{{
    {{VOCT}/{ノートコントロール用の1V/Oct入力/}},
    {{OCT}/{オクターブ単位のチューニングノート +/- 8 オクターブ}},
    {{NOTE}/{半音単位のチューニングノート +12 半音 (1 オクターブ)}},
    {{FINE}/{セント単位のチューニングノート +/- 1 半音}}%
  }}%
  }}%
}

\def\qntout{
  {{NOTE}/{Quantised Note}/{クオンタイズされたノート (クオンタイズ スケールの最も近いピッチに合わせられたノート)}}%
}

\def\refcardqnt{
  \makereferencecard{\textcolor{black}{\textbf{回す}} ノートをオン/オフ、スケールセレクターではスケールセレクトモードに切り替え}{%
  \newline\textcolor{black}{\textbf{[Mode]}} シンプルモードとアドバンスモードの切り替え,
  \newline\textcolor{black}{\textbf{[SET]}} モジュールの設定,
  \newline\textcolor{black}{\textbf{[Inputs]}} モジュールの入力を構成する,
  \newline\textcolor{black}{\textbf{[Back]}} パッチメニューに戻る}{
    NA,
    有効化/無効化 Note C\#,
    有効化/無効化 Note D\#,
    NA,
    有効化/無効化 Note F\#,
    有効化/無効化 Note G\#,
    有効化/無効化 Note A\#,
    \textcolor{black}{\textbf{[KEY]}} スケールのキーを変更 / Base Frequencyの変更
    }
    {%
      有効化/無効化 Note C,
      有効化/無効化 Note D,
      有効化/無効化 Note E,
      有効化/無効化 Note F,
      有効化/無効化 Note G,
      有効化/無効化 Note A,
      有効化/無効化 Note B,
      \textcolor{black}{\textbf{[NOTE]}} NOTE CV入力を選択,
  }
}

\def\refcardqntadv{
  \makereferencecard{\textcolor{black}{\textbf{[回す]}} 現在のディレクトリ内の SCL ファイルを選択する  \newline\textcolor{black}{\textbf{[クリック]}} ファイルブラウザに入る}{%
  \newline\textcolor{black}{\textbf{[Mode]}} シンプルモードに切り替え,
  \newline\textcolor{black}{\textbf{[SET]}} モジュールの設定,
  \newline\textcolor{black}{\textbf{[Inputs]}} モジュールの入力を構成する,
  \newline\textcolor{black}{\textbf{[Back]}} パッチメニューに戻る}{
    \textcolor{black}{\textbf{[FREQ]}} ベース周波数の調整  (coarse),
    \textcolor{black}{\textbf{[FREQ]}} ベース周波数を調整する (fine),
    NA,
    NA,
    NA,
    NA,
    NA,
    NA%
  }
  {%
    NA,
    NA,
    NA,
    NA,
    NA,
    NA,
    NA,
    \textcolor{black}{\textbf{[NOTE]}} NOTE CV入力を選択%
  }
}


\def\modnameshortqnt{QNT}
\def\modnamelongqnt{Quantiser}

\newcommand{\modqnt}{
  \newpage
  \modmakesection{qnt}
  
  \paragraph*{} Quantiser (QNT) は、入力ピッチを事前に定義された必要なピッチに揃えます。クオンタイザーは、入力ピッチを進行中の限定/設定済みのピッチのセットに制限します。
  
  \subsection{操作}%
  \paragraph*{}Quantiser(QNT) を使用すると、入力ピッチを選択した一連の出力ピッチに修正できます。通常、特定の音階でのみ使用可能な音符の数を制限するために使用されます。 Quantizer は Input:NOTE からピッチを取得し、近似法によってスケール内の最も近い対応するピッチを見つけます。
  
  \subsection{Operating Modes}%
  \paragraph*{}Quantizer は、\textbf{\textit{Simple}} モードまたは \textbf{\textit{Advanced}} モードで動作できます。
  
  \subsection{Simple Mode}%
  \showrefcardsubsection{qnt}%
  \subsubsection{Musical Scales}
  \paragraph*{} 現在、クオンタイザーで事前構成されている音階はごくわずかです。事前設定された各スケールは、適切な \textbf{Key} に調整することもできます (LVL8 ノブまたは設定を使用)。.

  \subsubsection{Custom Scales}
  \paragraph*{} カスタム スケールは、ユーザーが設定したメモのセットです。ノートに対応するノブを回すと (上図のように)、ノートのオンとオフが切り替わります。ノートの構成は、エンコーダー [SELECT] を回転させ、選択したノートでエンコーダーをクリックすることによっても行うことができます。
  \newline$\blacksquare$ ノブを回すかノートをクリックすると、モジュールが自動的にカスタムスケールに切り替わります。
  \newline$\bigstar$ クオンタイザーで選択されたノートが 1 つだけの場合、たとえばノート C で入力ノートが E4 の場合、出力ノートは C4 になります。ただし、ノート G4 の場合、C5 は C4 よりも G4 に近いため (近似値)、出力ノートは C5 になります。

  \subsection{Advanced Mode}%
  \showrefcardsubsection{qntadv}%
  \paragraph*{}Advanced modeでは、シンセサイザーのチューニングおよびマイクロチューニングの業界標準である Scala (SCL) ファイルを使用します。 Scala ファイルの最大のデータベースには、5000 以上のファイルが含まれています。
  \newline$\blacksquare$ \textbf{注意:} Center はメモリが限られているため、フォルダーごとに大量のファイルがあるフォルダーを読み取ることができません。 The Centerリポジトリから Scala で準備されたファイルを使用してください。%
  \newline$\bigstar$ こちらから: \underline{\url{https://github.com/1V-Oct/the_centre_waveforms/releases/tag/0.0.2}} 
  
  \begin{center}
    \centering\textbf{\textit{以下の QR コードをスキャンして、ダウンロード先にアクセスしてください}}
  \end{center}
  \begin{center}
    \qrcode{https://github.com/1V-Oct/the_centre_waveforms/releases/tag/0.0.2}
  \end{center}
  \paragraph*{}このモードでは、ユーザーが SD カードから SCL ファイルを選択すると、SCL ファイルで事前定義されたノートの比率が画面に表示されます。 
  ノブ \textbf{LVL1 - Coarse} と \textbf{LVL2 - Fineı} を使用してベース周波数を調整します。
%    \subsection*{LFS \- Low Frequency Shaper (Low Frequency Oscillator generating irregular shapes)}
% This module creates low frequency oscillations of irregular shapes. The shapes can be either edited by user or imported from .shp (Serum) and .vitallfo (Vital) formats.

% To load a shape press \textbf{[SELECT]} (Encoder down) and navigate in file browser to directory with shapes and press \textbf{[SELECT]} again to load the desired shape.

% Rotating Encoder cycles through shapes in current directory.

% \subsubsection*{Main Menu}

% \tableofbuttons{\lfsbutmain}
\newpage
\section*{\modname}
\tableofoptionsnew{\qntset}{\optnameset}
\tableofoptionsnew{\qntinp}{\optnameinp}
\tableofoptionsnew{\qntout}{\optnameout}
% \setshowall{\lfsinfo}
}


\def\arpset{
  {{Timing Mode}/{Duration of arpeggiator pattern}/{
    \makesupopt{{
      {{BPM}/{Timing based on note duration calculated from BPM which is controlled by steps}},
      {{Hz}/{Frequency at which pattern repeats}}%
   }}
  }},
  {{Note Length}/{Fixed note or pattern length}/{
    Switch between fixed note or pattern duration
    \makesupopt{{
      {{Off}/{Fixed \textbf{Pattern} length is selected, note duration is pattern duration divided by number of steps}},
      {{On}/{Fixed \textbf{Note} length is selected, pattern duration equals note duration times number of steps}}%
   }}%
  }}%
} 

\def\arpinp{
  {{NOTE}/{Base note}/{Input note that is a base note for arpeggiator to create sequence.
  \newline See: \underline{\nameref{section:pitchcontrol}}
  \makesupopt{{
    {{VOCT}/{1V/Oct input for note control/}},
    {{OCT}/{Tuning note by octaves +/- 8 octaves}},
    {{NOTE}/{Tuning note by semitones +12 semitones (one octave)}},
    {{FINE}/{Tuning note by cents +/- one semitone}}%
  }}%
  }},
  {{Clock}/{Clock CV Source}/{Clock source to synchronise with other modules.}},
  {{Reset}/{Reset CV Source}/{Reset position to step 0}},
  {{Duration}/{Length of Pattern}/{Duration of pattern measured either in frequency (Hz) or note duration (BPM)}},
  {{Position}/{Position CV Source for pattern step}/{External CV to control position of pattern}}%
}

\def\arpout{
  {{NTE}/{Pitch Output of generated note}/{Output of current step's pitch  calculated by adding step's semitones to played note}},
  {{GTE}/{Gate Output of generated note}/{Output of Poly Rhythm setting gate to high upon beat}},
  {{VEL}/{Velocity Output of generated note}/{Output of velocity/modulation parameter associated with current step}}%
}

\def\refcardarp{
  \makereferencecard{\textcolor{black}{\textbf{[ROTATE]}} Alternate between editing pattern's semitones and velocity values\newline\textcolor{black}{\textbf{[CLICK]}} Exit
    }{
    \textcolor{black}{\textbf{[ON/OFF]}} Turning module On or Off (Bypass),
    \textcolor{black}{\textbf{[SET]}} Settings of the module,
    \textcolor{black}{\textbf{[Inputs]}} Configure Inputs of the module,
    \textcolor{black}{\textbf{[Back]}} Return back to Patch Menu,
  }{
    \textcolor{black}{\textbf{\mbox{[STEP 1]}}} Adjusts step in semitones (-24 to +24) or Modulation (Velocity),
    \textcolor{black}{\textbf{\mbox{[STEP 2]}}} Adjusts step in semitones (-24 to +24) or Modulation (Velocity),
    \textcolor{black}{\textbf{\mbox{[STEP 3]}}} Adjusts step in semitones (-24 to +24) or Modulation (Velocity),
    \textcolor{black}{\textbf{\mbox{[STEP 4]}}} Adjusts step in semitones (-24 to +24) or Modulation (Velocity),
    \textcolor{black}{\textbf{\mbox{[STEP 5]}}} Adjusts step in semitones (-24 to +24) or Modulation (Velocity),
    \textcolor{black}{\textbf{\mbox{[STEP 6]}}} Adjusts step in semitones (-24 to +24) or Modulation (Velocity),
    \textcolor{black}{\textbf{\mbox{[STEP 7]}}} Adjusts step in semitones (-24 to +24) or Modulation (Velocity),
    \textcolor{black}{\textbf{\mbox{[STEP 8]}}} Adjusts step in semitones (-24 to +24) or Modulation (Velocity),
    }{%
    \textcolor{black}{\textbf{[STEPS]}} Adjusts number of steps (between 1 and 8),
    \textcolor{black}{\textbf{[DURATION]}} Adjust duration of the arpeggiator sequence,
    \textcolor{black}{\textbf{[TIMING]}} Changes timing of pattern between time based (Hz) and note duration based (BPM),
    \textcolor{black}{\textbf{[CLK]}} Select CLOCK source CV,Select Trigger CV,
    \textcolor{black}{\textbf{[NOTE-OCT]}} Tune note by octave (+/- 8 octaves),
    \textcolor{black}{\textbf{[NOTE-NOTE]}} Tune sound by semitone (+12 semitones one octave),
    \textcolor{black}{\textbf{[NOTE-FINE]}} Fine-tune note in cents (+/- semitone),
    \textcolor{black}{\textbf{[NOTE-VOCT]}} Select CV input for note's V/OCT,
  }
}

\def\modnameshortarp{ARP}
\def\modnamelongarp{Arpeggiator}

\newcommand{\modarp}{
  \newpage
  \modmakesection{arp}

  \paragraph*{}
  Arpeggiator (ARP) cycles though a set of notes defined in short pattern of defined duration.
  \showrefcardsection{arp}

  \subsection{Operation}
  \paragraph*{} Arpeggiator (ARP) is a pitch manipulation utility that creates a repeptetive music pattern conisiting of up to 8 notes that are defined as a difference in semitones to the played note. The steps can be adjusted with knobs and the number of steps can vary between 1 and 8 steps. Duration of the pattern can be set by specyfying duration of whole pattern by frequency of oscillation of pattern in Hertz (Hz) or duration of pattern measured in length of notes or bars based on timing coming from clock source (BPM).
  Use encoder (SELECT) to change between editing pattern's semitones and velocity values. 
  \newline$\blacksquare$ Every step Arpeggiator will generate Gate signal that lasts for the whole duration of the step.
  \newline$\blacksquare$ Velocity value is just an arbitrary value assigned to each step and can be used for modulating any parameter within The Centre.
  \newpage
  \subsection{Arpeggiator Screen}
  \begin{figure}[h]
    \centering
    \includegraphics[width=0.35\linewidth]{arp1.png}
    \caption{Arpeggiator with rest set on STEP 1}
    \label{fig:arprests}
  \end{figure}
  % \FloatBarrier
  % \newpage
  % \begin{wrapfigure}{r}{0.5\textwidth}
  %   \centering
  %   \includegraphics[width=0.7\linewidth]{arp1.png}
  %   \caption{Arpeggiator with rest set on STEP 1}
  %   \label{fig:arprests}
  % \end{wrapigure}

  Arpeggiator menu description:
  \begin{itemize}
    \item \textbf{GRAY} - inactive steps
    \item \textbf{YELLOW} - Active steps
    \item \textbf{ORANGE} - Currently played note
    \item \textbf{RED} - For velocity set to \textbf{OFF} the gate signal will not be set
  \end{itemize}
  Elements on the screen:
  \begin{itemize}
    \item \textbf{GREEN BAR} Duration of note or pattern depending on setting \textbf{SET:Note Length}
    \item Note part contains a number of semitones that output note for given step will differ from input note
    \item Below semitone parts it shows actual output note for the step
    \item Velocity is the value assigned with note that can be used to modulate any parameter. If this value is set to OFF, the given step will not output GATE for this step.
  \end{itemize}
\FloatBarrier

  \subsection{Control arpeggiator position from external source}
  \paragraph*{}Arpeggiator position in default setting increases position with every step. Use Position CV to set position of arpeggiator according to external CV source.
  \newline$\blacksquare$ By setting external CV source, arpeggiator will output only TRIGGER and not GATE signal on GATE Output. Furthermore CLOCK will be ignored and externally controlled position will determine position of arpeggiator.
  \newline$\blacksquare$ Only positive values of CV input will affect position. Negative values will be clipped to position 0. Use either positive CV source or modify source with Attenution and Level.
  \newline$\bigstar$ Use LFO TRIANGLE input with ATTENUATOR set to 0.5 and LEVEL set to 0.5 to create TRAINGULAR input for arpeggiator to work in PING-PONG mode.
  \newline$\bigstar$ Change LFO to RAMP or SAW (LFO Skew) and arpeggiator will work either in ascending or descending mode respectively.

  \subsection{Adding rests}%
  \paragraph*{} Velocity (modulation) values can have range between 0 to 99 however when the editing knob is tuned fully to the left the velocity value will be marked as \textbf{OFF}. When Velocity is set to \textbf{OFF} the gate will not be set at this step.
  \subsection{Duration}%
  \paragraph*{}%
  Arpeggiator operates in fixed length either for a sequence of notes or fixed length of note. Operation mode of arpeggiator can be selected by \textbf{Setting:Note Length}
  \newpage
  \section*{\modname}
  \tableofoptionsnew{\arpset}{{Settings}}
  \tableofoptionsnew{\arpinp}{{Inputs}}
  \tableofoptionsnew{\arpout}{{Outputs}}
}


\def\chdset{
  % {{Timing Mode}/{Duration of chdeggiator pattern}/{
  %   \makesupopt{{
  %     {{BPM}/{Timing based on note duration calculated from BPM which is controlled by steps}},
  %     {{Hz}/{Frequency at which pattern repeats}}%
  %  }}
  % }},
  % {{Note Length}/{Fixed note or pattern length}/{
  %   Switch between fixed note or pattern duration
  %   \makesupopt{{
  %     {{Off}/{Fixed \textbf{Pattern} length is selected, note duration is pattern duration divided by number of steps}},
  %     {{On}/{Fixed \textbf{Note} length is selected, pattern duration equals note duration times number of steps}}%
  %  }}%
  % }}%
} 

\def\chdinp{
  {{NOTE}/{Base note}/{Input note that is a base note for chord generator to create chord.
  \newline See: \underline{\nameref{section:pitchcontrol}}
  \makesupopt{{
    {{NOTE}/{1V/Oct input for note control/}},
    {{OCT}/{Tuning note by octaves +/- 8 octaves}},
    {{NOTE}/{Tuning note by semitones +12 semitones (one octave)}},
    {{FINE}/{Tuning note by cents +/- one semitone}}%
  }}%
  }},
  {{Trigger}/{Trigger CV Source}/{Source of trigger to advance to next chord on the list and generate resulting notes for taht chord}},
  {{Steps}/{Number of Chords}/{Changes number of chords in chord progression}},
  {{Reset}/{Reset CV Source}/{Reset position to step 0}}%
}

\def\chdout{
  {{NTE1}/{Pitch Output of root note}/{Output of root notes's pitch adjusted by value of NOTE input}},
  {{GTE1}/{Gate Output of root note}/{Output of gate for root note as detected}},
  {{NTE2}/{Pitch Output of 1st chord note}/{Output of given notes step's pitch calculated by adding chord's 1st semitones to played note}},
  {{GTE2}/{Gate Output of 1st chord gate}/{Output of gate for given note in chord if available}},
  {{NTE3}/{Pitch Output of 2nd chord note}/{Output of given notes step's pitch calculated by adding chord's 2nd semitones to played note}},
  {{GTE3}/{Gate Output of 2nd chord gate}/{Output of gate for given note in chord if available}},
  {{NTE4}/{Pitch Output of 3rd chord note}/{Output of given notes step's pitch calculated by adding chord's 3rd semitones to played note}},
  {{GTE4}/{Gate Output of 3rd chord gate}/{Output of gate for given note in chord if available}}%
}

\def\refcardchd{
  \makereferencecard{\textcolor{black}{\textbf{[ROTATE]}} Navigate between steps\newline\textcolor{black}{\textbf{[CLICK]}} Cycle through chords for selected step
    }{
    \textcolor{black}{\textbf{[ON/OFF]}} Turning module On or Off (Bypass),
    \textcolor{black}{\textbf{[SET]}} Settings of the module,
    \textcolor{black}{\textbf{[Inputs]}} Configure Inputs of the module,
    \textcolor{black}{\textbf{[Back]}} Return back to Patch Menu,
  }{
    \textcolor{black}{\textbf{\mbox{[STEP 1]}}} Change chord for a given step,
    \textcolor{black}{\textbf{\mbox{[STEP 2]}}} Change chord for a given step,
    \textcolor{black}{\textbf{\mbox{[STEP 3]}}} Change chord for a given step,
    \textcolor{black}{\textbf{\mbox{[STEP 4]}}} Change chord for a given step,
    \textcolor{black}{\textbf{\mbox{[STEP 5]}}} Change chord for a given step,
    \textcolor{black}{\textbf{\mbox{[STEP 6]}}} Change chord for a given step,
    \textcolor{black}{\textbf{\mbox{[STEP 7]}}} Change chord for a given step,
    \textcolor{black}{\textbf{\mbox{[STEP 8]}}} Change chord for a given step,
    }{%
    \textcolor{black}{\textbf{[STEPS]}} Adjusts number of steps (between 1 and 8),
    NA,
    \textcolor{black}{\textbf{[TIMING]}} Changes timing of pattern between time based (Hz) and note duration based (BPM),
    \textcolor{black}{\textbf{[NOTE-VOCT]}} Select CV input for note's V/OCT,
    \textcolor{black}{\textbf{[NOTE-OCT]}} Tune note by octave (+/- 8 octaves),
    \textcolor{black}{\textbf{[NOTE-NOTE]}} Tune sound by semitone (+12 semitones one octave),
    \textcolor{black}{\textbf{[NOTE-FINE]}} Fine-tune note in cents (+/- semitone),
    NA,
  }
}

\def\modnameshortchd{CHD}
\def\modnamelongchd{Chords Generator}

\newcommand{\modchd}{
  \newpage
  \modmakesection{chd}

  \paragraph*{}
  Chords Generator (CHD) creates chords from given note
  \showrefcardsection{chd}

  \subsection{Operation}
  \paragraph*{} Chords Genertor (CHD) is a pitch manipulation utility that takes root note and outputs pitch of generated notes for up to 4 notes. 
  It takes predefined chords and uses them to add semitones to root note and create melodic chords.
  \newline$\blacksquare$ Every time trigger comes (usually caused by gate) Chords Generator will generate gate signal that will last
   as long as input gate signal for notes available in chords.
  \newline$\blacksquare$ Trigger advances chord progression to next chord. If there is only one chord on the list it will retrigger chord generation.
  \newpage
  \subsection{Chords Generator Screen}
  \begin{figure}[h]
    \centering
    \includegraphics[width=0.35\linewidth]{chd1.png}
    \caption{Chord Generator with multiple chords visibla and notes generated for current step}
    \label{fig:chdrests}
  \end{figure}
  % \FloatBarrier
  % \newpage
  % \begin{wrapfigure}{r}{0.5\textwidth}
  %   \centering
  %   \includegraphics[width=0.7\linewidth]{chd1.png}
  %   \caption{Chdeggiator with rest set on STEP 1}
  %   \label{fig:chdrests}
  % \end{wrapigure}

  Chords menu description:
  \begin{itemize}
    \item \textbf{DARG GREEN} - Generated chord
    \item \textbf{YELLOW} - Shows list of active chords in chord progression
    \item \textbf{ORANGE} - Current chord
  \end{itemize}
  To change number of chords use \textbf{ATT1} (STEPS) to adjust between 1 and 8 chords.
  Use  \textbf{ATT3} and \textbf{ATT4} to change source of Trigger CV and NOTE CV respectively.
\FloatBarrier

  \subsection{Duration}%
  \paragraph*{}%
  Duration of chord is connected with duration of gate send to Trigger input.
  \newpage
  \section*{\modname}
  \tableofoptionsnew{\chdset}{{Settings}}
  \tableofoptionsnew{\chdinp}{{Inputs}}
  \tableofoptionsnew{\chdout}{{Outputs}}
}


\def\midset{
  {{Channel}/{MIDI Channel to receive messages}/{
    \makesupopt{{
      {{ALL}/{Listen on all channels and any note arriving on any channel will trigger gate action}},
      {{1-16}/{Selected MIDI channel}}%
   }}
  }},
  {{MODE}/{Mode of MIDI operation}/{
    Switch between different modes of operation
    \makesupopt{{
      {{Single Note}/{Operation in mono mode with every new note generating outputs on NTE1, GAT1 and VEL1 outpus}},
      {{Polyphonic}/{Switches outputs to 4 outputs and consecutive notes on on 4 channels arriving will generate signals on outpus NTE1-4, GAT1-4 and VEL1-4}},
      {{Drum Rack}/{Drum rack mode when NOTE is being matched with output channel number and on MIDI NOTE ON the gate and velocity are set}}%
   }}%
  }},
  {{Drum Note 1-4}/{Assigned note to output in Drum mode}/{
    Assigns note to channel in Drum Rack mode. The note when matched only will allow Gate (GATx) and Velocity (VELx) of channel to be set.
  }}%
} 

\def\midinp{
  {{Level 1-4}/{Volume Level for channel}/{Volume Level for channel or Amplitude Modulation Level}}%
  {{Pan 1-4}/{Balance for channel}/{Balance (Left-Right panning) for channel 1-4}}%
}

\def\midout{
  {{NTE1-4}/{Pitch Output of received note}/{Note coresponding to MIDI note on given channel}},
  {{GTE1-4}/{Gate Output of received note}/{Output of gate}},
  {{VEL1-4}/{Velocity Output of received note}/{Output of velocity}}%
}


\def\refcardmid{
  \makereferencecard{\textcolor{black}{\textbf{[ROTATE]}} Alternate between MIDI modes (Single, Poly, Drum)\newline\textcolor{black}{\textbf{[CLICK]}} N/A
    }{
      NA,
      NA,
      NA,
      NA
  }{
    NA,
    NA,
    NA,
    NA,
    NA,
    NA,
    NA,
    NA
}{%
    \textcolor{black}{\textbf{\mbox{[NOTE 1]}}} In Drum Rack selects note trigger for output 1,
    \textcolor{black}{\textbf{\mbox{[NOTE 2]}}} In Drum Rack selects note trigger for output 2,
    \textcolor{black}{\textbf{\mbox{[NOTE 3]}}} In Drum Rack selects note trigger for output 3,
    \textcolor{black}{\textbf{\mbox{[NOTE 4]}}} In Drum Rack selects note trigger for output 4,
    \textcolor{black}{\textbf{[MIDI-CH]}} Change MIDI channel,
    NA,
    NA,
    NA
}
}

\def\modnameshortmid{MID}
\def\modnamelongmid{MIDI}

\newcommand{\modmid}{
  \newpage
  \modmakesection{mid}

  \paragraph*{}
  MIDI (MID) Allows processing external MIDI input and converting it into internal signals
  \showrefcardsection{mid}

  \subsection{Operation}
  \paragraph*{} MIDI (MID) is an utility that allows converting external MIDI input into NOTE, GATE and VELOCITY signals. External MIDI input requires extension for The Centre. 
  Those extensions currently are 1V/Oct's Tamar and Taipo modules.
  \newline$\blacksquare$ MIDI Note velocity will be sent on Output VEL
  \newline$\blacksquare$ There can be multiple MIDI modules running at the same time responding on different channels

  \newpage
  \subsection{Operating Modes}
  Use \textcolor{black}{\textbf{[ROTATE]}} to alternate between MIDI modes: Single Note, Polyphony, Drum Rack.
  \newline$\blacksquare$ Intensity of GREEN colour indicates the velocity of the note.

  \subsubsection{Single Note}
  In this mode every new incoming MIDI note replaces current note and generates single output of pitch, gate and velocity.
  \begin{figure}[h]
    \centering
    \includegraphics[width=0.35\linewidth]{mid1.png}
    \caption{MIDI module in Single Note mode}
    \label{fig:midsinglenote}
  \end{figure}

  \subsubsection{Polyphony}
  Polyphonic mode allows 4 MIDI notes to be processed at the time and generate for outputs consisting of note pitch, gate and velocity named NTEx, GATx, VELx with x being respectfully within 1 and 4. 
  The notes won't be replaced and will occupy the given output until the MIDI note is released (send MIDI NOTE OFF message). That means pressing 5th and subsequent notes on MIDI controller will 
  not replace notes unlike Single Note mode.
  \newline$\bigstar$ Create 4 WTO modules each one Note-VOCT input routed from MID.NTEx All 4 

  \begin{figure}[h]
    \centering
    \includegraphics[width=0.35\linewidth]{mid2.png}
    \caption{MIDI module in Polyphony mode}
    \label{fig:midpolyphony}
  \end{figure}

  \subsubsection{Drum Rack}
  Drum Rack allows allocating (fixing) the notes that will respond to MIDI notes being sent. This means 4 channel output consisting of NTEx, GATx and VELx (with x being in range 1 to 4) will be only triggered
  when the requested note is being sent. For example setting note C2 (General MIDI - Bass Drum 1 as on below picture) in output number 1, will only trigger gate when that note is being sent by MIDI.
  Please be aware that BeatStep Pro from Arturia sends different notes in Drum mode and D is sent as C-sharp, E as D, etc.
  Each output can be then routed to different modules, for example 4 Sample (SMP) modules playing different samples of drums.
  \newline$\blacksquare$ In Drum Rack mode attention need to be paid to select a MIDI channel that Drum Rack operates on to not pickup notes from other channels.

  \begin{figure}[h]
    \centering
    \includegraphics[width=0.35\linewidth]{mid3.png}
    \caption{MIDI module in Drum Rack mode}
    \label{fig:middrumrack}
  \end{figure}

  % \FloatBarrier
  % \newpage
  % \begin{wrapfigure}{r}{0.5\textwidth}
  %   \centering
  %   \includegraphics[width=0.7\linewidth]{mid1.png}
  %   \caption{Arpeggiator with rest set on STEP 1}
  %   \label{fig:midrests}
  % \end{wrapigure}

  \FloatBarrier
  $\blacksquare$ In Drum Rack menu use ATT1-4 knobs to change notes for each drum channel respectively. Use ATT5 to change MIDI channel.

  \newpage
  \section*{\modname}
  \tableofoptionsnew{\midset}{{Settings}}
  \tableofoptionsnew{\midinp}{{Inputs}}
  \tableofoptionsnew{\midout}{{Outputs}}
}


\def\mixset{
  {{ChanAudio Outputnel}/{Audio Output of Mixer}/{Audio Output of Mixed 4 channels from Input Sources}},
  {{Audio Input 1-4}/{Audio Input for Channel 1 to 4}/{Source of Audio Input for corresponding channel number. \newline$\bigstar$Please note that channel must be enabled or input source has no effect}},
  {{Enable 1-4}/{Input Enable for Channel 1 to 4}/{Enables corresponding channel for mixing}},
  {{Downmix}/{Downmix of polyphony}/{Enable or disable downmix of polyphony channels into dual stereo output. \newline$\bigstar$Please note that downmix only works in VOUT mode by outputting left channel in VOUT1 and right in VOUT2. Downmix is disabled in monophonic mode or when output is directed to VBuf}},
} 

\def\mixinp{
  {{LVL1-4}/{Level for channels 1-4}/{CV input for channel 1-4 volume level (AM)}},
  {{PAN1-4}/{Balance for channels 1-4}/{CV input for balance of channels 1-4 }},
}

\def\mixout{
  {{MIX}/{Mixed output level}/{Level of mixed channels}},
}


\def\refcardmix{
  \makereferencecard{\textcolor{black}{\textbf{[ROTATE]}} Select channel\newline\textcolor{black}{\textbf{[CLICK]}} Enter/exit channel\newline\textcolor{black}{\textbf{[ROTATE]}} Select Source
    }{
      \textcolor{black}{\textbf{[ON/OFF]}} Turning channel On or Off,
      \textcolor{black}{\textbf{[SET]}} Settings of the module,
      \textcolor{black}{\textbf{[Inputs]}} Configure Inputs of the module,
      \textcolor{black}{\textbf{[Back]}} Return back to Patch Menu,
    }{
    \textcolor{black}{\textbf{\mbox{[LVL 1]}}} Adjusts volume level for channel 1,
    \textcolor{black}{\textbf{\mbox{[PAN 1]}}} Adjusts balance for channel 1,
    \textcolor{black}{\textbf{\mbox{[LVL 2]}}} Adjusts volume level for channel 2,
    \textcolor{black}{\textbf{\mbox{[PAN 2]}}} Adjusts balance for channel 2,
    \textcolor{black}{\textbf{\mbox{[LVL 3]}}} Adjusts volume level for channel 3,
    \textcolor{black}{\textbf{\mbox{[PAN 3]}}} Adjusts balance for channel 3,
    \textcolor{black}{\textbf{\mbox{[LVL 4]}}} Adjusts volume level for channel 4,
    \textcolor{black}{\textbf{\mbox{[PAN 4]}}} Adjusts balance for channel 4
}{%
    \textcolor{black}{\textbf{\mbox{[AUDIO INPUT 1]}}} Select input source for channel 1,
    \textcolor{black}{\textbf{\mbox{[AUDIO INPUT 2]}}} Select input source for channel 2,
    \textcolor{black}{\textbf{\mbox{[AUDIO INPUT 3]}}} Select input source for channel 3,
    \textcolor{black}{\textbf{\mbox{[AUDIO INPUT 4]}}} Select input source for channel 4,
    \textcolor{black}{\textbf{[DOWNMIX]}} Enable Downmix,
    NA,
    NA
    \textcolor{black}{\textbf{[OUT]}} Select Audio Output Destination,
}
}

\def\modnameshortmix{MIX}
\def\modnamelongmix{Mixer}

\newcommand{\modmix}{
  \newpage
  \modmakesection{mix}

  \paragraph*{}
  Mixer (MIX) is a stereo 4 channel mixer that allows mixing (prefereably virtual buffers - VBuf) of 4 input modules into single output 
  while applying levels and panning. Levels and panning can be controlled by CV. 
  In fact Mixer module is a quad stereo VCA.
  \showrefcardsection{mix}

  \subsection{Operation}
  \paragraph*{} Mixer (MIX) starts with 4 panels (channels) in off mode.
  \textbf{SELECT} encodser allows navigating over those 4 channels. 
  By pressing \textbf{SELECT} down we are entering control of channel. Now encoder allows to select a source for that channel.
  Alternative method of controlling input source of channel is to select channel with encoder, enable channel by pressing button \textbf{On/Off} and then use coresponding attenuator knob ATT1-4 to select input source for different channel.

  Level and stereo balance (left-right panning) can be achieved by using coresponding LVL knobs.

  \begin{figure}[h]
    \centering
    \includegraphics[width=0.35\linewidth]{mix1.png}
    \caption{MIX with all channels disabled}
    \label{fig:mixpolyphony}
  \end{figure}

  \FloatBarrier
  \paragraph*{}$\blacksquare$ Mixer starts with all channels off. Rotate encode \textbf{SELECT} and press \textbf{On/Off} button to enable channels.
  \paragraph*{}$\blacksquare$ Use Attenuator ATT8 to select output of mixed channels.

  \begin{figure}[h]
    \centering
    \includegraphics[width=0.35\linewidth]{mix2.png}
    \caption{MIX with all channels enabled}
    \label{fig:mixpolyphony}
  \end{figure}

  \FloatBarrier
  $\blacksquare$ There are hints on every channel to use Level and Attenuator knobs to control channel. For example for channel 1: L1 controls Level, L2 pan and A1 allows selecting input source.

  \subsection{Polyphony and Downmix}
  \paragraph*{} Mixer (MIX) in Polyphony mode will mix 4 inputs each consisting of 4 channels of polyphony and output 4 channels of polyphony or \textbf{downmix} of polyphony to stereo output available in VOUT1 and VOUT2 outputs. 
  Downmix can be enabled either in Settings or by using attenuator A5 on Mixer module screen.

  \paragraph*{}$\blacksquare$ Downmix is disabled when output is directed into VBuf or when module is in monophonic mode.
  \paragraph*{}$\blacksquare$ Downmix is disabled by default and needs to be enabled to allow downmixing otherwise output is quad polyphonic individual channels.
  

  \newpage
  \section*{\modname}
  \tableofoptionsnew{\mixset}{{Settings}}
  \tableofoptionsnew{\mixinp}{{Inputs}}
  \tableofoptionsnew{\mixout}{{Outputs}}
}


\def\outset{
  % {{Audio Output}/{Audio Output of Drum Rack}/{Selects audio output destination for OUT. All 4 sample slots are sharing the same audio output where they get downmixed \newline See: \underline{\nameref{section:audiooutputs}}}},
}

\def\outinp{
  {{Release}/{Release for Limiter}/{リミッターのリリースレート}}%
}

\def\outout{
  % {{Gate 1}/{Gate 1 Output}/{Output of Channel 1 setting gate to high upon beat}},
  % {{Gate 2}/{Gate 2 Output}/{Output of Channel 2 setting gate to high upon beat}},
  % {{Gate 3}/{Gate 3 Output}/{Output of Channel 3 setting gate to high upon beat}},
  % {{Gate 4}/{Gate 4 Output}/{Output of Channel 4 setting gate to high upon beat}}%
}

\def\refcardout{
  \makereferencecard{
    % \textcolor{black}{\textbf{[ROTATE]}} Select sample slot
    % \newline\textcolor{black}{\textbf{[CLICK]}} Enter file browser to select sample from SD card
    NA
  }{
    % \textcolor{black}{\textbf{[On/Off]}} Controls module on and off,
    % \textcolor{black}{\textbf{[SET]}} Settings of the module,
    % \textcolor{black}{\textbf{[Inputs]}} Configure Inputs of the module,
    % \textcolor{black}{\textbf{[Back]}} Return back to Patch Menu,
    NA,
    NA,
    NA,
    NA,
  }{
    \textcolor{black}{\textbf{[RELEASE]}} リミッター解除速度,
    NA,
    NA,
    NA,
    NA,
    NA,
    NA,
    NA,
  }{%
    NA,
    NA,
    NA,
    NA,
    NA,
    NA,
    NA,
    NA,
  }
}



\def\modnameshortout{OUT}
\def\modnamelongout{Output}

\newcommand{\modout}{
  \newpage
  \modmakesection{out}

  \paragraph*{}
  Output (OUT) は、オーバーステアされたオーディオ レベルのリミッターおよびクリッパーです。

  \showrefcardsection{out}

  \subsection{操作}
  \paragraph*{} 出力モジュールはリミッターとして機能し、オーディオ信号が再生可能な範囲を超えると、即座に反応してオーディオ信号をダッキング (制限) し (クイック アタック モード)、\textbf{RELEASE} 入力設定に従ってダッキングを解除します。リリースの高い値 (短い時間) は、オーディオ レベルをすばやく元に戻すため、リミッターは単純なコンプレッサーのように動作します。リリース レートが長い (値が小さい) と、制限を超えるピークのオーディオ レベルに対して、リミッターがわずかに不安定なサウンドを生成します。
  % \newline$\blacksquare$ クロックソースがない場合、BPM は 120 BPM に固定されます。
  
  \newpage
  \section*{\modname}
  \tableofoptionsnew{\outset}{\optnameset}
  \tableofoptionsnew{\outinp}{\optnameinp}
  \tableofoptionsnew{\outout}{\optnameout}
}


