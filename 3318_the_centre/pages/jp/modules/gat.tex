\def\gatset{
  {{NONE}/{}/{}}%
}

\def\gatinp{
  {{Clock}/{クロック CV ソース}/{Pulse CV source to base divisions on}},
  {{Reset}/{CV ソースをリセット}/{すべての位置を開始に設定することにより、すべてのディバイダの位置を同期するリセット信号}},
  {{Intverval 1}/{パルスインターバル}/{高ゲート時のゲートインターバル}},
  {{Gate Len 1}/{パルスの長さ}/{チャンネル内のビート数}},
  {{Delay 1}/{パルスのトリガー前のディレイ}/{ディレイは、パルス信号がゲートをHighにするまでの期間}},
  {{Intverval 2}/{パルスインターバル}/{高ゲート時のゲートインターバル}},
  {{Gate Len 2}/{パルスの長さ}/{チャンネル内のビート数.}},
  {{Delay 2}/{パルスのトリガー前のディレイ}/{ディレイは、パルス信号がゲートをHighにするまでの期間}},
  {{Intverval 3}/{パルスインターバル}/{高ゲート時のゲートインターバル}},
  {{Gate Len 3}/{パルスの長さ}/{チャンネル内のビート数}},
  {{Delay 3}/{パルスのトリガー前のディレイ}/{ディレイは、パルス信号がゲートをHighにするまでの期間}},
  {{Intverval 4}/{パルスインターバル}/{高ゲート時のゲートインターバル}},
  {{Gate Len 4}/{パルスの長さ}/{チャンネル内のビート数}},
  {{Delay 4}/{パルスのトリガー前のディレイ}/{ディレイは、パルス信号がゲートをHighにするまでの期間}}%
}

\def\gatout{
  {{Gate 1}/{Gate 1出力}/{チャンネル 1 パルス信号の出力}},
  {{Gate 2}/{Gate 2 出力}/{チャンネル 2 パルス信号の出力}},
  {{Gate 3}/{Gate 3 出力}/{チャンネル 3 パルス信号の出力}},
  {{Gate 4}/{Gate 4 出力}/{チャンネル 4 パルス信号の出力}}%
}

\def\refcardgat{
  \makereferencecard{
    \textcolor{black}{\textbf{[回す]}}Gate Dividerチャンネル 1 ~ 4 を選択,
    }{
    \textcolor{black}{\textbf{[ON/OFF]}} モジュールのオン/オフ (バイパス),
    \textcolor{black}{\textbf{[SET]}} モジュールの設定,
    \textcolor{black}{\textbf{[Inputs]}} モジュールの入力を構成する,
    \textcolor{black}{\textbf{[Back]}} パッチメニューに戻る,
  }{
    \textcolor{black}{\textbf{[INTV1]}} チャンネル 1 のインターバル,
    \textcolor{black}{\textbf{[GLEN1]}} チャンネル 1 のゲートの長さ,
    \textcolor{black}{\textbf{[INTV2]}} チャンネル 2 のインターバル,
    \textcolor{black}{\textbf{[GLEN2]}} チャンネル 2 のゲートの長さ,
    \textcolor{black}{\textbf{[INTV3]}} チャンネル 3 のインターバル,
    \textcolor{black}{\textbf{[GLEN3]}} チャンネル 3 のゲートの長さ,
    \textcolor{black}{\textbf{[INTV4]}} チャンネル4 のインターバル,
    \textcolor{black}{\textbf{[GLEN4]}} チャンネル 4 のゲートの長さ,
  }{%
    \textcolor{black}{\textbf{[DLY1]}} Channel 1のDelay,
    \textcolor{black}{\textbf{[DLY2]}} Channel 2のDelay,
    \textcolor{black}{\textbf{[DLY3]}} Channel 3のDelay,
    \textcolor{black}{\textbf{[DLY4]}} Channel 4のDelay,
    \textcolor{black}{\textbf{[CLK]}}CLOCKソースCVを選択,
    \textcolor{black}{\textbf{[RST]}} RESETソースCVを選択,
    NA,
    NA,
  }
}

\def\modnameshortgat{GAT}
\def\modnamelonggat{Gate Divider}

\newcommand{\modgat}{
  \newpage
  \modmakesection{gat}

  \paragraph*{}
 Gate Divider (GAT) は、パルス クロック信号 (クロック) をより長い周期のパルスクロック信号に分割します。

  \refcardgat%

  \subsection{操作}
  \paragraph*{} Gate Divider (GAT) は、入力されたパルス信号 (通常はクロック) を分割してパルス信号 (位相幅の異なるSquare波) を生成します。入力信号は 4 チャネルのカウンターをトリガーし、設定されたパラメーターに基づいて、位相の幅が変更された長いパルス信号 (負と正の両方) を作成します。
  \newline$\blacksquare$ このような分割は、クロック信号を等間隔の休符を持つ等間隔の音符、または特定の音符の長さを持つ等間隔の音符トリガーに変換することと見なすことができます。
  \newline$\bigstar$ 分割されたゲートは、繰り返しのタイム ジェネレーター (ドラム マシン) として使用される繰り返しパターンの優れたソースです。gate dividerからドラム サンプルをトリガーし、gate divider divisionsを異なるノートに設定すると、通常のビートパターンを作成できます。

  \newpage
  \section*{\modname}
  \tableofoptionsnew{\gatset}{\optnameset}
  \tableofoptionsnew{\gatinp}{\optnameinp}
  \tableofoptionsnew{\gatout}{\optnameout}
}

