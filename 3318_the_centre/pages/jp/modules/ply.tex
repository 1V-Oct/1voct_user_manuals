\def\plyset{
  {{Pattern Length}/{barでのパターンの長さ}/{バーでのパターンの長さ。各パターンは Inputs:Pulses で設定されたステップ数に分割されます}},
  {{Gate length}/{gateの長さ}/{
    \makesupopt{{
      {{Trigger}/{Triggerのみ}/{パルス発生時の非持続信号}}%
    }}%
  }}%
}

\def\plyinp{
  {{Clock}/{Clock CV Source}/{Poly Rhythmを他のモジュールと同期させるためのクロックソース}},
  {{Reset}/{Reset CV Source}/{トリガー時にパターン内のすべてのチャンネルの位置をリセット}},
  {{Pulses 1}/{チャンネル 1 のパルス数}/{パターンが分割され、各ステップの開始時にビートを生成するステップの数}},
  {{Pulses 2}/{チャンネル 2 のパルス数}/{パターンが分割され、各ステップの開始時にビートを生成するステップの数}},
  {{Pulses 3}/{チャンネル 3 のパルス数}/{パターンが分割され、各ステップの開始時にビートを生成するステップの数}},
  {{Pulses 4}/{チャンネル 4 のパルス数}/{パターンが分割され、各ステップの開始時にビートを生成するステップの数}}%
}

\def\plyout{
  {{Gate 1}/{Gate 1 出力}/{ビートに対してPoly RhythmゲートをHighにする出力}},
  {{Gate 2}/{Gate 2 出力}/{ビートに対してPoly RhythmゲートをHighにする出力}},
  {{Gate 3}/{Gate 3 出力}/{ビートに対してPoly RhythmゲートをHighにする出力}},
  {{Gate 4}/{Gate 4 出力}/{ビートに対してPoly RhythmゲートをHighにする出力}}%
}

\def\refcardply{
  \makereferencecard{
    NA%\textcolor{black}{\textbf{[ROTATE]}} Turn note on or off and in Scale Selector turn to Scale Select mode
  }{
    \newline\textcolor{black}{\textbf{[RESET]}} ボタンを放すと、パターン内のすべてのチャンネルがリセットされます,
    \newline\textcolor{black}{\textbf{[SET]}} モジュールの設,
    \newline\textcolor{black}{\textbf{[Inputs]}} モジュールの入力を構成する,
    \newline\textcolor{black}{\textbf{[Back]}} パッチメニューに戻る,
  }{
    チャネル 1 のステップ数を選択,
    チャネル 2 のステップ数を選択,
    チャネル 3 のステップ数を選択,
    チャネル 4のステップ数を選択,
    NA,
    NA,
    NA,
    NA,
  }{%
    \textcolor{black}{\textbf{[PLEN]}}パターンの長さを選択,
    NA,
    NA,
    NA,
    \textcolor{black}{\textbf{[CLK]}} CLOCKソースCVを選択,
    NA,
    NA,
    NA,
  }
}


\def\modnameshortply{PLY}
\def\modnamelongply{Polyrhythm}

\newcommand{\modply}{
  \newpage
  \modmakesection{ply}  
  \def\modname{PLY - Polyrhythm}% chktex 8
  

  \paragraph*{}
  Polyrhythm (PLY) は、パターン内で対照なリズムを生成します。

  \showrefcardsection{ply}

  \subsection{操作}
  \paragraph*{} Polyrhythmは、パターンの長さを入力パラメータ Inputs:Steps で決定される等しいステップ数に分割することにより、4 チャンネルのリズムを生成します。パターンは、指定された BPMでの小節数 (4 拍または 4 分音符) をカバーする期間として定義されます。チャンネルのステップ数は、Inputs:Beats X (X はチャンネル 1 ~ 4 の番号) で制御できます。
  \newline$\blacksquare$ クロックソースがない場合、BPM は 120 BPM に固定されます。
  
  \newpage
  \section*{\modname}
  \tableofoptionsnew{\plyset}{\optnameset}
  \tableofoptionsnew{\plyinp}{\optnameinp}
  \tableofoptionsnew{\plyout}{\optnameout}
}

