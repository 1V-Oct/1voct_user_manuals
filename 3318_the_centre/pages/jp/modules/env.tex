\def\envset{
  {{Audio Output}/{ENV出力}/{ENV で処理された信号のオーディオ出力。信号は、LevelとSidechainの 2 つのモジュレーターの合計によって変調された振幅を持っています。\newline 参照: \underline{\nameref{section:audiooutputs}}}},
  {{Audio Input}/{モジュレーションされるオーディオ信号}/{モジュレーターで変調された振幅 (レベル) を持つオーディオ信号 (任意の信号) 入力: Level}},
  {{Mode}/{Envelope Mode}/{AHDSR (サステインあり) エンベロープと AHDR (サステインなし) エンベロープのモード切り替え
    \makesupopt{{
      {{Gate}/{エンベロープは、ゲート信号が開いている (高い) 限り持続し、Sustain levelで音量を維持します。}},
      {{Trigger}/{エンベロープはステージAHDRの合計の間だけ持続し、音を維持することはありません。}}%
    }}
  }}%
}

\def\envinp{
  {{Attack}/{Attack Stage}/{Attackはエンベロープを初段に上げます。 0 秒から 32 秒まで。}},
  {{Hold}/{Hold Stage}/{Holdは、エンベロープの第2段を維持するフルボリュームレベルです。 0 秒から 32 秒まで。}},
  {{Decay}/{Decay Stage}/{Decayは、エンベロープがフル レベルのボリュームからレベルセットまで下降する第3段階です。 0 秒から 32 秒まで。}},
  {{Sustain}/{Sustain Stage}/{サステインは、ゲートが開いている間(高)、AHDステージの後の音レベルです。デシベル単位の音のレベル}},
  {{Release}/{Release Stage}/{リリースは、サウンドがレベル 0 に減衰する最終段階です。持続時間は 0 秒から 32 秒です。}},
  {{Gate}/{Gate CV Source}/{エンベロープ生成を開始および維持するためのゲートまたはトリガー信号}},
  {{Attack Curve}/{Exponent of Attack stage}/{Attack Curveは、段階を線形ではなく指数に変更するためのアタックの指数です}},
  {{Decay Curve}/{Exponent of Decay stage}/{Decay Curveは減衰段階の指数であり、段階を線形ではなく指数に変更します}},
  {{Release Curve}/{Exponent of Release stage}/{Release Curveは、ステージを線形ではなく指数関数に変更するためのリリース段階の指数です。
}}%
}

\def\envout{
  {{Envelope}/{Output of Envelope}/{コントロール電圧として他のモジュールにルーティングされるエンベロープ信号の出力}}%
}

\def\refcardenv{
  \makereferencecard{
    \textcolor{black}{\textbf{[回す]}} 波形の切り替え(Triangle Square Sine),
  }{
    \textcolor{black}{\textbf{[ON/OFF]}} モジュールのオン/オフ (Bypass),
    \textcolor{black}{\textbf{[SET]}} モジュールの設定,
    \textcolor{black}{\textbf{[Inputs]}} モジュールの入力を構成する,
    \textcolor{black}{\textbf{[Back]}} パッチメニューに戻る,
  }{
   0.0~31.5秒の範囲でATTACKを調整,
   HOLD を 0 ~ 31.5 秒の範囲で調整\newline$\blacksquare$ HOLDが0のときエンベロープはADSR、 DECAY時間を0.0s~31.5sの範囲で調整,
    SUSTAIN levelの調整,
    RELEASE時間を0.0s~31.5sの範囲で調整,
   NA,
   NA,
   NA,
  }{
   ATTACKのカーブ形状を調整,
   NA,
   DECAYのカーブ形状を調整,
    RELEASEのカーブ形状を調整,
   \textcolor{black}{\textbf{[GAT]}} GATE 入力 CV の選択,
   \textcolor{black}{\textbf{[MODE]}} トリガー (AHDR) モードとゲート (AHDSR) モードの切り替え,
   NA,
   エンベロープのAUDIO出力を選択します(エンベロープを送信して他のデバイスを制御できます),
  }
}

\def\modnameshortenv{ENV}
\def\modnamelongenv{Envelope Generator}

\newcommand{\modenv}{
  \newpage
  \modmakesection{env}

  \paragraph*{}エンベロープ ジェネレーターは、指定されたレートで上昇および減衰するモジュレーション信号を生成します。
  
  \showrefcardsection{env}

  \subsection{Operation}
  エンベロープ ジェネレーターは、シンセサイザーのモジュールのパラメーターを変調するコントロール ボルテージ信号である AHDSR (Attack, Hold, Decay, Sustain, Release) または AHDR エンベロープを生成します (通常は、ENV の振幅と VCF のカットオフ周波数に限定されません)。
  \subsection{Envelope Stages}
  \makesupopt{{
    {{A: Attack}/{Attack stageは最初に一定時間信号の音量を 0 から 100\% まで上げる段階です。}},
    {{H: Hold}/{Hold stageは、信号を一定時間最大音量レベルに保ちます}},
    {{D: Decay}/{Decay stage は、音量が \textbf{Hold} レベルから \textbf{Sustain} レベルに低下する時間の長さを決定します。 Decay stageの長さは期間として提示されます。}},
    {{S: Sustain}/{Sustain stageは、演奏されたノートの長さに関連するゲート信号が高い信号を維持する前に、サウンドが再生し続ける音量レベルを決定します。エンベロープの他のパラメーターとは異なり、Sustain stageは時間の長さではなく、音量のレベルとして提示されます。}},
    {{R: Release}/{Releaseは、ゆっくりと減衰するか、急速に終了するサウンドの最終段階です。 ステージは期間として提示されます。}}%
  }}%
  \section*{\modname}

$\bigstar$\textbf{Gate} 信号の長さ (演奏されたノートの長さ) が \textbf{Atack} + \textbf{Hold} + \textbf{Decay} の合計の長さよりも短い場合、これらの段階のいくつかは Sustain までスキップされます。 \textbf{Gate} 信号が Low になるとサステイン段階が開始されます。したがって、\textbf{Gate} 信号の代わりに \textbf{Trigger} 信号が適用された場合、すべてのステージ (A+H+D) がスキップされ、エンベロープはリリース ステージのみに制限されます。

  \subsection{Envelope Types}
  最も一般的なエンベロープは ADSR (Attack, Decay, Sustain, Release) です。これは、演奏されたノートの長さに応じて可変長のゲート信号用に設計されたエンベロープです。 ゲート信号が上がると、エンベロープはアタックとディケイ段階を実行し、ゲート信号が低くなるまで振幅をサステインレベルに保ち(ノートの長さが終了する)、リリース段階で振幅が減衰し続けるのを助けます。
   対照的に、AD エンベロープはトリガー信号用に設計されており、ノートの長さに関係なくアタックとディケイの両方の段階を実行します。 ノートの長さは、AttackとDecayのステージの長さによって決まります。
   \paragraph*{} \textbf{AHDSR} エンベロープの \textbf{Hold} ステージは、典型的な ADSR エンベロープの拡張に過ぎず、最大上昇ステージ (\textbf{Attack} の後) を一定期間維持できます。 Holdステージの期間の長さを 0 秒に設定すると、AHDSR エンベロープは標準のADSRエンベロープになります。
  
  \subsection{Trigger Mode}
  エンベロープは、\textbf{Gate} と \textbf{Trigger} の間でSetting(Mode)を切り替えることによってアクティブ化できます。
  \makesupopt{{
    {{Gate}/{このモードでは、エンベロープは提供されたゲート信号の長さに基づいています。ゲートがハイ モードの場合、エンベロープは AHD ステージを通過し、ゲートが閉じて (信号がローになる) リリース ステージが処理されるまでサステイン ステージにとどまります。}},
    {{Trigger}/{このモードでは、エンベロープは常に完全に処理されますが、サステイン要素はありません。 エンベロープは高いトリガー (ゲート) 信号によって引き起こされ、その信号の長さに関係なく、AHDR ステージが実行され、サステイン部分が省略されます。
    \newline$\bigstar$ このモードは、ステージ A と D を目的の値に設定し、他のすべてのステージを 0 に設定することで、古い AD エンベロープをエミュレートするのに理想的です。}}%
  }}%
  % \newpage
  \section*{\modname}
  \tableofoptionsnew{\envset}{\optnameset}
  \tableofoptionsnew{\envinp}{\optnameinp}
  \tableofoptionsnew{\envout}{\optnameout}
}

