\def\drcset{
  {{Audio Output}/{Drum Rackのオーディオ出力}/{DRCの音声出力先を選択します。 4 つのサンプルスロットはすべて、ダウンミックスされる同じオーディオ出力を共有しています。 \newline 参照: \underline{\nameref{section:audiooutputs}}}},
}

\def\drcinp{
  {{Gate 1}/{スロット 1 のゲート CV 入力}/{サンプルの再生を開始するゲートまたはトリガー信号}},
  {{Pitch 1}/{スロット 1 のピッチ調整}/{サンプル再生の相対ピッチ調整}},
  {{Level 1}/{スロット 1 の振幅 (レベル)}/{サンプル再生用のレベル (Amplitude Modulation)}},
  {{Steps 2}/{チャネル 2 のステップ数}/{チャネル内のステップ数}},
  {{Beats 2}/{チャンネル 2 の拍数}/{チャンネル内のビート数。}},
  {{Offset 2}/{チャネル 2 の最初のステップのオフセット}/{チャネルの最初のステップのオフセット}},
  {{Steps 3}/{チャンネル 3 のステップ数}/{チャネル内のステップ数。}},
  {{Beats 3}/{チャンネル 3 のビート数}/{チャンネル内のビート数。}},
  {{Offset 3}/{チャンネル 3 の最初のステップのオフセット}/{チャネルの最初のステップのオフセット}},
  {{Steps 4}/{チャネル 4 のステップ数}/{チャネル内のステップ数。}},
  {{Beats 4}/{チャンネル 4 のビート数}/{チャンネル内のビート数。}},
  {{Offset 4}/{チャンネル4 の最初のステップのオフショット}/{チャネルの最初のステップのオフセット}},
}

\def\drcout{
  % {{Gate 1}/{Gate 1 Output}/{Output of Channel 1 setting gate to high upon beat}},
  % {{Gate 2}/{Gate 2 Output}/{Output of Channel 2 setting gate to high upon beat}},
  % {{Gate 3}/{Gate 3 Output}/{Output of Channel 3 setting gate to high upon beat}},
  % {{Gate 4}/{Gate 4 Output}/{Output of Channel 4 setting gate to high upon beat}}%
}

\def\refcarddrc{
  \makereferencecard{
    \textcolor{black}{\textbf{回す]}} サンプルスロットを選択
    \newline\textcolor{black}{\textbf{[クリック]}} ファイルブラウザに入り、SD カードからサンプルを選択します
  }{
    \textcolor{black}{\textbf{[On/Off]}} モジュールのオンとオフを制御します,
    \textcolor{black}{\textbf{[SET]}} モジュールの設定,
    \textcolor{black}{\textbf{[Inputs]}} モジュールの入力を構成する,
    \textcolor{black}{\textbf{[Back]}} パッチメニューに戻る,
  }{
    \textcolor{black}{\textbf{[PITCH]}} スロット 1 のピッチ コントロール,
    \textcolor{black}{\textbf{[PITCH]}} スロット 2 のピッチ コントロール,
    \textcolor{black}{\textbf{[PITCH]}} スロット 3 のピッチ コントロール,
    \textcolor{black}{\textbf{[PITCH]}} スロット 4 のピッチ コントロール,
    \textcolor{black}{\textbf{[LEVEL]}} スロット 1 のレベル (振幅) コントロール,
    \textcolor{black}{\textbf{[LEVEL]}} スロット 2 のレベル (振幅) コントロール,
    \textcolor{black}{\textbf{[LEVEL]}} スロット 3 のレベル (振幅) コントロール,
    \textcolor{black}{\textbf{[LEVEL]}} スロット 4 のレベル (振幅) コントロール,
  }{%
    \textcolor{black}{\textbf{[GATE]}} スロット 1 のゲート CV を選択します,
    \textcolor{black}{\textbf{[GATE]}} スロット 2 のゲート CV を選択します,
    \textcolor{black}{\textbf{[GATE]}} スロット 3 のゲート CV を選択します,
    \textcolor{black}{\textbf{[GATE]}} スロット 4 のゲート CV を選択します,
    \textcolor{black}{\textbf{[OUT]}} 音声出力先の選択,
    NA,
    NA,
  }
}



\def\modnameshortdrc{DRC}
\def\modnamelongdrc{Drum Rack}

\newcommand{\moddrc}{
  \newpage
  \modmakesection{drc}

  \paragraph*{}
  Drum Rack (DRC) は、ピッチ調整とボリューム レベル コントロール (振幅変調) を備えた 4 スロットのシンプルなサンプル プレーヤーです。

  \showrefcardsection{drc}

  \subsection{操作}
  \paragraph*{} ドラムラックは、サンプルをロードするための 4 つのスロットで構成されています。 \textbf{GATE} 入力でトリガーまたはゲート信号を受信すると、サンプルはシングル ショットとして開始されます。 現在調整するパラメータは 2 つだけで、両方とも静的な値を設定するか、CV を使用して変調することで制御できます。 これらのパラメーターは、ボリューム レベル (振幅) とピッチ補正です。 ピッチ補正は、サンプルのピッチを調整します
  % \newline$\blacksquare$ クロックソースがない場合、BPM は 120 BPM に固定されます。
  
  \newpage
  \section*{\modname}
  \tableofoptionsnew{\drcset}{\optnameset}
  \tableofoptionsnew{\drcinp}{\optnameinp}
  \tableofoptionsnew{\drcout}{\optnameout}
}

