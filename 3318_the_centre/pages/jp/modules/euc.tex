\def\eucset{
  {{Step Length}/{1 ステップの長さ}/{1ステップの長さ、パターンは任意の数の等しい長さで構成されています}}%
  % {{Gate length}/{Length of gate}/{
  %   \makesupopt{{
  %     {{Trigger}/{Trigger only}/{Unsustained signal when the pulse occurs}}%
  %   }}%
  % }}%
}

\def\eucinp{
  {{Clock}/{クロック CV ソース}/{Poly Rhythmを他のモジュールと同期させるためのクロックソース}},
  {{Reset}/{CV ソースをリセット}/{トリガー時にパターン内のすべてのチャンネルの位置をリセットする}},
  {{Steps 1}/{チャンネル 1 のステップ数}/{チャンネル内のステップ数}},
  {{Beats 1}/{チャンネル 1 のビート数}/{チャンネル内のビート数}},
  {{Offset 1}/{チャンネル1 の最初のステップのオフセット}/{チャンネルの最初のステップのオフセット}},
  {{Steps 2}/{チャンネル 2 のステップ数}/{チャンネル内のステップ数}},
  {{Beats 2}/{チャンネル 2 のビート数}/{チャンネル内のビート数}},
  {{Offset 2}/{チャンネル2 の最初のステップのオフセット}/{チャンネルの最初のステップのオフセット}},
  {{Steps 3}/{チャンネル 3 のステップ数}/{チャンネル内のステップ数}},
  {{Beats 3}/{チャンネル 3 のビート数3}/{チャンネル内のビート数}},
  {{Offset 3}/{チャンネル3 の最初のステップのオフセット3}/{チャンネルの最初のステップのオフセット}},
  {{Steps 4}/{チャンネル 4のステップ数}/{チャンネル内のステップ数}},
  {{Beats 4}/{チャンネル 4 のビート数}/{チャンネル内のビート数}},
  {{Offset 4}/{チャンネル4 の最初のステップのオフセット}/{チャンネルの最初のステップのオフセット}},
}

\def\eucout{
  {{Gate 1}/{Gate 1出力}/{ビートに対してゲートをHighにするチャネル1出力}},
  {{Gate 2}/{Gate 2 出力}/{ビートに対してゲートをHighにするチャネル2出力}},
  {{Gate 3}/{Gate 3 出力}/{ビートに対してゲートをHighにするチャネル3出力}},
  {{Gate 4}/{Gate 4 出力}/{ビートに対してゲートをHighにするチャネル4出力}}%
}

\def\refcardeuc{
  \makereferencecard{
    NA%\textcolor{black}{\textbf{[ROTATE]}} Turn note on or off and in Scale Selector turn to Scale Select mode
  }{
    \textcolor{black}{\textbf{[RESET]}} パターン内のすべてのチャンネルが開始位置にリセット,
    \textcolor{black}{\textbf{[SET]}} モジュールの設定,
    \textcolor{black}{\textbf{[Inputs]}} モジュールの入力を構成する,
    \textcolor{black}{\textbf{[Back]}} パッチメニューに戻る,
  }{
    \textcolor{black}{\textbf{[STEP]}} チャンネル 1 のステップ数,
    \textcolor{black}{\textbf{[BEAT]}} チャンネル 1 のビート数,
    \textcolor{black}{\textbf{[STEP]}} チャンネル 2 のステップ数2,
    \textcolor{black}{\textbf{[BEAT]}} チャンネル 2 のビート数,
    \textcolor{black}{\textbf{[STEP]}} Nチャンネル 3 のステップ数,
    \textcolor{black}{\textbf{[BEAT]}} チャンネル 3 のビート数,
    \textcolor{black}{\textbf{[STEP]}} チャンネル 4 のステップ数,
    \textcolor{black}{\textbf{[BEAT]}}チャンネル 4 のビート数,
  }{%
    \textcolor{black}{\textbf{[OFFSET1]}} チャンネル1のオフセット開始,
    \textcolor{black}{\textbf{[OFFSET2]}} チャンネル2のオフセット開始,
    \textcolor{black}{\textbf{[OFFSET3]}} チャンネル3のオフセット開始,
    \textcolor{black}{\textbf{[OFFSET4]}} チャンネル4のオフセット開始,
    \textcolor{black}{\textbf{[CLK]}} CLOCKソースCVを選択,
    \textcolor{black}{\textbf{[LEN]}} ステップの長さを調整する,
    NA,
    NA,
  }
}



\def\modnameshorteuc{EUC}
\def\modnamelongeuc{Euclidean Rhythm Generator}

\newcommand{\modeuc}{
  \newpage
  \modmakesection{euc}

  \paragraph*{}
  Euclidean Rhythm Generator(EUC) は、ユークリッドのアルゴリズムに基づいてリズムパターンを生成し、期間を均等に分割します。

  \showrefcardsection{euc}

  \subsection{操作}
  \paragraph*{} ポリリズムは、パターンの長さを均等なステップに分割することにより、4 チャンネルのリズムを生成します。パターンは、指定された BPM (拍) での小節数 (4 拍または 4 分音符) をカバーする期間として定義されます。チャンネルのステップ数は、Inputs:Beats X (X はチャンネル 1 ~ 4 の番号) で制御できます。
  \newline$\blacksquare$ クロックソースがない場合、BPM は 120 BPM に固定されます。
  
  \newpage
  \section*{\modname}
  \tableofoptionsnew{\eucset}{\optnameset}
  \tableofoptionsnew{\eucinp}{\optnameinp}
  \tableofoptionsnew{\eucout}{\optnameout}
}

