\section{Patch}\label{section:oppatch}%
\subsection{概要}
パッチ別名プリセットは、サウンドを構築できる The Centreの基本操作モードです。 The Centre を起動すると、\textbf{Patch Edit} メニューが表示され、個々のモジュールを追加、削除、および制御できます。
\subsection{基本操作}
Patch Edit のメイン画面には、モジュールのリストが含まれています。
\newline$\blacksquare$ The Centre のモジュールは、\textbf{Patch Edit} メニューにある順序で実行されます。
\newline$\blacksquare$ PatchはModule OUTを含みます。 \nameref{section:modout}
\begin{figure}[h]
  \centering
  \includegraphics[width=0.35\linewidth]{pch1.png}
      \caption{パッチエディットのメインメニュー}
      \label{fig:oppatcheditmain}
  \end{figure}

\paragraph*{}\textbf{[Manage]} ボタンを押すことで、操作モードをパッチ管理に変更します。
  \begin{figure}[h]
    \centering
    \includegraphics[width=0.35\linewidth]{pch3.png}
    \caption{パッチ管理メニュー}
    \label{fig:oppatcheditmanage}
  \end{figure}

\subsubsection{パッチの初期化}
\paragraph*{}\textbf{[Init]}\textbf{[Init]} ボタンをクリックすると、システム初期化ダイアログが開き、センターを初期状態に設定し、作業モードを選択できます。 The Center の 2 つの作業モードは、\textbf{Patch} とマルチパッチ (別名 Set) で利用できます\underline{\nameref{section:multipatch}} 
  
  \begin{figure}[h]
    \centering
    \includegraphics[width=0.35\linewidth]{set2.png}
    \caption{パッチまたはマルチパッチの初期化}
    \label{fig:oppatcheditmanage}
  \end{figure}
  
  \FloatBarrier
  


  \subsubsection{モジュールの追加と削除}
\paragraph*{}\textbf{[Add]} ボタンを押すことでモジュールを追加できます。 モジュールは、選択したモジュールの前に挿入されます。 \textbf{[Del]} を押すと、選択したモジュールが削除されます。

モジュールを追加するには、\textbf{[Add]} ボタンを押してモジュール選択画面を開き、\textbf{[SELECT]} エンコーダーでモジュールを選択すると、モジュールがパッチ エディターの選択された位置に挿入されます。

\begin{figure}[hbt!]
  \centering
  \begin{tabular}{ll}
    \includegraphics[width=0.35\linewidth]{pch4.png}
    &
    \includegraphics[width=0.35\linewidth]{pch5.png}
  \end{tabular}
  \caption{Addを押してモジュールを追加すると、選択した位置にモジュールが挿入されます}
  \label{fig:oppatcheditaddmod}
\end{figure}
\FloatBarrier


\subsubsection{}


\newpage
\clearpage
\section{Multi-Patch 別名 Set}\label{section:multipatch}%
\paragraph*{}%
\subsection{概要}%
\textbf{Set} とも呼ばれるマルチパッチは、The Centre のマルチティンバー機能に簡単にアクセスできるようにするための特別なモードです。 The Centreは(コンピュータリソースから)複数の音声を実行する可能性があるため、マルチティンバー音声生成を提供できるように、4 つのゲート、4 つの V/OCT、および 4 つのオーディオ出力で設計されています。
\subsection{操作}%
マルチパッチ モードに切り替えるには、パッチ メニューに移動し、\textbf{Manage -> Init} を選択し、選択ダイアログから \textbf{Set} を選択します。
\begin{figure}[h]
  \centering
  \begin{tabular}{ll}
  \includegraphics[width=0.35\linewidth]{set1.png}
  &
  \includegraphics[width=0.35\linewidth]{set2.png}
  \end{tabular}
  \caption{左: Manageサブメニューから Initを選択します。 右: ダイアログボックスからSetを選択}
  \label{Fig:initset}
\end{figure}
\subsubsection{Setメニューの操作}
マルチパッチ (Set) モードでモジュールを起動した後、ユーザーには以下の画面が表示されます。

\begin{figure}[h]
  \centering
  \includegraphics[width=0.35\linewidth]{set3.png}
      \caption{Setオプションとマルチパッチメニュー}
      \label{fig:setmain}
  \end{figure}
\paragraph*{}%
一番上の行にはセットの名前が含まれており、カーソルが一番上の行に置かれると、ボタンの割り当てが変更され、Save Set、Init、および Auto のオプションが表示されます。

ユーザーは \textbf{[Save]} および \textbf{[Save As]} を選択してセットを保存できます。
\newline$\blacksquare$ Set は VXS ファイル形式で保存され、Patch は VXP ファイル形式で保存されます。こちらを参照 \underline{\nameref{section:fileformats}}

\FloatBarrier

\subsubsection*{}

マルチパッチメニューにはパッチ用の4つのスロットがあり、VXP 形式の標準パッチ \underline{\nameref{section:patch}} を SD カードからロードしたり、Patch Edit メニューで編集したりできます。
エンコードを回転させ、4 つのスロットのいずれかで \textbf{[SELECT]} を押すと、パッチ\underline{\nameref{section:patch}} メニューに入り、選択したスロットのパッチを編集できます。

\begin{figure}[htb!]
  \centering
  \includegraphics[width=0.35\linewidth]{set4.png}
  \caption{各スロットで使用可能なマルチパッチ メニュー オプション}
  \label{fig:setpatch}
\end{figure}

\FloatBarrier

\paragraph*{}%
一番上の行にはセットの名前が含まれており、カーソルが一番上の行に置かれると、ボタンの割り当てが変更され、Save Set、Init、および Auto のオプションが表示されます。

\FloatBarrier
