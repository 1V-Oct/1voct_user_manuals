% \begin{filecontents*}{\jobname.dat}
%   {(1,1)}/{(3,3)},
%   {(-1,1)}/{(-3,3)},
%   {(1,-1)}/{(3,-3)}%
%   \end{filecontents*}
  
%   \documentclass[border=5mm]{standalone}
%   \usepackage{tikz}
%   \usepackage{catchfile}
  
%   \newcommand\loaddata[1]{\CatchFileDef\loadeddata{#1}{\endlinechar=-1}}
  
%   \begin{document}
  
%   \begin{tikzpicture}
%   \loaddata{\jobname.dat}
%   \foreach \from/\to in \loadeddata{\draw [->] \from -- \to;}
%   \end{tikzpicture}
  
%   \end{document}
% \newcommand{\showsettings}[1] {
% {\renewcommand\arraystretch{1.25}
% \begin{tabularx}{\textwidth}{|c||X|}
% \textcolor{blue}{qrcap.Serie}& \textit{Class} \\ \hline \hline
% \multirow{5}{*}{Parameters:} & \textbf{Audio Output} : \textit{Select Audio Output of Module} \\
%                              & \multicolumn{1}{@{\quad\quad}X}{Description 1.} \\
%                              & \textbf{Audio Input} : \textit{pandas.DataFrame} \\ 
%                              & \multicolumn{1}{@{\quad\quad}X}{Description2 Description2 Description2 Description2  Description2}   \\ \hline
% \end{tabularx}}
% }

%  shit below probably better as new command

% \begin{filecontents*}[overwrite]{set.dat}
%     {Audio Output}/{Selects Audio Output for the module}/{Audio Output can be directed into VOUT outputs or into internal VBuf Virtual Buffer},
%     {(-1,1)}/{(-3,3)},
%     {(1,-1)}/{(3,-3)}%
% \end{filecontents*}

% \newcommand\loaddata[1]{\CatchFileDef\loadeddata{#1}{\endlinechar=-1}}

\makeatletter
    \newread\dataread
    \def\loaddata#1{%
        \gdef\loadeddata{}%
        \openin\dataread=#1
        \begingroup\endlinechar=-1
        \@whilesw\unless\ifeof\dataread\fi{%
            \read\dataread to \dataline
            \edef\rslt{\noexpand\g@addto@macro\noexpand\loadeddata{\dataline}}%
            \rslt%expanded argument
        }%
        \endgroup
        \closein\dataread
    }%
\makeatother
% \documentclass[pt,12pt]{article}
    \makeatletter
    \newcommand*{\compress}{\@minipagetrue}
    \makeatother
%     \begin{document}

\def\ifempty#1{\def\temp{#1} \ifx\temp\empty }

\renewcommand{\tabularxcolumn}[1]{>{\compress\arraybackslash}m{#1}}

\newcommand{\tableofoptions}[1]{
\begin{center}
    \sffamily
    % \begin{tabularx}{\linewidth}{@{}lX@{}}%
    %     \textcolor{blue}{qrcap.Serie} & \textit{Class} \\ \hline \hline
    % \end{description}

    \foreach \settype/\dataset in {{Settings}/{#1/set.dat}, {Inputs}/{#1/inp.dat}, {Outputs}/{#1/out.dat}} {
        % \\[2mm]
    \begin{tabularx}{\linewidth}{@{}>{\hsize=0.2\hsize \columncolor{gray!10}[0pt]}XX@{}}%
    % \hline \hline
    % \\[12mm]
        % \begin{tabularx}{\linewidth}{|X|X|X|}
    % \begin{tabularx}{\linewidth}{}%
     \settype & \smallskip 
    \begin{description}[font =\sffamily, after=\vspace*{-\topsep}]
        \IfFileExists{\dataset}{
        \loaddata{\dataset}
        \foreach \param/\descshort/\desclong in \loadeddata{\item[\param :] \textit{\descshort} \newline\desclong}
        }{\item[] N/A \newline}
    % {\hline \hline
    %     % \begin{tabularx}{\linewidth}{|X|X|X|}
    % % \begin{tabularx}{\linewidth}{}%
    %  \settype & \smallskip 
    % \begin{description}[font =\sffamily, after=\vspace*{-\topsep}]
    %     \loaddata{\dataset}
    %     \foreach \param/\descshort/\desclong in \loadeddata{\item[\param :] \textit{\descshort} \newline\desclong}
    % }
        % draw [->] \from -- \to;}
    % \item[parameter1 :] \textit{string} \newline
    % Description 1.
    % \item[parameter2 :]\textit{pandas.DataFrame} \newline
    % Description2.
    \end{description}
\end{tabularx}
}
\end{center}
}

\ExplSyntaxOn
  \cs_set_eq:NN \expandx \exp_args:Nx
\ExplSyntaxOff



\def\pagesdir{pages}

% \newcommand{\setrecord}[3]{#1\slash#2\slash#3}

    % \immediate\write18{mkdir \pagesdir/\modulename}
\def\setaudiooutput{{Audio Output}/{Selects Audio Output for the module}/{Audio Output can be directed into VOUT outputs or into internal VBuf Virtual Buffer}}
\def\inpfreqcoarse{{Frequency Coarse}/{Adjust Frequency of oscillator (coarse)}/{Adjusts frequency of oscillator in big steps +/- one octave}}
\def\inpnote{{Note}/{Change note input}/{Note defines MIDI note that will be converted to pitch of oscillator. VOCT is the CV input of a note, OCT modifies Octave of note in range +/- 8 octaves, Note tunes note in semitone steps (up to +12 semitones - one octave), Fine tunes note in cents in range of +/- semitone}}
\def\xxxset{
    {NA}/{}/{}%
}



\def\wtoset{
    \setaudiooutput,
    {Polyphony}/{Enables number of polyphony voices from WTO}/{Polyphony turns on WTO into polyphonic mode and up to 4 channels of polyphony are achievable. By turning Polyphony on, the VOCT channels used to control notes will be taken from the next one.},
    {Glide Scaled}/{}/{},
    {Warp Mode}/{}/{},
    {Smoothing}/{}/{}%
}

\def\wtoinp{
    {Unison}/{Number of unison voices oscillators}/{Turns on unison of voices and their pitch can be detuned by Detune parameter},
    {Detune}/{Detune oscillators in Unison mode}/{Every oscillator in Unison mode will be detuned by fraction of Detune parameter. Detune can detune a note by 1 semitone. At full detune, when there are two oscillators one will be at previous semitone of tuning frequency and anothe at the next semitone of tuning frequency},
    {WT}/{Wavetable pointer}/{Adjust pointer of current waveform in the wavetable}/
    {Reset}/{Reset input to reset oscillator phase}/{Upon trigger or gate high the oscillator phase will get reset},
    {AM}/{Amplitude Modulation or output Level}/{AM input modulates amplitude of oscillator and can be used as output level or by connecting modulation signal it turns WTO into ring modulator.},
    {Detune WT}/{Detune Wavetable Pointer}/{Detunes Wavetable pointer in Unison mode so two or more oscillators can run different waveforms from same wavetable.},
    {Frequency Coarse}/{Adjust Frequency of oscillator (coarse)}/{Adjusts frequency of oscillator in big steps +/- one octave},
    {Frequency Fine}/{Adjusts pitch of oscillator in small steps +/- semitone},
    {Glide Time}/{Time to glide between notes}/{},
    {Warp Mode}/{Special mode depending on selected function}/{}%
}
\def\envinp{
    {Attack}/{Value of Attack in range 0s-32s (Level-Attenuator-CV)}/{Value of Attack of AHDSR Envelope, set to small time for punch envelopes},
    {Hold}/{Value of Hold in range 0s-32s (Level-Attenuator-CV)}/{Value of Hold of AHDSR Envelope, Keeps sound at highest level for period of time after raising attack, set to 0s for ADSR envelopes},
    {Decay}/{Value of Decay in range 0s-32s (Level-Attenuator-CV)}/{Decays is a time that elapses from soundreaching maximum volume level (after A and H stages) till it decresases in volume to Sustain level},
    {Sustain}/{Value of Sustain as volume level when AHD stages passed but sound still has Gate ON (Level-Attenuator-CV)}/{Value of Sustain of AHDSR Envelope, Keeps sound at highest level for period of time after raising attack, set to 0s for ADSR envelopes},
    {Release}/{Value of Hold in range 0s-32s (Level-Attenuator-CV)}/{Value of Hold of AHDSR Envelope, Slowly fades sound out. Short Release time make sound end rapidly}%
}

%     \begin{filecontents*}[overwrite]{\pagesdir/\modulename/out.dat}
%         {ENV}/{Envelope}/{CV Output of current envelope value}%
%     \end{filecontents*}

\def\wtoout{
    {OSC}/{Output of oscillator}/{Ouputs current level of oscillator}/{Level of oscillator can be used as envelope for side chaining},
    {RST}/{Phase Reset}/{Outputs trigger every time phase pases through zero}%
}

\newcommand{\wtoinfo}{{\wtoset}/{\wtoinp}/{\wtoout}}


% \newcommand{\vcoinfo}{{\vcoset}/{\vcoinp}/{\vcoout}}
% \newcommand{\lfoinfo}{{\vcoset}/{\vcoinp}/{\vcoout}}



% \newcommand{\emptyinfo}{{\xxxset}/{\xxxset}/{\xxxset}}
% \newcommand{\vcoinfo}{{\vcoset}/{\xxxset}/{\xxxset}}
% \newcommand{\lfoinfo}{{\xxxset}/{\xxxset}/{\xxxset}}

\NewDocumentCommand{\infotype}{m}
 {
  \use:c { info #1 }
 }
\ExplSyntaxOff


\newcommand{\setshowall}[1]{
    \parsevars{#1}
    % #1
    \begin{center}
        \sffamily
        \expandx\tableofoptionsnew{\varone}{Settings}
        \expandx\tableofoptionsnew{\vartwo}{Inputs}
        \expandx\tableofoptionsnew{\varthree}{Outputs}


    \end{center}
}


% \newcommand{\setshowall}[3]{
%     % \parsevars{#1}
%     % #1
%     \begin{center}
%         \sffamily
%         \expandx\tableofoptionsnew{#1}{Settings}
%         \expandx\tableofoptionsnew{#2}{Inputs}
%         % \e/xpandx\tableofoptionsnew{#3}{Outputs}
%     \end{center}
% }

% \setshowall{\wtoinfo}
% \setshowall{\wtoset}{\wtoinp}{\wtoout}
% \newcommand{\infotype}[1]{\csname info#1\endcsname}
% \setshowall{\infotype{wto}}

% \setshowall{\wtoset}{\wtoinp}{\wtoout}
% {\wtoinp}{\wtoout}


% {
%     \def\modulename{wto}
%     \immediate\write18{mkdir \pagesdir/\modulename}
    
%     \begin{filecontents*}[overwrite]{\pagesdir/\modulename/set.dat}
%         {\setaudiooutput},
%         {Polyphony}/{}/{},
%         {Glide Scaled}/{}/{},
%         {Warp Mode}/{}/{},
%         {Smoothing}/{}/{}%
%     \end{filecontents*}

%     \begin{filecontents*}[overwrite]{\pagesdir/\modulename/inp.dat}
%         {\setaudiooutput}%
%     \end{filecontents*}

%     \begin{filecontents*}[overwrite]{\pagesdir/\modulename/out.dat}
%         {ENV}/{Envelope}/{CV Output of current envelope value}%
%     \end{filecontents*}
    
%     \tableofoptions{\pagesdir/\modulename}
% }
% %     \end{document}

% {
%     \def\modulename{env}
%     \immediate\write18{mkdir \pagesdir/\modulename}

%     \begin{filecontents*}[overwrite]{\pagesdir/\modulename/set.dat}
%         {\setaudiooutput}%
%     \end{filecontents*}

%     \begin{filecontents*}[overwrite]{\pagesdir/\modulename/inp.dat}
%         {Attack}/{Value of Attack in range 0s-32s (Level-Attenuator-CV)}/{Value of Attack of AHDSR Envelope, set to small time for punch envelopes},
%         {Hold}/{Value of Hold in range 0s-32s (Level-Attenuator-CV)}/{Value of Hold of AHDSR Envelope, Keeps sound at highest level for period of time after raising attack, set to 0s for ADSR envelopes},
%         {Decay}/{Value of Decay in range 0s-32s (Level-Attenuator-CV)}/{Decays is a time that elapses from soundreaching maximum volume level (after A and H stages) till it decresases in volume to Sustain level},
%         {Sustain}/{Value of Sustain as volume level when AHD stages passed but sound still has Gate ON (Level-Attenuator-CV)}/{Value of Sustain of AHDSR Envelope, Keeps sound at highest level for period of time after raising attack, set to 0s for ADSR envelopes},
%         {Release}/{Value of Hold in range 0s-32s (Level-Attenuator-CV)}/{Value of Hold of AHDSR Envelope, Slowly fades sound out. Short Release time make sound end rapidly}%
%     \end{filecontents*}

%     \begin{filecontents*}[overwrite]{\pagesdir/\modulename/out.dat}
%         {ENV}/{Envelope}/{CV Output of current envelope value}%
%     \end{filecontents*}

%     \tableofoptions{\pagesdir/\modulename}
% }

% {
%     \def\modulename{zzz}
%     \tableofoptions{\pagesdir/\modulename}
% }

% {
%     \tableofoptionsnew{
%         {Audio Output}/{Selects Audio Output for the module}/{Audio Output can be directed into VOUT outputs or into internal VBuf Virtual Buffer},
%         \setaudiooutput,
%         {Polyphony}/{}/{},
%         {Glide Scaled}/{}/{},
%         {Warp Mode}/{}/{},
%         {Smoothing}/{}/{}%
%     }{
%         {ENV}/{Envelope}/{CV Output of current envelope value}%
%     }{
%         {ENV}/{Envelope}/{CV Output of current envelope value}%
%     }
% }