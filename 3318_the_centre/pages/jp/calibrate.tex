\section{Calibration}


The Centre のキャリブレーションについての動画はこちら。

% \href{https://www.youtube.com/watch?v=uEFr7RkuP7k}{https://www.youtube.com/watch?v=uEFr7RkuP7k}
\begin{center}
  \centering\textbf{\textit{\dl{Click on the link or scan QR Code to watch video}{下記のQRコードを読み込んで下さい}}}
\end{center}
\qrdlink
[このビデオでは、The Centreのキャリブレーション方法を示しています]
{https://www.youtube.com/watch?v=uEFr7RkuP7k}
{The Centreのキャリブレーション}

\subsection{なぜキャリブレーションが必要?}
すべての電子部品には許容誤差があります。これらは通常、パーセンテージで表示され、抵抗の場合は \%1、コンデンサの場合は \%5 のようになります。各回路には複数の電子部品が含まれており、これらの部品の許容誤差が合計されるため、回路全体の許容誤差の割合が非常に大きくなります。回路が動作しているとき、その許容誤差は変化しません。許容差は、製造段階でその構成を調整します。したがって回路は常に安定していますが、わずかにずれが生じる可能性があります。
\newline\newline
これらに対処する方法は複数ありますが、一番取り組みやすいのがキャリブレーションと言えます。
\newline
\newline
キャリブレーションにより、制御された環境下におけるデフォルト値、つまり基準点を設定できます。

\subsection{楽器のキャリブレーション}

多くの楽器にはキャリブレーションが必要です。 ピアノやギターの調律も。 The Centreのようなデジタルまたはハイブリッドモジュールの場合、キャリブレーションプロセスでは、2つ以上の基準点で電圧の安定した値を提供する、適切にキャリブレーションされたピッチソースを接続します。通常は 2 点で十分です。 ソフトウェアはすべての値を再計算し、適切なアルゴリズムを適用して常に電圧の正しいピッチを生成します。

\subsection{The Centreのキャリブレーション(動画を観ながら進めるとより理解しやすいです)}

センターには、2V で区切られた 2 つの電圧が必要です。 言い換えると、センターには 2 オクターブ離れた 2 つの C ノートのコントロール ボルテージ (CV) が必要です。 理想的には C1 と C3 ですが、現在の MIDI キーボードの多くは、C2 と C8 に変換される 0V から 5V の間の電圧しか供給しません。 V/Oct の電圧を音符に割り当てる基準はないので、1V/OCT では C2 が 0V であると仮定します。 いずれにせよ、すべての V/Oct 入力にはオクターブとノートの補正があるため、問題ありません。
\newline
\newline
V/OCT 入力を調整するには、中央の 2 つのボタン (ボタン 2 + 3) を押すと、システム メニューが表示されます。 そこから「Calibrate」を選択し、エンコーダーつまみを押して選択します。
\newline
\newline
すべてのチャンネルを個別に、または4つ同時にキャリブレーションできるようになりました。 エンコーダーを使用して、チャンネルまたはすべてのチャンネルを選択します (すべてのチャンネルをキャリブレーションする場合は、スプリッターを使用して、キャリブレーション (ピッチ) 用の CV 電圧をすべての入力に送信します)。
\newline
\newline
そこからは画面の指示に従ってください。 最初にノート C 以外のノートをキーボードから送信し(Cノートを基準に全体をキャリブレーションします)、Start (ボタン 1) を押します。 次に、オクターブ下の C ノート (C3 としましょう) を押して、次の指示を 10 秒間待ち、2 オクターブ上に移動してノート C5 を送信します。 10 秒待つと、ユニットが完全に校正されます。
\newline
\newline
キャリブレーションを保存して完了します。