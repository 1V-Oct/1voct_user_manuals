% v3.4
\section{Calibration}
Quick Start: For a visual demonstration of the calibration process for The Centre, refer to the following video:
\qrdlink
[This video demonstrates the calibration procedure for The Centre]
{https://www.youtube.com/watch?v=uEFr7RkuP7k}
{Calibration of The Centre}
\subsection{Why Calibrate?}
In an ideal scenario, electronic circuits perform perfectly. However, real-world components introduce imperfections, quantified by manufacturing tolerances. These tolerances, typically expressed as percentages (e.g., 1\% for precision resistors and 5\% for standard capacitors), accumulate across multiple components within a circuit, resulting in an overall deviation from the intended performance.
Once manufactured, these tolerances remain fixed, ensuring circuit stability during operation but potentially causing slight offsets from nominal values.
Various techniques address these non-idealities, ranging from costly hardware solutions (e.g., precision component selection) to more economical software-based methods, such as calibration.
Calibration establishes reference values from the circuit's output under controlled conditions, enabling compensation for tolerances and accurate operation.
\subsection{Calibration of Musical Instruments}
Calibration is essential for many musical instruments, including the tuning of acoustic ones like pianos and guitars. For digital or hybrid modules such as The Centre, calibration involves connecting a stable, reference pitch source to provide precise voltages at two or more defined points. Typically, two reference points suffice. The module's firmware then computes scaling factors and applies corrective algorithms to ensure accurate pitch tracking across the voltage range.
\subsection{Calibrating The Centre}
The Centre requires two reference voltages separated by at least 2V (two octaves), corresponding to control voltages (CV) for two C notes spanning two octaves in a 1 V/octave system. Ideally, these would be C1 (e.g., -2 V) and C3 (0 V), but many MIDI keyboards output CV in the 0--5 V range, mapping to C0 through C5. Note that no universal standard exists for voltage-to-note assignment in V/oct systems; here, we assume C0 at 0V. Regardless, each V/oct input includes independent octave and note offset adjustments.
To initiate V/OCT input calibration, press the two central buttons (buttons 2 and 3) to access the System menu. Navigate to "Calibrate" and confirm by pressing the encoder knob.
Channels can be calibrated individually or simultaneously (up to four inputs). Use the encoder to select the desired channel(s); for multi-channel calibration, employ a signal splitter to distribute the reference CV to all inputs.
Follow the on-screen prompts. First, send a non-C note from your MIDI keyboard and press Start (button 1) to establish a baseline. Next, trigger a low-octave C note (e.g., C1) and hold for 10 seconds. The display will then prompt for the upper reference: advance two octaves (e.g., to C3) and hold for another 10 seconds. Calibration is now complete.
Save the calibration data via the menu to apply it persistently, ensuring precise pitch tracking for your module.