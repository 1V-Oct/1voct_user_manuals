\def\envset{
  {{Audio Output}/{Output of ENV}/{Audio Output of processed signal by ENV. Signal has amplitude modulated by sum of two modulators: Level and Sidechain. \newline See: \underline{\nameref{section:audiooutputs}}}},
  {{Audio Input}/{Audio signal to be modulated}/{Audio signal (can be any signal) that will have amlitude (level) modulated with modulator Input: Level}},
  {{Mode}/{Envelope Mode}/{Mode switch between AHDSR (with Sustain) and AHDR (no sustain) envelopes
    \makesupopt{{
      {{Gate}/{The envelope lasts as long as the gate signal is open (high) and sustains volume at the Sustain level}},
      {{Trigger}/{The enevelope only lasts as long as sum of stages AHDR and never keeps note sustained}}%
    }}
  }}%
}

\def\envinp{
  {{Attack}/{Attack Stage}/{Attack is the raising first stage of envelope. Duration between 0s and 32s}},
  {{Hold}/{Hold Stage}/{Hold is the full volume level sustained second stage of envelope. Duration between 0s and 32s}},
  {{Decay}/{Decay Stage}/{Decay is the falling third stage of envelope going from full level of volume to level set . Duration between 0s and 32s}},
  {{Sustain}/{Sustain Stage}/{Sustain is the level of the sound after AHD stages at which sound is kept as long as Gate is open (high). Level of sound in decibels}},
  {{Release}/{Release Stage}/{Release is the final stage where the sound decays to level 0. Duration between 0s and 32s}},
  {{Gate}/{Gate CV Source}/{Gate or Trigger signal to start and sustain envelope generation}},
  {{Attack Curve}/{Exponent of Attack stage}/{Attack Curve is the exponent of the attack stage to modify stage to exponential rather than linear}},
  {{Decay Curve}/{Exponent of Decay stage}/{Decay Curve is the exponent of the Decay stage to modify stage to exponential rather than linear}},
  {{Release Curve}/{Exponent of Release stage}/{Release Curve is the exponent of the Release stage to modify stage to exponential rather than linear}}%
}

\def\envout{
  {{Envelope}/{Output of Envelope}/{Output of Envelope signal to be routed to other modules as a Control Voltage}}%
}

\def\refcardenv{
  \makereferencecard{
    \textcolor{black}{\textbf{[ROTATE]}} Switches between waveforms (Triangle Square Sine),
  }{
    \textcolor{black}{\textbf{[ON/OFF]}} Turning module On or Off (Bypass),
    \textcolor{black}{\textbf{[SET]}} Settings of the module,
    \textcolor{black}{\textbf{[Inputs]}} Configure Inputs of the module,
    \textcolor{black}{\textbf{[Back]}} Return back to Patch Menu,
  }{
   Adjust ATTACK in range of 0.0 to 31.5s,
   Adjust HOLD in range 0 to 31.5s. \newline$\blacksquare$ When HOLD is 0s envelope is ADSR,
   Adjust DECAY time in range of 0.0s to 31.5s,
   Adjust SUSTAIN level,
   Adjust RELEASE time in range of 0.0s to 31.5s,
   NA,
   NA,
   NA,
  }{
   Adjust curve shape of ATTACK,
   NA,
   Adjust curve shape of DECAY,
   Adjust curve shape of RELEASE,
   \textcolor{black}{\textbf{[GAT]}} Select GATE Input CV,
   \textcolor{black}{\textbf{[MODE]}} Change between Trigger (AHDR) and Gate (AHDSR) modes,
   NA,
   Select AUDIO Output of Envelope (allows to send envelope out to control other devices),
  }
}

\def\modnameshortenv{ENV}
\def\modnamelongenv{Envelope Generator}

\newcommand{\modenv}{
  \newpage
  \modmakesection{env}

  \paragraph*{}Envelope Generator produces a modulation signal that in its substance is of raise and decay at the given rates.
  
  \showrefcardsection{env}

  \subsection{Operation}
  Envelope Generator produces AHDSR (Attack, Hold, Decay, Sustain and Release) or AHDR envelopes that are Control Voltage signals to modulate parameters of modules in the synthesiser (usually, but not limited to, Amplitude of ENV and Cutoff Frequency of VCF).
  \subsection{Envelope Stages}
  \makesupopt{{
    {{A: Attack}/{Attack stage is first and the raising stage of signal from 0 to 100\% volume of signal during given period of time.}},
    {{H: Hold}/{Hold stage keeps signal at the highest volume level for a given period of time}},
    {{D: Decay}/{Decay stage determines the length of time the sound volume will be dropping from \textbf{Hold} level to \textbf{Sustain} level. The Decay stage length is given as period of time.}},
    {{S: Sustain}/{Sustain stage determines the volume level at which sound will keep playing before the gate signal that is relative to played note length keeps high signal. Sustain stage unlike other parameters of envelope is not given in duration of time but rather as a level of volume.}},
    {{R: Release}/{Release is the final stage of sound when it either slowly decays or rapienv ends. Stage is defined as period of time. }}%
  }}%
  \section*{\modname}

$\bigstar$ If the \textbf{Gate} signal length (played note length) is shorter than combined length of \textbf{Atack} + \textbf{Hold} + \textbf{Decay} then some of those stages will be skipped all the way to Sustain stage as the Sustain stage starts when \textbf{Gate} signal becomes low. Therefore if \textbf{Trigger} signal is applied instead of \textbf{Gate} signal then all stages (A+H+D) will be skipped and the envelope will be limited to only Release stage.

  \subsection{Envelope Types}
  The most popular envelops out there are ADSR (Attack, Decay, Sustain and Release) - this is envelope designed for Gate signal of variable length that is relative to length of played note. The envelope upon raise of gate signal will execute Attack and Decay stages and will keep amplitude at Sustain level until gate signal turns low (note length ends) and the release stage will help amplitude to continue decay.
  On the contrary, AD envelopes are designed for Trigger signal and they execute both Attack and Decay stage regardless of note length. The length of note is dictated by length of Attack and Decay stages.
  \paragraph*{} \textbf{Hold} stage in \textbf{AHDSR} envelope is just extension of typical ADSR envelope where the maximum raised stage (after \textbf{Attack}) can be sustained for period of time. With Hold stage period length set to 0s AHDSR envelope becomes standard ADSR envelope.
  
  \subsection{Trigger Mode}
  Envelopes can be generated in two modes which can be activated by switching Setting:Mode btween \textbf{Gate} and \textbf{Trigger}.
  \makesupopt{{
    {{Gate}/{In this mode the envelope is based on the length of provided gate signal being high. When the gate is in high mode the envelope processes through AHD stages and stays in Sustain stage until gate closes (signal turns low) and the Release stage is being processed}},
    {{Trigger}/{In this mode the envelope is always fully processed however without Sustain element. The envelope is beign triggered by high Trigger (Gate) signal and regardless of length of that signal will execute AHDR stages and the Sustain part will be ommited.
    \newline$\bigstar$ This mode is ideal to emulate older AD envelopes by setting stages A and D to desired values and every other stage to 0}}%
  }}%
  % \newpage
  \section*{\modname}
  \tableofoptionsnew{\envset}{{Settings}}
  \tableofoptionsnew{\envinp}{{Inputs}}
  \tableofoptionsnew{\envout}{{Outputs}}
}

