\def\plyset{
  {{Pattern Length}/{Length of pattern in bars}/{Length of pattern in bars. Each pattern is divided into number of steps configured by Inputs:Pulses}},
  {{Gate length}/{Length of gate}/{
    \makesupopt{{
      {{Trigger}/{Trigger only}/{Unsustained signal when the pulse occurs}}%
    }}%
  }}%
}

\def\plyinp{
  {{Clock}/{Clock CV Source}/{Clock source to synchronise Poly Rhythm with other modules.}},
  {{Reset}/{Reset CV Source}/{Reset position for all channels in pattern upon trigger}},
  {{Pulses 1}/{Number of pulses for channel 1}/{Number of steps that the pattern gets divided into and generating a beat at the beginning of each step}},
  {{Pulses 2}/{Number of pulses for channel 2}/{Number of steps that the pattern gets divided into and generating a beat at the beginning of each step}},
  {{Pulses 3}/{Number of pulses for channel 3}/{Number of steps that the pattern gets divided into and generating a beat at the beginning of each step}},
  {{Pulses 4}/{Number of pulses for channel 4}/{Number of steps that the pattern gets divided into and generating a beat at the beginning of each step}}%
}

\def\plyout{
  {{Gate 1}/{Gate 1 Output}/{Output of Poly Rhythm setting gate to high upon beat}},
  {{Gate 2}/{Gate 2 Output}/{Output of Poly Rhythm setting gate to high upon beat}},
  {{Gate 3}/{Gate 3 Output}/{Output of Poly Rhythm setting gate to high upon beat}},
  {{Gate 4}/{Gate 4 Output}/{Output of Poly Rhythm setting gate to high upon beat}}%
}

\def\refcardply{
  \makereferencecard{
    NA%\textcolor{black}{\textbf{[ROTATE]}} Turn note on or off and in Scale Selector turn to Scale Select mode
  }{
    \newline\textcolor{black}{\textbf{[RESET]}} Resets all channels in pattern upon release of button,
    \newline\textcolor{black}{\textbf{[SET]}} Settings of the module,
    \newline\textcolor{black}{\textbf{[Inputs]}} Configure Inputs of the module,
    \newline\textcolor{black}{\textbf{[Back]}} Return back to Patch Menu,
  }{
    Slect number of steps for channel 1,
    Slect number of steps for channel 2,
    Slect number of steps for channel 3,
    Slect number of steps for channel 4,
    NA,
    NA,
    NA,
    NA,
  }{%
    \textcolor{black}{\textbf{[PLEN]}} Select pattern length,
    NA,
    NA,
    NA,
    \textcolor{black}{\textbf{[CLK]}} Select CLOCK source CV,
    NA,
    NA,
    NA,
  }
}


\def\modnameshortply{PLY}
\def\modnamelongply{Polyrhythm}

\newcommand{\modply}{
  \newpage
  \modmakesection{ply}  
  \def\modname{PLY - Polyrhythm}% chktex 8
  

  \paragraph*{}
  Polyrhythm (PLY) generates generates contrasting rhythms within a pattern.

  \showrefcardsection{ply}

  \subsection{Operation}
  \paragraph*{} Polyrhythm generates 4 channels of rhythms by dividing pattern duration into equal number of steps decided by input parameter Inputs:Steps. Pattern is defined as period of time covering number of bars (4 beats or 4 quarternotes) at given BPM (beats). Number of steps in channel can be controlled via Inputs:Beats X (where X is the number of channel 1 to 4).
  \newline$\blacksquare$ Without clock source the BPM is fixed at 120 BPM the pattern
  
  \newpage
  \section*{\modname}
  \tableofoptionsnew{\plyset}{{Settings}}
  \tableofoptionsnew{\plyinp}{{Inputs}}
  \tableofoptionsnew{\plyout}{{Outputs}}
}

